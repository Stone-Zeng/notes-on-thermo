%% Copyright (C) 2016--2018 by Xiangdong Zeng <pssysrq@163.com>
%%
%% -* Notes on Thermodynamics and Statistical Physics *-
%%
%% This file may be distributed and/or modified under the
%% Creative Commons Attribution Share Alike 4.0 license.

\PassOptionsToPackage{log-declarations=false}{xparse}
\PassOptionsToPackage{no-math}{fontspec}
\documentclass[UTF8, heading=true, fontset=none, a4paper]{ctexbook}
\usepackage{amsmath,mathtools,physics,unicode-math}
\usepackage[thmmarks, amsmath]{ntheorem}
\usepackage[ntheorem]{empheq}
\usepackage[stable, perpage, bottom]{footmisc}
\usepackage{geometry,fancyhdr,enumitem,graphicx,array,tabularx,booktabs,%
  caption,mhchem,siunitx,hyperref,bookmark}

% 西文字体
\setmainfont{Libertinus Serif}
\setsansfont{Source Sans Pro Semibold}[BoldFont = Source Sans Pro Bold]
\setmathfont{Libertinus Math}[math-style = ISO, bold-style = ISO]

% 中文字体
\setCJKmainfont{Source Han Serif SC}%
  [ItalicFont = FZKai-Z03, BoldItalicFont = FZKai-Z03]
\setCJKsansfont{Source Han Sans SC Medium}%
  [BoldFont = Source Han Sans SC Bold, ItalicFont = FZKai-Z03, BoldItalicFont = FZKai-Z03]
\setCJKmonofont{FZFangSong-Z02}%
  [BoldFont = *, ItalicFont = *, BoldItalicFont = *]
\newCJKfontfamily\kaishu{FZKai-Z03}%
  [BoldFont = *, ItalicFont = *, BoldItalicFont = *]

% 页面尺寸
\geometry{
  hmargin    = 1.0 in,
  vmargin    = 1.5 in,
  headheight = 15 pt
}

% 中文排版样式
%\ctexset{
%  part = {
%    format = {\bfseries \Huge \centering},
%    name   = {第, 篇}
%  },
%  chapter/format = {\bfseries \LARGE \raggedright},
%  section/format = {\bfseries \Large \centering},
%  subsection = {
%    format    = {\bfseries},
%    name      = {,、\hspace{-1 em}},
%    numbering = true,
%    number    = \chinese{subsection},
%  }
%}
%\setcounter{secnumdepth}{4}

%\let\OldSubsection=\subsection
%\RenewDocumentCommand\subsection{sm}
%  {
%    \IfBooleanTF{#1}
%      {\OldSubsection*{#2}\addcontentsline{toc}{subsection}{#2}}
%      {\OldSubsection{#2}}
%  }

% 页眉页脚
\fancyhf{}
\fancyhead[EL]{\small\nouppercase{\kaishu\leftmark}}
\fancyhead[OR]{\small\nouppercase{\kaishu\rightmark}}
\fancyfoot[C]{\small\thepage}
\renewcommand\headrulewidth{0pt}
\pagestyle{fancy}

% 章末空白页
\makeatletter
\renewcommand\cleardoublepage{\clearpage\if@twoside\ifodd\c@page\else
  \hbox{}
  \vspace*{\fill}
  \begin{center}
    \itshape This page is intentionally left blank.
  \end{center}
  \vspace{\fill}
  \thispagestyle{empty}
  \newpage
  \if@twocolumn\hbox{}\newpage\fi\fi\fi}
\makeatother

% 脚注
\makeatletter\ExplSyntaxOn
\cs_set:Npn \thefootnote { $ \my_footnote_symbol:N \c@footnote $ }
\cs_new:Npn \my_footnote_symbol:N #1
  {
    \int_compare:nTF { #1 >= 10 }
      {
        \int_compare:nTF { #1 >= 36 }
          { \symbol { \int_eval:n { "24B6 - 36 + #1 } } }
          { \symbol { \int_eval:n { "24D0 - 10 + #1 } } }
      }
      { \symbol { \int_eval:n { "2460 - 1 + #1 } } }
  }
\cs_set:Npn \@makefntext #1
  {
    \mode_leave_vertical:
    \hbox_to_wd:nn { 1.5 em } { \@thefnmark \hfil }
    #1
  }
\ExplSyntaxOff\makeatother

% 定理类环境
\newenvironment{proof}{\par\mbox{}\par\small}{\par\mbox{}\par}
\begingroup
  \theoremstyle{empty}
  \theoremheaderfont{\sffamily}
  \theorembodyfont{\rmfamily\kaishu}
  \theoremprework{%
    \list{}{%
      \setlength{\rightmargin}{2em}%
      \setlength{\leftmargin}{2em}}%
    \item\relax}
  \theorempostwork{\endlist\vspace*{1ex}}
  \newtheorem{theorem}{}
\endgroup
\begingroup
  \theoremstyle{plain}
  \theoremheaderfont{\sffamily}
  \theorembodyfont{\normalfont}
  \theoremsymbol{}
  \newtheorem{example}{例}[chapter]
\endgroup

% 公式
%\numberwithin{equation}{section}
%\renewcommand{\theequation}{\thesection.\arabic{equation}}
\NewDocumentEnvironment{boxeq}{}
  {\empheq[box=\fbox]{equation}}
  {\endempheq}
\makeatletter\ExplSyntaxOn
\ctex_patch_cmd:Nnn \HyOrg@subequations
  { \def \theequation { \theparentequation \alph { equation } } }
  { \def \theequation { \theparentequation - \alph { equation } } }
\ExplSyntaxOff\makeatother
\NewDocumentEnvironment{braced}{O{}O{align}}
  {\subequations#1
   \setkeys{EmphEqEnv}{#2}
   \setkeys{EmphEqOpt}{left=\empheqlbrace}
   \EmphEqMainEnv}
  {\endEmphEqMainEnv\endsubequations}
\NewDocumentEnvironment{braced*}{O{align*}}
  {\setkeys{EmphEqEnv}{#1}
   \setkeys{EmphEqOpt}{left=\empheqlbrace}
   \EmphEqMainEnv}
  {\endEmphEqMainEnv}

% 杂项
\sisetup{
  number-math-rm = \ensuremath,
  inter-unit-product = \ensuremath{{}\cdot{}},
  group-minimum-digits = 4
}
\hypersetup{
  bookmarksdepth     = 4,
  bookmarksnumbered  = true,
  bookmarksopen      = true,
  bookmarksopenlevel = 1,
  colorlinks         = true,
  hyperfootnotes     = false
}

% 命令
%% 数学字体
\newcommand\V[1]{\symbf{#1}}   % Vecotr
\renewcommand\T[1]{\symbf{#1}} % Tensor
%% 排列组合
\newcommand\Pnum[2]{\symrm{P}^{#2}_{#1}}
\newcommand\Cnum[2]{\symrm{C}^{#2}_{#1}}
%% 数学、物理常数
\newcommand\ee{\symrm{e}}
\newcommand\ii{\symrm{i}}
\newcommand\pp{\symrm{\pi}}
\newcommand\kB{k_{\symrm{B}}} % Boltzmann constant
\newcommand\pfs{\epsilon_0}   % Permittivity of free space
%% 重命名
\newcommand\incr{\increment}
\newcommand\defeq{\equiv}
%% 杂项
\newcommand\dotsint{\int\!\cdots\!\int}
\newcommand\const{\text{const.}}
\newcommand\myTag[1]{}
%% 特殊命令
%% see `physics.sty`
\DeclareDocumentCommand\differential{ogd()}
  {
    \IfNoValueTF{#2}
      {
        \IfNoValueTF{#3}
          {\diffd\IfNoValueTF{#1}{}{^{#1}}}
          {\mathinner{\diffd\IfNoValueTF{#1}{}{^{#1}}\,\argopen(#3\argclose)}} % Add \, here
      }
      {\mathinner{\diffd\IfNoValueTF{#1}{}{^{#1}}#2}\IfNoValueTF{#3}{}{(#3)}}
  }
\def\diffdbar{\mbox{đ}} % U+0111
\DeclareDocumentCommand\dbar{ogd()}
  {
    \IfNoValueTF{#2}
      {
        \IfNoValueTF{#3}
          {\diffdbar\IfNoValueTF{#1}{}{^{#1}}}
          {\mathinner{\diffdbar\IfNoValueTF{#1}{}{^{#1}}\,\argopen(#3\argclose)}} % Add \, here
      }
      {\mathinner{\diffdbar\IfNoValueTF{#1}{}{^{#1}}#2}\IfNoValueTF{#3}{}{(#3)}}
  }
\DeclareDocumentCommand\trigbraces{mod()}
  {
    \IfNoValueTF{#3}
      {#1 \IfNoValueTF{#2}{}{[#2]}}
      {#1 \IfNoValueTF{#2}{}{^{#2}}\,\argopen(#3\argclose)} % Add \, here
  }
%% 文本命令
\newcommand\kwd[1]{\textsf{#1}}
\newcommand\cmark{\symbol{"2714}}
\newcommand\xmark{\symbol{"2718}}
\newcommand\blankline{\mbox{}}
\newcommand\FIGPLACEHOLDER{\includegraphics[draft]{foo}}
%\def\secref#1{\S\ref{#1}}
%\def\subsecref#1{第\ref{#1}小节}
%\def\egref#1{例 \ref{#1}}
%% 句号
\catcode`\。=\active
\let。=.

% 表格
%\newcolumntype{M}{>{$}c<{$}} %数学模式,居中
%\newcolumntype{Y}{>{\centering\arraybackslash}X} %定宽居中

% 浮动体
\captionsetup[figure]{font = small, labelsep = quad}
\captionsetup[table]{font = {small, sf}, labelsep = quad}

% 标题页
\title{\vspace{-4 cm}\bfseries 热力学与统计物理 I}
\author{\kaishu 复旦大学\quad 陈焱}
\date{\kaishu\today}


\begin{document}

%\maketitle
%
\frontmatter
%  \tableofcontents
%  \chapter{绪论}

本课程包含两部分内容:热力学、统计物理。

\kwd{热力学(thermodynamics)}是一种自上而下(top-down)的研究方式,它是
形而上的、唯象的。用一句话可以形容热力学:“知其然而不知其所以然”。它主要研究
宏观物理量之间的关系。

对热力学有重大贡献的物理学家有 Carnot、Joule、Clausius、Kelvin 等。

\blankline

\kwd{统计物理(statistical Physics)}是一种自下而上(bottom-up)的研究方式,
它是形而下的、微观的。

对统计物理有重大贡献的物理学家见表 \ref{tab:physicist-in-statistical-physics}。

\begin{table}[h]
  \centering
  \caption{对统计物理有重大贡献的物理学家}
  \label{tab:physicist-in-statistical-physics}
  \begin{tabular}{cc}
    \toprule
      \textbf{阶段} & \textbf{人物} \\
    \midrule
      经典统计 & Maxwell、Boltzmann、Gibbs、Einstein 等 \\
      量子概念 & Planck、Einstein、Fermi、Dirac、Pauli、Bose 等 \\
      量子统计 & von Neumann、Landau、Kramers、Pauli 等 \\
    \bottomrule
  \end{tabular}
\end{table}

统计物理可分为两个阶段。1860 年至 1902 年,人们主要研究近独立子体系(简单地说,
就是理想气体);1902 年以后,出现系综理论,开始研究凝聚态系统。

统计物理的基础可以概括为\kwd{等概率原理}。


\mainmatter
  %\part{热力学}
    %\chapter{热力学基础}
    %\chapter{热力学基础}

\section{平衡态及其描述} \label{sec:equilibrium-state}

\subsection{热力学系统}

热力学研究的对象是\kwd{热力学系统}(简称系统)。它是宏观体系,粒子数的量级约为
\num{e23}。与系统相对应的是\kwd{外界},也称为环境或热库。按照与外界的关系,
可将系统分为三种:\emph{孤立系}、\emph{封闭系}和\emph{开放系},
见表~\ref{tab:definition-of-systems}。

\begin{table}[h]
  \centering
  \begin{tabular}{cccc}
    \toprule
    & \kwd{孤立系 (isolated)} & \kwd{封闭系 (closed)} & \kwd{开放系 (open)} \\
    \midrule
    物质交换 & \xmark & \xmark & \cmark \\
    能量交换 & \xmark & \cmark & \cmark \\
    \bottomrule
  \end{tabular}
  \caption{三种热力学体系} \label{tab:definition-of-systems}
\end{table}

以后我们将会了解到,这三种系统,分别对应着\emph{微正则系综}、\emph{正则系综}%
与\emph{巨正则系综}。

\subsection{平衡态}

经典热力学研究\kwd{平衡态}。它有两个要素:状态不随时间变化、孤立系。若不是
孤立系,但满足状态不随时间变化的条件,则称\emph{稳恒状态},如日光灯
(显然有能量交换)。

平衡态只说明宏观性质不随时间变化,而微观态仍可有变化(微观粒子不断变化)。
因此称为\kwd{动态平衡}(就微观状态而言,也称为\emph{细致平衡})。

\subsection{平衡态的描述} \label{subsec:平衡态的描述}

平衡态可以利用\kwd{状态变量}(可以测量)来描述,如压强 $p$、体积 $V$、温度 $T$
等。需注意,这些量未必互相独立。因此需要选取\emph{独立}的\kwd{状态参量}。
当某一状态变量可以用其他状态参量来描述时,则称它为一个\kwd{状态函数}。

状态变量可分为两种:\kwd{强度量},如压强 $p$、温度 $T$、密度 $\rho$ 等,
它们不随粒子数的增加而增加;\kwd{广延量},如体积 $V$、内能 $U$、熵 $S$ 等,
它们随粒子数的增加而增加。

\section{温度;状态方程}

\subsection{热平衡定律}

\begin{theorem}[热平衡定律(热力学第零定律)]
  若物体 $A$ 分别与物体 $B$ 和 $C$ 处于热平衡,那么如果让 $B$ 与 $C$ 热接触,
  它们一定也处于热平衡。
\end{theorem}

该定律是温度测量的基础:互为热平衡的物体存在一个属于其固有属性的物理量,
即\kwd{温度}。一切互为热平衡的物体温度相等。

具体确定温度,需要选定\kwd{温标}。除了常见的摄氏、华氏温标,还有热膨胀温标、
热电阻温标、理想气体温标和热力学温标等。

\subsection{物态方程}

温度与其他状态参量 \footnote{即独立的状态变量。}
的函数关系称为\kwd{物态方程}:
\begin{equation}
  T = f(p, V)
\end{equation}
或
\begin{equation}
  g(p, V, T) = 0.
\end{equation}

Boyle 和 Mariotte 分别于 1662 年和 1676 年各自确立了\kwd{Boyle--Mariotte定律}:
\begin{equation}
  p V = p_0 V_0 \qc\qif* T = \const;
\end{equation}
1802年,Gay-Lussac 确立了\kwd{Gay-Lussac定律} \footnote{事实上,在 1787 年,
  Charles 就已经发现了这一定律,只是当时未发表,也未被人注意。}:
\begin{equation}
  \frac{V}{T} = \frac{V_0}{T_0} \qc\qif* p = \const
\end{equation}
综合两式,可得 \footnote{该式中常数的值显然与物质的量 $n$ 有关,故用带引号的
  $\text{``const.''}$ 表示。}
\begin{equation} \label{eq:ideal-gas-law-a}
  \frac{p V}{T} = \frac{p_0 V_0}{T_0} = \text{``const.''} = n R,
\end{equation}
即\kwd{理想气体物态方程}
\begin{boced} \label{eq:ideal-gas-law-b}
  p V = n R T = N \kB T
\end{boced}

\begin{proof}
  式~\eqref{eq:ideal-gas-law-a} 的推导如下。
  设某气体初状态为 $p_0$、$V_0$、$T_0$,末状态为 $p$、$V$、$T$。

  先保持温度 $T_0$ 不变,则有
  \begin{equation} \label{eq:ideal-gas-law-proof}
    p_0 V_0 = p V';
  \end{equation}
  再假设压强 $p$ 不变,于是
  \begin{equation}
    \frac{V'}{T_0} = \frac{V}{T},
  \end{equation}
  即
  \begin{equation}
    V' = \frac{V T_0}{T}.
  \end{equation}
  代入 \eqref{eq:ideal-gas-law-proof}~式,即有
  \begin{equation}
    \frac{p V}{T} = \frac{p_0 V_0}{T_0} = \text{``const.''}.
  \end{equation}
\end{proof}

除了通过统计物理推导,也可通过测量膨胀系数、压强系数和等温压缩系数
(亦可简称压缩系数)来得到物态方程。

\kwd{膨胀系数} $\alpha$ 定义为
\begin{equation}
  \alpha \defeq \frac{1}{V} \qty(\pdv{V}{T})_p,
\end{equation}
\kwd{压强系数} $\beta$ 定义为
\begin{equation}
  \beta \defeq \frac{1}{p} \qty(\pdv{p}{T})_V,
\end{equation}
\kwd{等温压缩系数} $\kappa_T$ 定义为
\begin{equation}
  \kappa_T \defeq -\frac{1}{V} \qty(\pdv{V}{p})_T.
\end{equation}

可以证明,以上三个常数满足下面的关系:
\begin{equation} \label{eq:relation-of-alpha-beta-gamma}
  \alpha = \beta \kappa_T p.
\end{equation}
该式说明三者并非独立。通常会直接测量 $\alpha$ 和 $\kappa_T$,而通过计算得到
$\beta$。

\begin{proof}
  先证明两个结论。若 $x, \, y, \, z$ 满足 $F(x, y, z) = \const$,其中
  $F(x, y, z)$ 是一个可微函数。那么
  \begin{braced}
    \dd{x} &= \qty(\pdv{x}{y})_z \dd{y} + \qty(\pdv{x}{z})_y \dd{z}, \\
    \dd{z} &= \qty(\pdv{z}{x})_y \dd{x} + \qty(\pdv{z}{y})_x \dd{y}.
  \end{braced}
  把第一式代入第二式,得
  \begin{align}
    \dd{z} &= \qty(\pdv{z}{x})_y
      \qty[\qty(\pdv{x}{y})_z \dd{y} + \qty(\pdv{x}{z})_y \dd{z}]
      + \qty(\pdv{z}{y})_x \dd{y} \notag \\
    &= \qty[\qty(\pdv{z}{x})_y \qty(\pdv{x}{y})_z + \qty(\pdv{z}{y})_x] \dd{y}
      + \qty(\pdv{z}{x})_y \qty(\pdv{x}{z})_y \dd{z}.
  \end{align}
  于是
  \begin{equation} \label{eq:coefficient-of-dy-and-dz}
    \qty[\qty(\pdv{z}{x})_y \qty(\pdv{x}{y})_z + \qty(\pdv{z}{y})_x] \dd{y}
    + \qty[\qty(\pdv{z}{x})_y \qty(\pdv{x}{z})_y - 1] \dd{z} = 0.
  \end{equation}
  因为 $y$ 与 $z$ 是独立的变量,所以 $\dd{y}$ 与 $\dd{z}$ 也是独立的。这就要求
  它们的系数均为零。由 $\dd{z}$ 的系数为零,可得到\kwd{倒数关系}:
  \begin{equation} \label{eq:reciprocal-relation}
    \qty(\pdv{z}{x})_y \qty(\pdv{x}{z})_y = 1;
  \end{equation}
  由 $\dd{y}$ 的系数为零,可得
  \begin{equation} \label{eq:cyclic-relation}
    \qty(\pdv{z}{x})_y \qty(\pdv{x}{y})_z = -\qty(\pdv{z}{y})_x.
  \end{equation}
  利用倒数关系式~\eqref{eq:reciprocal-relation},即有\kwd{偏导数三乘积法则}:
  \begin{equation}
    \qty(\pdv{x}{y})_z \qty(\pdv{y}{z})_x \qty(\pdv{z}{x})_y = -1.
  \end{equation}

  根据式~\eqref{eq:cyclic-relation},用热力学中的变量将其写成
  \begin{equation}
    \qty(\pdv{V}{T})_p = -\qty(\pdv{V}{p})_T \qty(\pdv{p}{T})_V.
  \end{equation}
  代入前文中的定义,就得到了 \eqref{eq:relation-of-alpha-beta-gamma}~式:
  \begin{equation}
    \alpha = \beta \kappa_T p.
  \end{equation}
\end{proof}

考虑分子之间的相互作用后,对理想气体进行修正,这就是\kwd{van der Waals 气体}。
其物态方程为
\begin{equation} \label{eq:van-der-waals-gas}
  \qty(p + \frac{n^2 a}{V^2}) (V - nb) = nRT,
\end{equation}
其中 $n^2 a / V^2$ 代表分子之间吸引力所引起的修正,而 $nb$ 则代表排斥力所引起
的修正。

\section{功;热力学第一定律}

\subsection{准静态过程的功}

静态过程即平衡态。

\kwd{准静态过程}指过程中的每一步都是平衡态。这就要求外界的条件变化地足够缓慢。
令 $\tau$ 为外界条件变化的特征时间,$\incr{t}$ 为系统趋于与外界条件对应的平衡态
的特征时间(即\kwd{弛豫时间}),那么准静态过程相当于
\begin{equation}
  \frac{\tau}{\incr{t}} \to 0
\end{equation}
的极限。

在一个过程中,如果每一步都可以在相反的方向进行而不引起外界的变化,则称为%
\kwd{可逆过程}。可逆过程的实质是没有耗散。

下面举两个准静态过程的例子。

第一个例子是 $p$-$\!V$ 系统。考虑盛于带活塞容器内的气体。气体体积变化 $\dd{V}$ 时,
\emph{外界对系统}做功
\begin{equation}
  \dbar{W} = -p \dd{V}.
\end{equation}
这里的 $p$ 指外界对系统的压强。由于是准静态过程,它又等于容器内气体对器壁的
压强。外界压力作用下体积减小,$\dd{V} < 0$;而外界却对系统做正功,因此会出现
负号。如果令 $W'$ 为\emph{系统对外界}做的功,则有
\begin{equation}
  \dbar{W'} = - \dbar{W} = p \dd{V}.
\end{equation}

体积由 $V_1$ 变化到 $V_2$,外界对系统做的总功为
\begin{equation}
  W = -\int_{V_1}^{V_2} p \dd{V}.
\end{equation}

% \begin{figure}[h]
%   \centering
%   \begin{asy}
%     pair O = (0, 0), x_axes = (10, 0), y_axes = (0, 10);
%     
%     pair p1 = (2.2, 7.5), p2 = (8, 2.5), p3 = (4, 4.5);
%     pair p1_x = (p1.x, 0), p2_x = (p2.x, 0);
%     
%     fill(p1_x--p1..p3..p2--p2_x--cycle, color1 + opacity(0.2));
%     
%     draw(Label("$V$", EndPoint), O--x_axes, Arrow);
%     draw(Label("$p$", EndPoint), O--y_axes, Arrow);
%     
%     draw(p1..p3..p2, linewidth(1) + color1);
%     draw(p1_x--p1, dashed + color1);
%     draw(p2_x--p2, dashed + color1);
%     
%     label("$O$", O, SW);
%     label("状态 $1$", p1, (0, 2));
%     label("状态 $2$", p2, (0.5, 2));
%     label("$V_1$", p1_x, S);
%     label("$V_2$", p2_x, S);
%   \end{asy}
%   \caption{$p \text{-} V$ 系统的状态空间}
%   \label{fig:pV-diagram}
% \end{figure}

在 $p$-$\!V$ 图中标出状态 1 和状态 2,则连接它们的曲线就表示一个准静态过程,曲线
下的面积等于 $-W$,如图~\ref{fig:pV-diagram} 所示。显然,$W$ 与路径有关,即不是
一个全微分,因此我们用“$\dbar{W}$”表示微功。

另一个例子是电介质极化系统。考虑均匀电场 $\V{E}$ 中的均匀电介质。电位移 $\V{D}$
变化 $\dd{\V{D}}$ 时,电场做功
\begin{equation}
  \dbar{W} = V \V{E} \cdot \dd{\V{D}},
\end{equation}
其中的 $V$ 表示电介质的体积。这里,$\V{E}$ 是变化的外因,$\V{D}$ 则是内果。
令 $\V{P}$ 为极化强度,则
\begin{equation}
  \V{D} = \pfs \V{E} + \V{P}.
\end{equation}
于是
\begin{equation}
  \dbar{W} = V \pfs \V{E} \cdot \dd{\V{E}} + V \V{E} \cdot \dd{\V{P}}
  = V \dd(\frac{1}{2} \pfs \V{E}^2) + V \V{E} \cdot \dd{\V{P}}.
\end{equation}
式中的第一项代表电场能量的变化,第二项则代表\emph{极化功}。

对以上两种情况进行推广,可得
\begin{equation} \label{eq:work-general}
  \dbar{W} = Y_1 \dd{y_1} + Y_2 \dd{y_2} + \cdots + Y_n \dd{y_n}
  = \sum_{i=1}^{n} Y_i \dd{y_i}.
\end{equation}
这里的 $Y_i$ 是\kwd{广义力},如压强 $p$、电场强度 $\V{E}$、磁场强度 $\V{H}$ 等;
$y_i$ 是\kwd{广义坐标},如体积 $V$、电位移 $\V{D}$、磁感应强度 $\V{B}$ 等。

\subsection{非静态过程}

等容过程是一种非静态过程。由于体积不变,即 $\dd{V} = 0$,因此
\begin{equation} \label{eq:work-fixed-V}
  \dbar{W} = 0.
\end{equation}

另一种非静态过程是等压过程。由于\emph{外界}压强不变,即 $p_\text{ext}=\const$,
因此体积从 $V_1$ 变化到 $V_2$ 时外界对系统做功
\begin{equation}
  W = -p_\text{ext} \qty(V_2 - V_1) = -p_\text{ext} \incr{V}.
\end{equation}
若系统初、终态压强相等,并等于外部压强,即
\begin{equation}
  p_1 = p_2 \defeq p = p_\text{ext} .
\end{equation}
则有
\begin{equation} \label{eq:work-fixed-p}
  W = -p \qty(V_2 - V_1) = -p \incr{V}.
\end{equation}
这里的 $p$ 指系统内的压强。注意,我们并没有要求\emph{整个过程}中系统内的压强
都等于 $p$,只需要初、终态压强等于 $p$ 即可。

\subsection{热力学第一定律}

\kwd{热力学第一定律}的主要建立者有 Mayer、Joule、Helmholtz、Carnot 等。它描述
了\emph{功}、\emph{热量}、\emph{内能}三者之间的关系,是\kwd{能量守恒定律}在宏观
热现象过程中的表现形式。

\begin{theorem}[能量守恒定律]
  自然界中的一切物质都具有能量。能量有各种不同的形式,能够从一种形式转化为
  另一种形式,从一个物体传递给另一个物体。在转化和传递中,能量的总量不变。
\end{theorem}

对绝热过程而言,从状态 1 变化到状态 2 的过程中,只有外界做功,因此
\begin{equation}
  U_2 - U_1 = W_\text{a}.
\end{equation}
下标“a”表示绝热。

而对于非绝热过程,由于既有外界做功,又有热量传递,因此
\begin{equation} \label{eq:first-law-integral-form}
  U_2 - U_1 = W + Q.
\end{equation}
写成微分形式,为
\begin{boced} \label{eq:first-law-differential}
  \dd{U} = \dbar{W} + \dbar{Q}.
\end{boced}
这实际上就是热力学第一定律最常用的表述。

设两个全同系统内能分别为 $U_1$、$U_2$。则总内能 $U_\text{total}$ 除了
$U_1+U_2$,还需包括由于界面效应所导致的 $U_{12}$。但在热力学中,界面效应基本
可以忽略,因此近似有
\begin{equation}
  U_\text{total} = U_1 + U_2.
\end{equation}
这说明内能是一个广延量。

\section{热容与焓;理想气体的性质} \label{sec:heat-capacity-enthalpy-ideal-gas}

\subsection{热容与焓} \label{subsec:heat-capacity-and-enthalpy}

对于过程 $y$,定义\kwd{热容}
\begin{equation}
  C_y \defeq \frac{\dbar{Q_y}}{\dd{T}},
\end{equation}
式中的 $\dbar{Q_y}$ 是温度升高 $\dd{T}$ 时系统所吸收的热量。

对于\emph{等容过程},有\kwd{定容热容}
\begin{equation}
  C_V \defeq \frac{\dbar{Q_V}}{\dd{T}}.
\end{equation}
根据式~\eqref{eq:work-fixed-V},可知
\begin{equation}
  \dbar{Q_V} = \dd{U} - \dbar{W} = \dd{U} - 0 = \dd{U},
\end{equation}
因此
\begin{equation} \label{eq:heat-capacity-fixed-V}
  C_V = \qty(\pdv{U}{T})_V.
\end{equation}

对于\emph{等压过程},有\kwd{定压热容}
\begin{equation}
  C_p \defeq \frac{\dbar{Q_p}}{\dd{T}}.
\end{equation}
根据式~\eqref{eq:work-fixed-p},可知
\begin{equation} \label{eq:dQ-fixed-p}
  \dbar{Q_p} = \dd{U} - \dbar{W} = \dd{U} - p \dd{V},
\end{equation}
因此
\begin{equation} \label{eq:heat-capacity-fixed-p}
  C_V = \qty(\pdv{U}{T})_p + p \, \qty(\pdv{V}{T})_p.
\end{equation}

% 可以发现,
% \begin{equation}
%   C_p = C_V + p \, \qty(\pdv{V}{T})_p.
% \end{equation}
% 这是它们的关系之一。

下面引入一个新的物理量——\kwd{焓},其定义为
\begin{equation}
  H \defeq U + p V.
\end{equation}
由于 $U$、$p$ 和 $V$ 均是状态函数,因此焓也是状态函数。利用焓,
可将式~\eqref{eq:heat-capacity-fixed-p} 改写为
\begin{equation} \label{eq:heat-capacity-fixed-p-with-H}
  C_p = \qty(\pdv{H}{T})_p,
\end{equation}
同时将式~\eqref{eq:dQ-fixed-p} 改写为
\begin{equation}
  \dbar{Q_p} = \dd{H}.
\end{equation}
这就是说,\emph{在等压过程中,物体吸收的热量等于焓的增加量}。

显然,热容与焓均是广延量。单位质量下的热容称为\kwd{比热容}或\kwd{比热},
它是强度量。

\subsection{理想气体的性质} \label{subsec:ideal-gas-property} %\label{subsec:理想气体的性质}

% \begin{figure}[h]
%   \begin{asy}
%     pair p1 = (0, 7), p2 = (0, 0), p3 = (12, 0), p4 = (p3.x, p1.y), p5 = (0, 6), p6 = (p3.x, p5.y);
%     pair m1 = (p1+p4)/2, m2 = (p2+p3)/2;
%     transform myReflect = reflect(m1, m2);
%     
%     real boxWidth = 3, boxHeight = 2;
%     pair pA1 = (1.8, 2.5), pA2 = (pA1.x+boxWidth, pA1.y), pA3 = (pA2.x, pA1.y+boxHeight), pA4 = (pA1.x, pA3.y);
%     path boxA = pA1--pA2--pA3--pA4--cycle;
%     path boxB = myReflect*boxA;
%     
%     real pipeSize = 0.2, pipeHeight = 3.5;
%     pair ppA1 = (pA4.x+(boxWidth-pipeSize)/2, pA4.y),
%          ppA2 = (ppA1.x, ppA1.y+pipeHeight),
%          ppA3 = (ppA1.x+pipeSize, ppA1.y),
%          ppA4 = (ppA3.x, ppA2.y-pipeSize);
%     pair ppB1 = myReflect*ppA1, ppB2 = myReflect*ppA2, ppB3 = myReflect*ppA3, ppB4 = myReflect*ppA4;
%     path pipe = ppA1--ppA2--ppB2--ppB1--ppB3--ppB4--ppA4--ppA3--cycle;
%     
%     real valveHeight = 0.6;
%     pair pv1 = (m1.x, ppA2.y-(valveHeight+pipeSize)/2), pv2 = (pv1.x, pv1.y+valveHeight);
%     
%     real thermometerHeight = 5, thermometerR = 0.2;
%     pair pt1 = (1, 3), pt2 = (pt1.x, pt1.y+thermometerHeight);
%     
%     fill(p5--p2--p3--p6--cycle, color1+opacity(0.2));
%     draw(p1--p2--p3--p4, linewidth(1)+color1);
%     draw(boxA, linewidth(1)+color2);
%     draw(boxB, linewidth(1)+color2);
%     draw(pipe, linewidth(1)+color2);
%     draw(pv1--pv2, linewidth(2)+color3);
%     draw(pt1--pt2, linewidth(2)+color4);
%     fill(circle(pt1, thermometerR), color4);
%     
%     label("A", (ppA1+ppA3)/2, (0, -3));
%     label("B", (ppB1+ppB3)/2, (0, -3));
%     
%     pair pwLabel1 = (p3.x-1, p3.y+1.4), pwLabel2 = (p3.x+1, p3.y+2.5);
%     draw(pwLabel1--pwLabel2);
%     label("水槽", pwLabel2, E);
%     
%     pair ptLabel1 = (pt1.x-0.25, pt2.y-1), ptLabel2 = (pt1.x-1.5, pt2.y+0.4);
%     draw(ptLabel1--ptLabel2);
%     label("温度计", ptLabel2, N);
%     
%     pair pvLabel1 = (pv1.x+0.25, pv2.y-0.1), pvLabel2 = (pv1.x+1.5, pv2.y+1);
%     draw(pvLabel1--pvLabel2);
%     label("阀门", pvLabel2, N);
%   \end{asy}
%   \caption{Joule实验的装置}
%   \label{fig-Joule-experiment}
% \end{figure}

先介绍 Joule 实验(1845 年)。设有一容器,分为 A、B 两个相同的部分。将它们置于
水槽内,并通过带阀门的细管相连,水温可以通过温度计测量,
如图~\ref{fig-Joule-experiment} 所示。

首先,在 A 内充满气体,而使 B 保持真空。然后打开阀门,气体将\emph{自由膨胀},
并充满整个容器。Joule 的实验结果是前后水温不变。

由于是等容过程,因此 $W=0$;水温不变,说明 $Q=0$。根据热力学第一定律,有
$\incr{U}=0$。假设内能是温度和体积的函数,即
\begin{equation}
  U = U(T, \, V).
\end{equation}
根据\emph{偏导数三乘积法则},可知
\begin{equation}
  \qty(\pdv{U}{V})_T \qty(\pdv{V}{T})_U \qty(\pdv{T}{U})_V = -1.
\end{equation}
于是
\begin{equation}
  \qty(\pdv{U}{V})_T
  = -\qty(\pdv{T}{U})_V^{-1} \qty(\pdv{V}{T})_U^{-1}
  = -\qty(\pdv{U}{T})_V \qty(\pdv{T}{V})_U.
\end{equation}
注意到系统满足 $\incr{U}=0$ 以及 $\incr{T}=0$,因此有
\begin{equation}
  \qty(\pdv{T}{V})_U = 0 \implies \qty(\pdv{U}{V})_T = 0.
\end{equation}
这就说明 $U = U(T)$,即内能只和温度有关。

不过,限于当时的技术水平,Joule 实验较为粗糙。更精确的实验表明,实际气体的内能
不仅与 $T$ 有关,还与 $V$ 有关。但对于理想气体,以上结论仍然成立,因此理想气体
具有如下两条性质:
\begin{braced}
  & pV = nRT,  \label{eq:ideal-gas-property-equation-of-state}\\
  & U  = U(T). \label{eq:ideal-gas-property-internal-energy}
\end{braced}
就目前而言,这两条性质彼此是独立的。但利用热力学第二定律或统计物理,可以由
\eqref{eq:ideal-gas-property-equation-of-state}~式推导
\eqref{eq:ideal-gas-property-internal-energy}~式。
%[见\secref{sec:Maxwell关系}\subsecref{subsec:简单应用_OF_MAXWELL关系}中的\egref{EG_pd_U/pd_V_WITH_FIXED_T}]

对于理想气体,根据焓的定义,可得
\begin{equation}
  H = U + pV = U(T) + nRT = H(T),
\end{equation}
即焓也仅是温度的函数。从而根据式~\eqref{eq:heat-capacity-fixed-V} 和
\eqref{eq:heat-capacity-fixed-p-with-H},又有
\begin{equation}
  C_p - C_V = \qty(\pdv{H}{T})_p - \qty(\pdv{U}{T})_V
  = \dv{H}{T} - \dv{U}{T} = \dv{T} \qty\Big(pV) = \dv{T} \qty\Big(nRT) = nR.
  \label{eq:C_p-C_V-for-ideal-gas}
\end{equation}

定义\kwd{热容比} \footnote{也称为\kwd{绝热指数},原因见第%
  \ref{subsec:adiabatic-process}小节。} $\gamma$ 为 $C_p$ 与 $C_V$ 之比:
\begin{equation} \label{eq:heat-capacity-ratio}
  \gamma \defeq \frac{C_p}{C_V}.
\end{equation}
对于理想气体,显然也有
\begin{equation}
  \gamma = \gamma(T).
\end{equation}

由式~\eqref{eq:C_p-C_V-for-ideal-gas} 和 \eqref{eq:heat-capacity-ratio},
可以解得
\begin{braced}
  C_V &= \frac{1}{\gamma-1} nR,      \label{eq:C_V-in-heat-capacity-ratio} \\
  C_p &= \frac{\gamma}{\gamma-1} nR. \label{eq:C_p-in-heat-capacity-ratio}
\end{braced}
因为 $\gamma$ 可以通过实验测量,由此算出热容后,就可以确定理想气体的内能与焓:
\begin{braced}
  U(T) &= \int C_V(T) \dd{T} + U_0, \\
  H(T) &= \int C_p(T) \dd{T} + H_0,
\end{braced}
其中的 $U_0$ 和 $H_0$ 是积分常数。将上式写成微分形式,为
\begin{braced}
  \dd{U} &= C_V(T) \dd{T}, \label{eq:dU-for-ideal-gas} \\ %\label{EQ_dU=Cv*dT_IDEAL_GAS}
  \dd{H} &= C_p(T) \dd{T}. \label{eq:dH-for-ideal-gas}    %\label{EQ_dH=Cp*dT_IDEAL_GAS}
\end{braced}

\subsection{绝热过程的过程方程} \label{subsec:adiabatic-process} %\label{subsec:绝热过程的过程方程}

\emph{本节叙述均只针对理想气体。}

根据理想气体状态方程
\begin{equation}
  pV = nRT,
\end{equation}
有
\begin{equation}
  \dd{T} = \frac{p\dd{V}+V\dd{p}}{nR}.
\end{equation}
对于绝热过程,根据热力学第一定律,有
\begin{align}
  0 = \dbar{Q} &= \dd{U} + p\dd{V} = C_V\dd{T} + p\dd{V} \notag \\
  &= C_V \, \frac{p\dd{V}+V\dd{p}}{nR} + p\dd{V} \notag \\
  &= \qty(\frac{C_V}{nR} + 1) p\dd{V} + \frac{C_V}{nR} V\dd{p} \notag
  \intertext{利用式~\eqref{eq:C_V-in-heat-capacity-ratio},可得}
  &= \frac{\gamma}{\gamma-1} p \dd{V} + \frac{1}{\gamma - 1} V \dd{p},
\end{align}
即
\begin{equation}
  \frac{\dd{p}}{p} + \gamma \frac{\dd{V}}{V} = 0.
\end{equation}
假定 $\gamma$ 为常数,积分可得
\begin{equation} \label{eq:equation-of-adiabatic-process-in-p-V}
  \ln p + \gamma \ln V = \ln(pV^\gamma) = \const \implies pV^\gamma = \const
\end{equation}
改用其他变量,可把该式写成
\begin{equation} \label{eq:equation-of-adiabatic-process-in-p-T}
  p^{(1 - \gamma) / \gamma} T = \const
\end{equation}
或
\begin{equation} \label{eq:equation-of-adiabatic-process-in-T-V}
  T V^{\gamma - 1} = \const
\end{equation}
的形式。

%    \begin{myExample}[海拔与气温的关系]
%      下面推导\emph{气温垂直递减率}。首先考虑\emph{干燥空气}的温度\emph{绝热}递减率。假设空气是理想气体。
%      
%      因为是绝热过程,因此
%      \begin{equation}
%        p^{(1 - \g) / \g} T = \const
%      \end{equation}
%      两边求微分,得
%      \begin{equation}
%        \frac{1 - \g}{\g} p^{\frac{1 - \g}{\g} - 1} T \dd p + p^{\frac{1 - \g}{\g} - 1} \dd T = 0 \comma
%      \end{equation}
%      即
%      \begin{equation} \label{EQ_dT/dp_IN_EXAMPLE_OF_LAPSE_RATE}
%        \frac{\dd T}{\dd p} = \frac{1 - \g}{\g} \frac{T}{p} \fullstop
%      \end{equation}
%      
%      假设大气处于平衡状态,则有
%      \begin{equation}
%        \dd p = -\r g \dd z \comma
%      \end{equation}
%      其中的 $\r$ 是空气密度,
%      \begin{equation}
%        \r = \frac{m}{V} = \frac{n M}{n R T / p} = \frac{p M}{R T} \comma
%      \end{equation}
%      这里的 $M$ 是空气的平均摩尔质量, $g$ 是重力加速度,$m$、$V$ 分别是一定量空气的质量和体积。代入 \eqref{EQ_dT/dp_IN_EXAMPLE_OF_LAPSE_RATE} 式,可得
%      \begin{equation}
%        \frac{\dd T}{\dd z} = -\frac{1 - \g}{\g} \frac{\r g T}{p} = -\frac{1 - \g}{\g} \frac{M g}{R} \fullstop
%      \end{equation}
%      代入空气的热容比 $\g = 1.4$、平均摩尔质量 $M = \SI{28.8e-3}{\kg\per\mol}$ 等数值,可得
%      \begin{equation}
%        \frac{\dd T}{\dd z} = \SI{-9.7}{\kelvin\per\km} \fullstop
%      \end{equation}
%      
%      \blankline
%      
%      若为水汽饱和的湿空气,有下面的近似公式:%TODO:20160315 参考文献
%      \begin{equation} \label{EQ_SATURATED_ADIABATIC_LAPSE_RATE}
%        \frac{\dd T}{\dd z}
%        = -g \, \frac{1 + \dfrac{L_\text{vap} r}{R_\text{s,\,dry} T}}{c_{p,\,\text{dry}} + \dfrac{L_\text{vap}^2 r}{R_\text{s,\,water} T^2}}
%        = -g \, \frac{1 + \dfrac{L_\text{vap} r}{R_\text{s,\,dry} T}}{c_{p,\,\text{dry}} + \dfrac{L_\text{vap}^2 r \e}{R_\text{s,\,dry} T^2}} \comma
%      \end{equation}
%      式中的各符号见表~\ref{TAB_SYMBOLS_IN_SATURATED_ADIABATIC_LAPSE_RATE}。%FIXME:20160401 qed位置
%      
%      \begin{myTable}{Mcc}{式~\eqref{EQ_SATURATED_ADIABATIC_LAPSE_RATE} 中所用到的符号}{TAB_SYMBOLS_IN_SATURATED_ADIABATIC_LAPSE_RATE}
%        \toprule
%        \text{\kwd{符号}} & \kwd{说明} & \kwd{数值} \\%HACK:20160330 表格首行加粗
%        \midrule
%        g & 重力加速度 & \SI{9.8076}{\metre\per\second\squared} \\
%        L_\text{vap} & 水的汽化热 & \SI{2257}{\kilo\joule\per\kg} \\
%        c_{p,\,\text{dry}} & 干燥空气的定压比热容 & \SI{1003.5}{\joule\per\kg\per\kelvin} \\
%        R_\text{s,\,dry} & 干燥空气的气体常数 & \SI{287}{\joule\per\kg\per\kelvin} \\
%        R_\text{s,\,water} & 水蒸气的气体常数 & \SI{461.5}{\joule\per\kg\per\kelvin} \\
%        \e = R_\text{s,\,dry} / R_\text{s,\,water} & 干燥空气与水蒸气的气体常数之比 & 0.622 \\
%        e & 饱和空气的水蒸气分压 & —— \\
%        p & 饱和空气的气压 & —— \\
%        r = \e e / (p - e) & 水蒸气的质量与干燥空气质量的混合比例 & —— \\
%        T & 饱和空气的温度 & —— \\
%        \bottomrule
%      \end{myTable}
%    \end{myExample}
%    
%\section{理想气体与Carnot循环;热力学第二定律} \label{sec:理想气体与Carnot循环_热力学第二定律}
%  \subsection{Carnot循环} \label{subsec:Carnot循环}
%    \kwd{Carnot循环}分为四个过程,如图~\ref{FIG_CARNOT_CYCLE} 所示。
%    \begin{myEnum2}
%      \item 等温膨胀:$(T_\text{H}, \, V_1) \rightarrow (T_\text{H}, \, V_2)$,
%      \item 绝热膨胀:$(T_\text{H}, \, V_2) \rightarrow (T_\text{C}, \, V_3)$,
%      \item 等温压缩:$(T_\text{C}, \, V_3) \rightarrow (T_\text{C}, \, V_4)$,
%      \item 绝热压缩:$(T_\text{C}, \, V_4) \rightarrow (T_\text{H}, \, V_1)$,
%    \end{myEnum2}
%    这里的 $T_\text{H}$ 和 $T_\text{C}$ 分别指高温和低温,并且还有 $V_1 < V_2$,$V_4 < V_3$。
%    
%    \begin{figure}[h]
%      \begin{asy}
%        import graph;
%        pair O = (0, 0), x_axes = (10, 0), y_axes = (0, 10);
%        draw(Label("$V$", EndPoint), O--x_axes, Arrow);
%        draw(Label("$p$", EndPoint), O--y_axes, Arrow);
%        
%        real gamma = 2.5;
%        real x1 = 2, x4 = 9;
%        real c1 = 10, c2 = 18;
%        real c3 = c2*x1^(gamma-1), c4 = c1*x4^(gamma-1);
%        real x2 = x1*(c1/c2)^(1/(1-gamma)), x3 = x4*(c2/c1)^(1/(1-gamma));
%        
%        path path1 = graph(new real(real x) {return c2/x;}, x1, x3);
%        path path2 = graph(new real(real x) {return c4/x^gamma;}, x3, x4);
%        path path3 = reverse(graph(new real(real x) {return c1/x;}, x2, x4));
%        path path4 = reverse(graph(new real(real x) {return c3/x^gamma;}, x1, x2));
%        //path path_TH = graph(new real(real x) {return c2/x;}, x3, 8);
%        //path path_TC = graph(new real(real x) {return c1/x;}, 1.7, x2);
%        
%        pair p1 = (x1, c2/x1), p2 = (x3, c2/x3), p3 = (x4, c1/x4), p4 = (x2, c1/x2);
%        
%        pen pen1 = linewidth(1)+color1;
%        
%        fill(path1 & path2 & path3 & path4 & cycle, color1+opacity(0.2));
%        
%        //draw(path_TH, dashed + color1);
%        //draw(path_TC, dashed + color1);
%        draw(path1, pen1, Arrow(position = Relative(0.7), arrowhead = HookHead, size = 4));
%        draw(path2, pen1, Arrow(position = Relative(0.5), arrowhead = HookHead, size = 4));
%        draw(path3, pen1, Arrow(position = Relative(0.7), arrowhead = HookHead, size = 4));
%        draw(path4, pen1, Arrow(position = Relative(0.45), arrowhead = HookHead, size = 4));
%        
%        draw(Label("$V_1$", EndPoint, black), p1--(p1.x, 0), dashed + color1);
%        draw(Label("$V_2$", EndPoint, black), p2--(p2.x, 0), dashed + color1);
%        draw(Label("$V_3$", EndPoint, black), p3--(p3.x, 0), dashed + color1);
%        draw(Label("$V_4$", EndPoint, black), p4--(p4.x, 0), dashed + color1);
%        
%        label("状态1", p1, N);
%        label("状态2", p2, NE);
%        label("状态3", p3, E);
%        label("状态4", p4, SW, Fill(white));
%        
%        label("I", path1, align = Relative(W));
%        label("II", path2, align = Relative(W));
%        label("III", path3, align = Relative(W));
%        label("IV", path4, align = Relative(W), Fill(white));
%      \end{asy}
%      \caption{Carnot循环示意图}
%      \label{FIG_CARNOT_CYCLE}
%    \end{figure}
%    
%    利用热力学第一定律,有
%    \begin{equation}
%      \oint \dd U = \oint \dbar Q + \oint \dbar W = 0 \comma
%    \end{equation}
%    其中的“$\oint$”代表沿循环过程的积分。
%    
%    整个过程中对外做的净功(它等于图~\ref{FIG_CARNOT_CYCLE} 中曲线包围起来的面积)
%    \begin{equation}
%      W' = -\oint \dbar W = \oint \dbar Q = Q_\text{H} + Q_\text{C} \comma
%    \end{equation}
%    其中的 $Q_\text{H}$ 和 $Q_\text{C}$ 分别为高温和低温时吸收的热量(可以有正负)。
%    
%    若工作物质为理想气体,则有
%    \begin{align}
%      Q_\text{H} &= \incr U_\text{I} - W_\text{I} \myTag{热力学第一定律}\\
%      &= 0 - W_\text{I} \notag \\
%      &= \int_{V_1}^{V_2} p \dd V \notag \\
%      &= n R T_\text{H} \int_{V_1}^{V_2} \frac{\dd V}{V} \myTag{根据 $p V = n R T$} \\
%      &= n R T_\text{H} \ln \frac{V_2}{V_1}
%      > 0 \label{EQ_Q_H_IN_CARNOT_CYCLE} \fullstop
%    \end{align}
%    同理,还有
%    \begin{equation} \label{EQ_Q_C_IN_CARNOT_CYCLE}
%      Q_\text{C} = -n R T_\text{C} \ln \frac{V_3}{V_4} < 0 \fullstop
%    \end{equation}
%    
%    根据绝热过程的过程方程~\eqref{eq:equation-of-adiabatic-process-in-T-V},有
%    \begin{mySubEq}
%      \begin{empheq}[left=\empheqlbrace]{align}
%        & T_\text{H} V_2^{\g - 1} = T_\text{C} V_3^{\g - 1} \comma \\
%        & T_\text{H} V_1^{\g - 1} = T_\text{C} V_4^{\g - 1} \comma
%    \end{empheq}
%    \end{mySubEq}
%    两边分别相除,得
%    \begin{equation} \label{EQ_V2/V1_IN_CARNOT_CYCLE}
%      \frac{V_2}{V_1} = \frac{V_3}{V_4} \fullstop
%    \end{equation}
%    
%    定义\kwd{热机效率}
%    \begin{equation}
%      \h \defeq \frac{W'}{Q_\text{H}} 
%      = \frac{Q_\text{H} - \abs{Q_\text{C}}}{Q_\text{H}} 
%      = 1 - \frac{\abs{Q_\text{C}}}{Q_\text{H}} \fullstop
%    \end{equation}
%    对于理想气体,代入式~\eqref{EQ_Q_H_IN_CARNOT_CYCLE} 和 \eqref{EQ_Q_C_IN_CARNOT_CYCLE},并利用式~\eqref{EQ_V2/V1_IN_CARNOT_CYCLE},可得
%    \begin{align}
%      \h &= 1 - \frac{T_\text{C}}{T_\text{H}} \frac{\ln (V_3 / V_4)}{\ln (V_2 / V_1)} \notag \\
%      &= 1 - \frac{T_\text{C}}{T_\text{H}} \label{EQ_CARNOT_CYCLE_EFFICIENCY_WITH_IDEAL_GAS} \fullstop
%    \end{align}%TODO:20160320 非理想气体的情况,见作业
%    
%    若Carnot循环反向进行,就成为\kwd{Carnot制冷机}。其\emph{制冷效率}定义为
%    \begin{equation}
%      \ve = \frac{Q_\text{C}}{W}
%      = \frac{Q_\text{C}}{Q_\text{H} - Q_\text{C}}
%      = \frac{T_\text{C}}{T_\text{H} - T_\text{C}} \comma
%    \end{equation}
%    它通常是大于 $1$ 的。
%    
%  \subsection{热力学第二定律}
%    \kwd{热力学第二定律}解决了有关过程\emph{方向性}的问题,它的主要建立者有Carnot、Clausius、Kelvin等。
%    
%    \begin{myThm}{热力学第二定律(Kelvin表述)}
%      不可能从单一热源吸热使之完全变为有用的功而不产生其他影响,即第二类永动机不可能实现。
%    \end{myThm}
%    \begin{myThm}{热力学第二定律(Clausius表述)}
%      不可能把热量从低温物体传到高温物体而不产生其他影响。
%    \end{myThm}
%    
%    \begin{proof}%TODO:20160318 图片
%      Kelvin表述 $\implies$ Clausius表述:
%      
%      采用反证法,即证明 $\neg \, (\text{Clausius表述}) \implies \neg \, (\text{Kelvin表述})$。
%      
%      Carnot热机A工作于高温热源 $T_\text{H}$ 和低温热源 $T_\text{C}$ 之间。它从 $T_\text{H}$ 处吸收热量 $Q_\text{H}$,向 $T_\text{C}$ 放出热量 $Q_\text{C}$,%并做功 $W = Q_\text{H} - Q_\text{C}$。假设Clausius表述不成立,就可以在不产生其他影响的前提下,使低温热源获得的热量 $Q_\text{C}$ 重新回到高温热源。净结果%便是从单一热源 $T_\text{H}$ 吸收了热量 $Q_\text{H} - Q_\text{C}$,并将其完全转化为功,这就违背了Kelvin表述。因此原假设不成立,即有 $\neg \, (\text%{Clausius表述}) \implies \neg \, (\text{Kelvin表述})$。
%      
%      \blankline
%      
%      Clausius表述 $\implies$ Kelvin表述:
%      
%      同样采用反证法,即证明 $\neg \, (\text{Kelvin表述}) \implies \neg \, (\text{Clausius表述})$。
%      
%      假设Kelvin表述不成立,就可以在不产生其他影响的前提下,从单一热源 $T_\text{H}$ 吸热 $Q_\text{H}$ 并将其完全转化为有用功 $W = Q_\text{H}$。它可以推动%Carnot制冷机从低温热源 $T_\text{CH}$ 吸收 $Q_\text{C}$ 的热量并传给高温热源 $Q_\text{H} + Q_\text{C}$ 的热量。净结果是热量 $Q_\text{C}$ 从低温热源传给了%高温热源,却没有产生其他影响,这就违背了Clausius表述。因此原假设不成立,即有 $\neg \, (\text{Kelvin表述}) \implies \neg \, (\text{Clausius表述})$。
%    \end{proof}
%    
%    热力学第二定律的核心内容可以概括为:自然界一切热现象过程都是不可逆的。
%    
%\section{热力学第二定律的数学解释;熵}
%  \subsection{Carnot定理}
%    \begin{myThm}{Carnot定理}
%      工作于两个确定温度之间的所有热机中,可逆热机效率最高。
%    \end{myThm}
%    
%    设两个热机A、B工作于高温热源 $\th_\text{H}$ 和低温热源 $\th_\text{C}$ 之间\footnote{
%      这里用 $\th$ 表示温度,而不是像前文一样使用 $T$,原因见下一小节。
%    },它们分别从 $\th_\text{H}$ 吸收 $Q_\text{H,\,A}$ 与 $Q_\text{H,\,B}$ 的热量,向 $\th_\text{C}$ 放出 $Q_\text{C,\,A}$ 与 $Q_\text{C,\,B}$ 的热量,并对外%做功 $W_\text{A}$ 与 $W_\text{B} \,$\footnote{
%      前文用撇号表示系统(热机)对外界做功,这里方便起见直接用 $W$。但需注意,$W_\text{A}$ 与 $W_\text{B}$ 均大于 $0$。
%    }。根据定义,其效率分别为
%    \begin{equation}
%      \h_\text{A} = \frac{W_\text{A}}{Q_\text{H,\,A}} \comma \, \h_\text{B} = \frac{W_\text{B}}{Q_\text{H,\,}} \fullstop
%    \end{equation}
%    设A是一个可逆热机,因此我们把 $\h_\text{A}$ 写成 $\h_\text{rev,\,A}$。根据Carnot定理,有
%    \begin{equation}
%      \h_\text{rev,\,A} \geqslant \h_\text{B} \fullstop
%    \end{equation}
%    
%    \begin{proof}
%      下面利用反证法证明Carnot定理,即假设 $\h_\text{rev,\,A} < \h_\text{B}$。因此
%      \begin{equation}
%        \frac{W_\text{A}}{Q_\text{H,\,A}} < \frac{W_\text{B}}{Q_\text{H,\,B}} \fullstop
%      \end{equation}
%      令A、B从高温热源 $\th_\text{H}$ 处吸收相同的热量,即 $Q_\text{H,\,A} = Q_\text{H,\,B}$,那么就有 $W_\text{A} < W_\text{B}$。因为A是可逆热机,所以不妨让B%热机输出功的一部分 $W_\text{A}$ 推动A热机逆向运行(此时A就是一个制冷机)。此时,B热机还可以输出功 $W_\text{B} - W_\text{A}$。
%      
%      根据热力学第一定律,有
%      \begin{mySubEq}
%        \begin{empheq}[left=\empheqlbrace]{align}
%          & W_\text{A} = Q_\text{H,\,A} - Q_\text{C,\,A} \comma \\
%          & W_\text{B} = Q_\text{H,\,B} - Q_\text{C,\,B} \comma
%        \end{empheq}
%      \end{mySubEq}
%      因此
%      \begin{equation}
%        W_\text{B} - W_\text{A} = Q_\text{C,\,A} - Q_\text{C,\,B} \fullstop
%      \end{equation}
%      
%      若A、B联合运行,其净结果便是从低温热源 $\th_\text{C}$ 处吸收 $Q_\text{C,\,A} - Q_\text{C,\,B}$ 的热量,并对外做了 $W_\text{B} - W_\text{A}$ 的功,即在不%产生其他影响的情况下完全把热转化为了功。这显然违背了热力学第二定律的Kelvin表述。因此原假设不成立,于是Carnot定理得证。
%    \end{proof}
%    
%    由Carnot定理,可以得到如下推论:
%    \begin{myThm*}
%      所有工作于两个确定温度之间的可逆热机效率均相等。
%    \end{myThm*}
%    
%  \subsection{热力学温标}
%    \kwd{温标(scale of temperature)},是以量化数值,配以温度单位来表示温度的方法。它包含三个要素:
%    \begin{myEnum2}
%      \item \emph{测温质}与\emph{测温参量};
%      \item 测温参量与温度的\emph{函数关系};
%      \item \emph{温度标准点}的选定。
%    \end{myEnum2}
%    
%    常用的经验温标有摄氏温标、华氏温标等。利用理想气体状态方程,可以定义\kwd{理想气体温标}:
%    \begin{equation}
%      T \defeq \frac{1}{n R} \lim\limits_{p \approach 0} p V \comma
%    \end{equation}
%    同时需要规定水的三相点温度 $T_\text{tr} \defeq \SI{273.16}{\kelvin}$。
%    
%    在 \secref{sec:理想气体与Carnot循环_热力学第二定律} \subsecref{subsec:Carnot循环} 中,我们使用 $T$ 表示温度。实际上,那里的“温度”是用\emph{理想气体温标}%表示的值。
%    
%    \blankline
%    
%    根据Carnot定理,可逆热机的效率只与两个热源的温度有关,而与工作物质的性质、吸放热多少、做功多少均无关。因此,可逆热机的效率是两个温度 $\th_\text{H}$、%$\th_\text{C}$ 的\emph{普适函数}。根据定义,热机的效率
%    \begin{equation}
%      \h = \frac{W}{Q_\text{H}} = 1 - \frac{Q_\text{C}}{Q_\text{H}} \fullstop
%    \end{equation}
%    因此有
%    \begin{equation}
%      \frac{Q_\text{C}}{Q_\text{H}} = F(\th_\text{H}, \, \th_\text{C}) \comma
%    \end{equation}
%    其中的 $F(\th_\text{H}, \, \th_\text{C})$ 是 $\th_\text{H}$ 与 $\th_\text{C}$ 的普适函数。
%    
%    下面证明
%    \begin{equation}
%      F(\th_\text{H}, \, \th_\text{C}) = \frac{f(\th_\text{C})}{f(\th_\text{H})} \comma
%    \end{equation}
%    其中的 $f$ 是另一个普适函数。%TODO:20160322 热力学温标证明
%    
%    由式~\eqref{EQ_CARNOT_CYCLE_EFFICIENCY_WITH_IDEAL_GAS},理想气体Carnot热机效率为
%    \begin{equation}
%      1 - \frac{T^*_\text{C}}{T^*_\text{H}} \comma
%    \end{equation}
%    这里用带 $^*$ 的 $T$ 表示理想气体温标下的温度。这与式 是相同的,即温度尺度相同。又因为理想气体温标也规定在水的三相点处 $T^*_\text{tr} = \SI{273.16}%{\kelvin}$,因此,理想气体温标与热力学温标是相同的。
%    
%  \subsection{Clausius不等式}
%    根据Carnot定理,工作于两个确定温度之间的所有热机,其效率均满足
%    \begin{equation}
%      \h = 1 - \frac{Q_2}{Q_1} \leqslant 1 - \frac{T_2}{T_1} \comma \footnote{
%        以后均直接用 $T$ 表示温度。
%      }
%    \end{equation}
%    对于可逆热机,取等号;对于不可逆热机,则取小于号。
%    
%    上式稍作变形,可得
%    \begin{align}
%      &\mathrel{\phantom{\implies}} \frac{Q_2}{Q_1} \geqslant \frac{T_2}{T_1} \\
%      &\implies \frac{Q_1}{T_1} - \frac{Q_2}{T_2} \leqslant 0 \fullstop
%    \end{align}
%    约定 $Q$ 始终表示吸收的热量,则放热应写作 $-Q$。于是
%    \begin{equation}
%      \frac{Q_1}{T_1} + \frac{Q_2}{T_2} \leqslant 0 \fullstop
%    \end{equation}
%    假设系统先后与温度分别为 $T_1, \, T_2 \, \dots, \, T_n$ 的 $n$ 个热源接触,又分别吸热 $Q_1, \, Q_2 \, \dots, \, Q_n$,则可以证明\kwd{Clausius不等式}:
%    \begin{equation}
%      \iTonSum \frac{Q_i}{T_i} \leqslant 0 \fullstop
%    \end{equation}
%    
%    \begin{proof}
%      设有程%TODO:20160322 证明过程没写
%    \end{proof}
%    
%    在 $n \approach \infty$ 的极限下,Clausius不等式过渡到积分形式:
%    \begin{equation} \label{EQ_CLAUSISU_INEQUALITY_IN_INTEGRAL}
%      \lim\limits_{n \approach \infty} \iTonSum \frac{Q_i}{T_i} \leqslant 0 \approach \oint \frac{\dbar Q}{T} \leqslant 0 \fullstop
%    \end{equation}
%    
%  \subsection{熵的定义}
%    对于可逆循环,根据式~\eqref{EQ_CLAUSISU_INEQUALITY_IN_INTEGRAL},有
%    \begin{equation}
%      \oint \frac{\dbar Q_\text{rev}}{T} = 0 \fullstop
%    \end{equation}
%    如图%TODO:20160323 图片
%    可以表示成两段路径之和:
%    \begin{equation}
%      \underset{C_1\phantom{M}}{\int_{(P_0)}^{(P)}} \, \frac{\dbar Q_\text{rev}}{T}
%      + \underset{C_2\phantom{M}}{\int_{(P)}^{(P_0)}} \, \frac{\dbar Q_\text{rev}}{T} = 0 \comma
%    \end{equation}
%    即
%    \begin{equation}
%      \underset{C_1\phantom{M}}{\int_{(P_0)}^{(P)}} \, \frac{\dbar Q_\text{rev}}{T}
%      = \underset{C_2\phantom{M}}{\int_{(P_0)}^{(P)}} \, \frac{\dbar Q_\text{rev}}{T} = \const
%    \end{equation}
%    可以看出,$\dbar Q_\text{rev} / T$ 是一个与路径无关的量。由此,定义一个新的状态函数——\kwd{熵(entropy)}:
%    \begin{equation}
%      S - S_0 = \int_{(P_0)}^{(P)} \, \frac{\dbar Q_\text{rev}}{T} \fullstop
%    \end{equation}
%    
%  \subsection{不可逆过程的数学表述}
%    \begin{myEnum1}
%      \myItem{初终态均是平衡态}
%        根据Clausius不等式,有
%        \begin{align}
%          &\mathrel{\phantom{\implies}} \underset{\text{irrev} + \text{rev}}{\oint} \frac{\dbar Q}{T} < 0 \notag \\
%          &\implies \int_{(P_0)}^{(P)} \, \frac{\dbar Q_\text{irrev}}{T} + \int_{(P)}^{(P_0)} \, \frac{\dbar Q_\text{rev}}{T} < 0 \notag \\
%          &\implies S - S_0 > \int_{(P_0)}^{(P)} \, \frac{\dbar Q_\text{irrev}}{T} \fullstop
%        \end{align}
%        
%      \myItem{初终态均是非平衡态}
%        采用\emph{局域平衡近似},仍旧可以推得
%        \begin{equation}
%          S - S_0 > \int_{(P_0)}^{(P)} \, \frac{\dbar Q_\text{irrev}}{T} \fullstop
%        \end{equation} %TODO:20160323 局域平衡近似的证明
%    \end{myEnum1}
%    
%    \blankline
%    
%    把对可逆过程与不可逆过程的表述合起来,就有
%    \begin{equation} \label{EQ_2ND_LAW_IN_INTEGRAL_FORM}
%      \incr S = S - S_0 \geqslant \int_{(P_0)}^{(P)} \, \frac{\dbar Q}{T} \semicolon
%    \end{equation}
%    写成微分形式,为
%    \begin{boced} \label{EQ_2ND_LAW_IN_DIFFERENTIAL_FORM}
%      \dd S \geqslant \frac{\dbar Q}{T}
%    \end{boced}
%    以上两式中,“$=$”适用于可逆过程,“$>$”适用于不可逆过程。这两式实际上便是热力学第二定律的数学表述。
%    
%  \subsection{熵的性质}
%    这里小结一下熵的性质。
%    
%    \begin{myEnum2}
%      \item 熵是\emph{状态函数}。
%      \item 熵是\emph{广延量}。
%      \item 对微小的\emph{可逆}过程,$\dd S = \dbar Q / T$。因此有
%      \begin{equation} \label{EQ_dQ=TdS}
%        \dbar Q = T \dd S \fullstop
%      \end{equation}
%      对于\emph{绝热}过程,有 $\dbar Q = 0$,因此
%      \begin{equation} \label{EQ_dS=0_FOR_ADIABATIC_REVERSIBLE_PROCESS}
%        \dd S = 0 \fullstop
%      \end{equation}
%    \end{myEnum2}%TODO:20160323 卡诺循环的TS表述
%    
%  \subsection{热力学基本方程}
%    热力学第一定律式~\eqref{EQ_1ST_LAW_IN_DIFFERENTIAL_FORM}:
%    \begin{equation}
%      \dd U = \dbar Q + \dbar W \semicolon
%    \end{equation}
%    由热力学第二定律,得可逆过程微热量的表达式 \eqref{EQ_dQ=TdS}:
%    \begin{equation}
%      \dbar Q = T \dd S \semicolon
%    \end{equation}
%    微功的一般表示式~\eqref{eq:work-general}:
%    \begin{equation}
%      \dbar W = \iTonSum Y_i \dd y_i \fullstop
%    \end{equation}
%    联立以上三式,可得
%    \begin{equation} \label{EQ_FUNDAMENTAL_EQUATION_OF_THERMODYNAMICS}
%      \dd U = T \dd S + \iTonSum Y_i \dd y_i \fullstop
%    \end{equation}
%    这就是\kwd{热力学基本微分方程}。
%    
%    对于 $p\text{-}V\text{-}T$ 系统,上式可简化为
%    \begin{equation} \label{EQ_FUNDAMENTAL_EQUATION_FOR_PVT_SYSTEM}
%      \dd U = T \dd S - p \dd V \fullstop
%    \end{equation}
%    
%    \begin{myExample}[理想气体的熵]
%      下面推导不同过程下理想气体的熵。
%      \begin{myEnum1}
%        \myItem{等容过程}
%          根据式~\eqref{EQ_dU=Cv*dT_IDEAL_GAS},有
%          \begin{equation}
%            \dd U = C_V \dd T \fullstop
%          \end{equation}
%          根据热力学基本微分方程式~\eqref{EQ_FUNDAMENTAL_EQUATION_FOR_PVT_SYSTEM},
%          \begin{align}
%            &\mathrel{\phantom{\implies}} T \dd S = \dd U + p \dd V \notag \\
%            &\implies \dd S = \frac{\dd U}{T} + \frac{p \dd V}{T} \notag \\
%            &\implies S = \frac{C_V}{T} \dd T + \frac{}{}
%          \end{align}%TODO:20160323 有问题?
%        
%        \myItem{等压过程}
%          没写%TODO:20160330 没写
%        \myItem{等温过程}
%      \end{myEnum1}
%    \end{myExample}
%  
%\section{熵增加原理;最大功} \label{sec:熵增加原理与最大功}
%  \subsection{熵增加原理}
%    根据热力学第二定律[式~\eqref{EQ_2ND_LAW_IN_INTEGRAL_FORM}],
%    \begin{equation}
%      \incr S \geqslant \int_{\text{I}}^{\text{II}} \frac{\dbar Q}{T} \fullstop
%    \end{equation}
%    对于\emph{绝热}过程(或孤立体系),有 $\dbar Q = 0$。因此
%    \begin{equation} \label{EQ_PRINCIPLE_OF_ENTROPY_INCREASE}
%      \incr S \geqslant 0 \fullstop \footnote{
%        注意与 \eqref{EQ_dS=0_FOR_ADIABATIC_REVERSIBLE_PROCESS} 式对比,它还要求\kwd{可逆}过程。
%      }
%    \end{equation}
%    这就是\kwd{熵增加原理},它说明绝热体系的熵永不减少。
%  \subsection{不可逆过程的熵变}
%    没写%TODO:20160323 没写
%  \subsection{最大功}
%    根据热力学第一定律[式~\eqref{EQ_1ST_LAW_IN_DIFFERENTIAL_FORM}],
%    \begin{equation}
%      \dd U = \dbar Q + \dbar W \fullstop
%    \end{equation}
%    令 $\dbar W' = - \dbar W$ 为系统对外界做的功,则
%    \begin{equation} \label{EQ_dU=dQ-dW'_IN_SECTION_MAX_WORK}
%      \dbar W' = \dbar Q - \dd U \fullstop
%    \end{equation}
%    根据热力学第二定律[式~\eqref{EQ_2ND_LAW_IN_DIFFERENTIAL_FORM}],
%    \begin{equation}
%      \dbar Q \leqslant T_\text{e} \dd S \fullstop
%    \end{equation}
%    代入式~\eqref{EQ_dU=dQ-dW'_IN_SECTION_MAX_WORK},可得
%    \begin{equation}
%      \dbar W' \leqslant T_\text{e} \dd S - \dd U \fullstop
%    \end{equation}
%    因此系统对外做的\kwd{最大功}为
%    \begin{equation}
%      \dbar W'_{\text{max}} = \dbar W'_{\text{rev}} = T \dd S - \dd U \comma
%    \end{equation}
%    这里的 $T = T_\text{e}$ 为系统的温度(因为是可逆过程)。
%    
%    对于不可逆过程,显然有
%    \begin{equation}
%      \dbar W'_{\text{irrev}} < \dbar W'_{\text{rev}} \fullstop
%    \end{equation}
%    
%    \begin{myExample}[水的混合]
%      两杯等量的水初始温度分别为 $T_1$、$T_2$。在等压、绝热条件下将它们混合均匀,求该过程的熵变。%TODO:20160323 没写
%      %TODO:20160330 T取平均值:假设热容为常数
%    \end{myExample}
%    
%    \begin{myExample}[制冷机所需的最小功]
%      两物体初始温度均为 $T_1$。一台制冷机工作于其间,使一物体温度升高至 $T_2$。假设这是一个等压过程,并且不考虑相变。证明:制冷机所需的最小功
%      \begin{equation}
%        W_{\text{min}} = C_p \left( \frac{T_1^2}{T_2} + T_2 - 2 T_1 \right) \fullstop%TODO:20160323 没写
%      \end{equation}
%    \end{myExample}
%  
%\section{自由能与Gibbs函数}
%%	根据熵增加原理[式~\eqref{EQ_PRINCIPLE_OF_ENTROPY_INCREASE}],对于孤立系统,有
%%	\begin{equation}
%%		\incr S \geqslant 0 \fullstop
%%	\end{equation}
%  \subsection{自由能}
%    考虑这样的\emph{等温过程}:热源维持恒定温度 $T$;系统初终态温度 $T_1$、$T_2$ 与热源温度相同,即 $T_1 = T_2 =T$。对于可逆过程,在全程中系统温度均为 $T$;%而对于不可逆过程,仅满足 $T_1 = T_2 =T$。
%    
%    由Clausius不等式%[\eqref{}]
%    \begin{equation}
%      \incr S = S_2 - S_1 \geqslant \int_{\text{I}}^{\text{II}} \frac{\dbar Q}{T} = \frac{1}{T} \int_{\text{I}}^{\text{II}} \dbar Q = \frac{Q}{T}\comma 
%    \end{equation}
%    即
%    \begin{equation}
%      Q \leqslant T (S_2 - S_1) \fullstop
%    \end{equation}
%    根据热力学第一定律[式~\eqref{EQ_1ST_LAW_IN_INTEGRAL_FORM}],
%    \begin{equation}
%      U_2 - U_1 = W + Q \comma
%    \end{equation}
%    因此
%    \begin{align}
%      -W &= (U_1 - U_2) + Q \notag \\
%      &\leqslant (U_1 - U_2) - T (S_2 - S_1) \notag \\
%      &=(U_1 - T S_1) - (U_2 - T S_2) \fullstop
%    \end{align}
%    定义\kwd{自由能} $F = U - T S$,则
%    \begin{equation}
%      -W \leqslant F_1 - F_2 \fullstop
%    \end{equation}
%    
%    如果该过程除了保持等温,还保持等容,即 $W = 0$,则有%TODO:20160325 怎么会有等温等容?
%    \begin{equation}
%      \incr F = F_2 - F_1 \leqslant 0 \comma
%    \end{equation}
%    这说明在等温等容过程中,系统向自由能减小的方向前进。
%    
%    自由能具有以下的性质:
%    \begin{myEnum2}
%      \item 态函数 %TODO:没写
%    \end{myEnum2}
%  \subsection{Gibbs函数}
%    考虑等温等压过程
    %\chapter{均匀系统的平衡特性}
    %\include{chapters/2}
    %\chapter{相变的热力学理论}
    %\include{chapters/3}
    %\chapter{多元复相 热力学第三定律}
    %%% Copyright (C) 2016--2018 by Xiangdong Zeng <pssysrq@163.com>
%%
%% -* Notes on Thermodynamics and Statistical Physics *-
%%
%% This file may be distributed and/or modified under the
%% Creative Commons Attribution Share Alike 4.0 license.

\section{多元系热力学函数及热力学方程}
\section{多元系的复相平衡}
\section{Gibbs相律}
\section{混合理想气体}
\section{热力学第三定律}

  %\part{统计物理}
    % 统计物理学基本概念
    %% Copyright (C) 2016--2018 by Xiangdong Zeng <pssysrq@163.com>
%%
%% -* Notes on Thermodynamics and Statistical Physics *-
%%
%% This file may be distributed and/or modified under the
%% Creative Commons Attribution Share Alike 4.0 license.

\chapter{统计物理学基本概念} \label{chap:statistical-physics-basis}

\section{微观态的经典及量子描述}

\subsection{单粒子的经典描述}

微观态的经典描述以经典力学为基础,通常采用广义坐标与广义动量的形式。

对于一个有 $r$ 个自由度的系统,需要用 $2r$ 个变量来描述其运动状态,即 $r$ 个广义坐标和 $r$ 个广义动
量:
\begin{equation}
  (q_i, \, p_i) \qq{where} i = 1, \, 2, \, \cdots, r.
\end{equation}
系统的 Hamilton 量为
\begin{equation}
  H = H (q_1, \, q_2, \, \cdots, q_r; \, p_1, \, p_2, \, \cdots, p_r; \, t),
\end{equation}
正则方程为
\begin{braced}
  \dot{q_i} &= \pdv{H}{p_i}, \\
  \dot{p_i} &= -\pdv{H}{q_i}.
\end{braced}

坐标和动量 $(q_1, \, q_2, \, \cdots, q_r; \, p_1, \, p_2, \, \cdots, p_r)$ 张成一个 $2r$ 维空间,称
为 \kwd{μ 空间}(或\kwd{子相空间}),每一组坐标和动量描述的点称为\kwd{代表点}。

\begin{example}[自由粒子]
  对于一个 $r=3$ 的自由粒子,有
  \begin{braced}
    & p_x = m \dot{x}, \\
    & p_y = m \dot{y}, \\
    & p_z = m \dot{z}.
  \end{braced}
  其 μ 空间由 $(x, \, y, \, z; \, p_x, \, p_y, \, p_z)$ 张成。Hamilton 量为
  \begin{equation}
    H = \frac{1}{2m} \qty\big(p_x^2+p_y^2+p_z^2).
  \end{equation}
% TODO: (2016-05-04) 示意图
\end{example}

\begin{example}[一维谐振子]
  质量为 $m$ 的物体受力 $F=-Ax$ 的作用做简谐运动,其角频率等于
  \begin{equation}
    \omega = \sqrt{\frac{A}{m}},
  \end{equation}
  而 Hamilton 量为
  \begin{equation}
    H = \frac{p^2}{2m} + \frac{A}{2} x^2
      = \frac{p^2}{2m} + \frac{1}{2} m \omega^2 x^2.
  \end{equation}

  若总能量一定,即 $H=E$,则
  \begin{equation}
    \frac{p^2}{2mE} + \frac{x^2}{2E / m \omega^2} = 1,
  \end{equation}
  这在 μ 空间中表示一个椭圆(见图~\ref{fig:harmonic-oscillator-mu-space}),其面积为
  \begin{equation} \label{eq:ellipse-of-harmonic-oscillator-mu-space}
    S_\text{ellipse} = \pp ab
      = \pp \cdot \sqrt{\frac{2E}{m \omega^2}} \cdot \sqrt{2mE}
      = \frac{2 \pp E}{\omega}.
  \end{equation}
\end{example}

\begin{figure}[ht]
  \centering
  \FIGPLACEHOLDER
  \caption{μ 空间中的一维谐振子}
  \label{fig:harmonic-oscillator-mu-space}
\end{figure}

\begin{example}[转子]
  如图~\ref{fig:rotator} 所示,质量为 $m$ 的物体被轻杆连接在 $O$ 点处,并可绕该点运动。其 Hamilton
  量为
  \begin{equation} \label{eq:rotator-hamiltonian}
    H = \frac{m}{2} \qty\big(\dot{x}^2+\dot{y}^2+\dot{z}^2).
  \end{equation}
  取球坐标系,则有
  \begin{braced}
    x &= r \sin{\theta}\cos{\phi}, \\
    y &= r \sin{\theta}\sin{\phi}, \\
    z &= r \cos{\theta}.
  \end{braced}
  求导,得
  \begin{braced}
    \dot{x} &= \dot{r}\sin{\theta}\cos{\phi}
             + r\dot{\theta}\cos{\theta}\cos{\phi}
             - r\dot{\phi}\sin{\theta}\sin{\phi}, \\
    \dot{y} &= \dot{r}\sin{\theta}\sin{\phi}
             + r\dot{\theta}\cos{\theta}\sin{\phi}
             + r\dot{\phi}\sin{\theta}\cos{\phi}, \\
    \dot{z} &= \dot{r}\cos{\theta} - r\dot{\theta}\sin{\theta}.
  \end{braced}
  计算可知
  \begin{equation}
    \dot{x}^2 + \dot{y}^2 + \dot{z}^2
    = \dot{r}^2 + (r\dot{\theta})^2 + (r\dot{\phi})^2 \sin^2\theta.
  \end{equation}
  代入 \eqref{eq:rotator-hamiltonian}~式,得
  \begin{align}
    H &= \frac{m}{2} \qty\big(\dot{r}^2 + r^2 \dot{\theta}^2
                              + r^2 \dot{\phi}^2 \sin^2\theta).
  \end{align}
  由于物体已被轻杆连接在了 $O$ 点,因而 $r$ 不变、$\dot{r}=0$。

  引入\kwd{共轭动量}
  \begin{braced}
    p_\theta &= m r^2 \dot{\theta}, \\
    p_\phi   &= m r^2 \dot{\phi} \sin^2\theta,
  \end{braced}
  则系统的 Hamilton 量可写为
  \begin{equation}
    H = \frac{1}{2I} \qty(p_\theta^2 + \frac{1}{\sin^2\theta} p_\phi^2),
  \end{equation}
  其中的 $I = mr^2$ 是物体关于 $O$ 点的\kwd{转动惯量}。

  在本例中,μ 空间由广义坐标和广义动量
  $(\theta, \, \phi; \, p_\theta, \, p_\phi)$ 张成,它是四维的。
\end{example}

\begin{figure}[ht]
  \centering
  \FIGPLACEHOLDER
  \caption{转子的示意图}
  \label{fig:rotator}
\end{figure}

\subsection{单粒子的量子描述}

微观态的量子描述以量子力学为基础。粒子的动量 $\V{p}$、能量 $E$ 满足 \kwd{de Broglie关系}:
\begin{braced}
  \V{p} &= \hbar\V{k}, \\
  E     &= \hbar\omega,
\end{braced}
其中的 $\hbar$ 称为\kwd{(约化)Planck 常数},其值为
\begin{equation}
  \hbar = \frac{h}{2\pp} = \SI{1.0545718e-34}{\joule\second}.
\end{equation}
式中的 $h$ 也称 Planck 常数。

De Broglie 关系说明微观粒子具有\kwd{波粒二象性}。这就引出了另一个重要结果——\kwd{不确定关系}:
\begin{equation}
  \incr{p} \incr{q} \gtrsim h.
  \footnote{更精确的表述为 $\incr{p} \incr{q} \geqslant \hbar / 2 $。}
\end{equation}
可见,在 $\incr{p}\to 0$ 时,必有 $\incr{q}\to\infty$。这说明粒子的动量和坐标不可能被同时精确测量,
因而其运动也就无法用经典的轨道概念来描述,必须改用\kwd{波函数}。

粒子波函数 $\Psi$ 满足的方程即 \kwd{Schrödinger 方程}:
\begin{equation}
  \ii \pdv{t} \Psi = \hat{H} \Psi,
\end{equation}
式中的 $\hat{H}$ 是 Hamilton 算符。在定态情况(即将时间变量分离后),Schrödinger 方程化为
\begin{equation}
  \hat{H} \psi = E \psi.
\end{equation}

\begin{example}[箱中的自由粒子]
  设粒子在边长为 $L$ 的立方体容器内运动,则其量子态(即波函数)有平面波的形式:
  \begin{equation}
    \Psi_{n_1, \, n_2, \, n_3} (\V{r})
    \sim \ee^{\ii\V{p}\cdot\V{r} / \hbar} \qc
    n_i = \pm 1, \, \pm 2, \, \pm 3, \, \cdots
  \end{equation}
  求解 Schrödinger 方程,可以发现动量与能量的本征值都是量子化的:
  \begin{braced}
    \V{p} &= p_x\V{i} + p_y\V{j} + p_z\V{k}
           = \frac{2\pp\hbar}{L} n_1\V{i} + n_2\V{j} + n_3\V{k}, \\
        E &= \frac{1}{2m} \qty\big(p_x^2 + p_y^2 + p_z^2)
           = \frac{2\pp^2 \hbar^2}{mL^2} \qty\big(n_1^2 + n_2^2 + n_3^2).
  \end{braced}
  量子化的能量也成为\kwd{能级}。对于能级
  \begin{equation}
    E = \frac{2\pp^2\hbar^2}{mL^2},
  \end{equation}
  它对应6种量子态:
  \begin{equation}
    (\pm 1, \, 0, \, 0) \qc (0, \, \pm 1, \, 0) \qc (0, \, 0, \, \pm 1).
  \end{equation}
  这种现象称为能级\kwd{简并}。同一能级对应量子态的数目称为\kwd{简并度}。显然,这里的简并度为 6。而能
  量更高的一个能级
  \begin{equation}
    E = \frac{2\pp^2\hbar^2}{mL^2} \times 3
      = \frac{2\pp^2\hbar^2}{mL^2} \times (1+1+1)
  \end{equation}
  则对应 $2^3=8$ 个量子态,它的简并度为 8。
\end{example}

\begin{example}[一维谐振子]
  频率为 $\nu$ 的谐振子,其能量为
  \begin{equation}
    E_n = \qty(n+\frac{1}{2}) \, h\nu \qc n = 0, \, 1, \, 2, \, \cdots
  \end{equation}
  可见该系统的简并度 $g=1$。

  根据式~\eqref{eq:ellipse-of-harmonic-oscillator-mu-space},μ 空间中的椭圆面积为
  \begin{equation}
    S_n = \frac{2\pp E_n}{\omega} =\frac{E_n}{\nu} = \qty(n+\frac{1}{2}) \, h.
  \end{equation}
  因此两个相轨道之间的面积为 $h$,它对应一个量子态。
\end{example}

\begin{example}[转子]
% TODO: (2016-06-24) 角动量量子化
\end{example}

\subsection{多粒子系统}

对于 $N$ 个粒子组成的系统,设每个粒子的自由度为 $r$,则每个粒子可用 $2r$ 个变量
$(q_1, \, q_2, \, \cdots, q_r; \, \allowbreak p_1, \, p_2, \, \cdots, p_r)$ 来描述。此时,系统的总自
由度
\begin{equation}
  f = Nr,
\end{equation}
因而系统的运动需要用 $2f$ 个变量 $(q_1, \, q_2, \, \cdots, q_f; \, p_1, \, p_2, \, \cdots, p_f)$ 来
刻画。这些变量张成了一个 $2f$ 维空间,称为\kwd{Γ 空间},也叫\kwd{相空间}。Γ 空间中的一个点就表示系统
的一个微观状态,状态运动的微小范围可用体积元表示:
\begin{equation}
  \incr{\Omega} = \incr{q_1} \cdots \incr{q_f} \incr{p_1} \cdots \incr{p_f}.
\end{equation}
根据不确定关系,取 $\incr{p} \incr{q} \simeq h$,则有
\begin{equation}
  \incr{\Omega} = \incr{q_1} \incr{p_1} \cdots \incr{q_f} \incr{p_f} \simeq h^f.
\end{equation}
这是多粒子系统的相格大小。

统计物理研究的系统往往由大量\kwd{全同粒子}组成。全同粒子是指内禀性质,如质量、电荷、自旋等均完全相同
的粒子。根据量子力学,全同粒子具有不可分辨性。换句话说,全同粒子的交换不引起新的量子态。

设系统的波函数为 $\Psi$。引入\kwd{交换算符} $\hat{P}$,根据全同粒子的不可分辨性,可有
\begin{equation}
  \abs{\hat{P}\Psi}^2 = \abs{\Psi}^2 \implies \hat{P}\Psi = \pm\Psi.
\end{equation}

波函数交换对称,即 $\hat{P}\Psi=+\Psi$ 的粒子,称为 \kwd{Bose 子}。它们遵循
\kwd{Bose--Einstein 统计},并且自旋为整数。光子、π 介子、胶子以及 Higgs 粒子等都是 Bose 子。单一量子
态上可占据任意数目的 Bose 子。

波函数交换反对称,即 $\hat{P}\Psi=-\Psi$ 的粒子,称为 \kwd{Fermi 子}。它们遵循
\kwd{Fermi--Dirac 统计},自旋为半整数。电子、质子、中子以及夸克等都是 Fermi 子。全同 Fermi 子组成的
系统满足 \kwd{Pauli 不相容原理},即单一量子态上只可占据 0 个或 1 个 Fermi 子。

就基本粒子而言,Bose 子传递相互作用,而 Fermi 子组成物质。对于复合粒子,如果含有偶数个 Fermi 子,则
为 Bose 子;如果含有奇数个 Fermi 子,则为 Fermi 子。例如氦的同位素 \ce{^4He},它包含两个质子、两个中
子和两个电子,为 Bose 子;而 \ce{^3He} 则包含两个质子、一个中子和两个电子,为 Fermi 子。

对于多粒子系统,如果各粒子的波函数分别局限在空间不同范围内,彼此交叠很少,则称为\kwd{定域子系}。此
时,交换两个粒子的波函数(对应于量子态),可以看出系统的微观状态发生了变化(见图~%
\ref{fig:localized-sub-system}),因而即使是全同粒子,在定域子系中也可分辨。定域子系遵循
\kwd{Boltzmann 统计}。

\begin{figure}[ht]
  \centering
  \FIGPLACEHOLDER
  \caption{定域子系}
  \label{fig:localized-sub-system}
\end{figure}

下面我们举例说明单粒子量子态与多体量子态之间的关系。设系统由两个粒子组成,每个粒子可以处在四个量子态
下。不同情况下,粒子的所有分布方式列于表~\ref{tab:two-particles-distribution}。由此可知,对于定域子
系、非定域 Bose 子和非定域 Fermi 子,系统分别有 16 个、10 个和 6 个量子态。

\begin{table}[ht]
  \def\B{{\Large\symbol{"25CB}}}
  \let\TC=\textcircled
  \newcommand\STATE[4]{%
    \CJKunderline{\makebox[2em][c]{#1}}\kern1em%
    \CJKunderline{\makebox[2em][c]{#2}}\kern1em%
    \CJKunderline{\makebox[2em][c]{#3}}\kern1em%
    \CJKunderline{\makebox[2em][c]{#4}}}
  \newcommand\TITLE[1]{\makebox[11em][c]{#1}}
  \centering
  \caption{两个粒子在四个量子态中的分布情况}
  \label{tab:two-particles-distribution}
  \begin{tabular}{c|c|c}
    \toprule
      \multicolumn{3}{c}{%
        \begin{tabular}{@{}ccc@{}}
          \TITLE{定域子系} & \TITLE{非定域 Bose 子} & \TITLE{非定域 Fermi 子}
        \end{tabular}} \\
    \midrule
      \STATE{\TC1\TC2}{}{}{} & \STATE{\B\B}{}{}{} & \\
      \STATE{}{\TC1\TC2}{}{} & \STATE{}{\B\B}{}{} & \\
      \STATE{}{}{\TC1\TC2}{} & \STATE{}{}{\B\B}{} & \\
      \STATE{}{}{}{\TC1\TC2} & \STATE{}{}{}{\B\B} & \\
      \STATE{\TC1}{\TC2}{}{} & \STATE{\B}{\B}{}{} & \STATE{\B}{\B}{}{} \\
      \STATE{\TC2}{\TC1}{}{} &                    &                    \\
      \STATE{\TC1}{\TC2}{}{} & \STATE{\B}{}{\B}{} & \STATE{\B}{}{\B}{} \\
      \STATE{\TC2}{}{\TC1}{} &                    &                    \\
      \STATE{\TC1}{}{\TC2}{} & \STATE{\B}{}{}{\B} & \STATE{\B}{}{}{\B} \\
      \STATE{\TC2}{}{}{\TC1} &                    &                    \\
      \STATE{}{\TC1}{\TC2}{} & \STATE{}{\B}{\B}{} & \STATE{}{\B}{\B}{} \\
      \STATE{}{\TC2}{\TC1}{} &                    &                    \\
      \STATE{}{\TC1}{}{\TC2} & \STATE{}{\B}{}{\B} & \STATE{}{\B}{}{\B} \\
      \STATE{}{\TC2}{}{\TC1} &                    &                    \\
      \STATE{}{}{\TC1}{\TC2} & \STATE{}{}{\B}{\B} & \STATE{}{}{\B}{\B} \\
      \STATE{}{}{\TC2}{\TC1} &                    &                    \\
    \bottomrule
  \end{tabular}
\end{table}

至于一般情况,我们将在 \ref{sec:distribution-and-microstate}~节中讨论。

\section{宏观量;等概率原理}

\subsection{宏观量的统计性质}

统计物理学的基本观点是:宏观量是微观量的统计平均。宏观观测有如下特点:

\begin{itemize}
  \item 空间尺度上,宏观小、微观大;
  \item 时间尺度上,宏观短、微观长。
\end{itemize}

由此我们可以知道,任意宏观态都对应着非常多的微观态。

\subsection{统计规律性}

就微观层面而言,粒子的运动遵循\kwd{力学规律}。无论是经典力学中的 Newton 方程,还是量子力学中的
Schrödinger 方程,都具有\kwd{时间反演}对称性,因而是可逆的。但在宏观层面,考虑到热力学第二定律,热现
象过程具有不可逆性,即时间有确定方向。这是\kwd{统计规律}的体现。

力学规律是确定性的,一旦运动方程确定,系统在任意时刻的状态就可以确定;而统计规律则具有不确定性,在一
定宏观条件下,系统总是以一定概率处于某一微观状态。

造成统计规律的原因主要有下面两点:

\begin{itemize}
  \item 宏观态对应着数量极其巨大的微观态,这些微观态不可能由宏观状态唯一决定;
  \item 系统与环境之间总是不可避免地存在相互作用,而且这些相互作用又有一定的随机性。
\end{itemize}

\subsection{等概率原理}

\kwd{等概率原理}(equal a priori probability postulate)最早由 Boltzmann 提出,它可以表述为:对于平
衡态下的孤立系,各微观态出现的概率相同。这是统计物理学的基本假设。

\section{分布和微观状态} \label{sec:distribution-and-microstate}

% TODO: (2018-01-08) 交叉引用
在本节以及第%\ref{}
章中,我们均要求粒子之间相互作用可以忽略。因而总能量等于各粒子的能量之和。这样
的系统称为\kwd{近独立粒子系统}。

设粒子的能级为 $\varepsilon_1, \, \varepsilon_2, \, \cdots, \, \varepsilon_l, \, \cdots$,各能级的简
并度为 $\omega_1, \, \omega_2, \, \cdots, \, \omega_l, \, \cdots$。我们把每个能级上占据的粒子数
$a_1, \, a_2, \, \cdots, \, a_l, \, \cdots$ 称为一个\kwd{(微观)分布},简记为 $\qty{a_l}$。对于平衡
态下的孤立系,能量 $E$、体积 $V$、粒子数 $N$ 都是给定的。因此分布 $\qty{a_l}$ 需要满足
\begin{braced}[\label{eq:constraint-condition-of-dist}]
  & \sum_l a_l = N,               \label{eq:constraint-condition-N-of-dist} \\
  & \sum_l \varepsilon_l a_l = E. \label{eq:constraint-condition-E-of-dist}
\end{braced}

注意分布与量子态是不同的概念。
% TODO: (2018-01-08) 不同之处?

\subsection{Boltzmann 体系}

Boltzmann 体系由定域子系组成。在能级 $\varepsilon_l$ 上,有 $\omega_l$ 个量子态(简并度),并占据着
$a_l$ 个粒子。粒子所处的量子态互不影响,且粒子可以编号。根据乘法原理,所有分布情况数等于每个粒子的可
能量子态数目之积,即 ${\omega_l}^{a_l}$。对于整个体系而言,需将所有能级上的量子态数目乘起来,即
$\prod_l {\omega_l}^{a_l}$。

分布 $\qty{a_l}$ 仅仅指定了粒子的数目,而没有指定具体是哪些粒子,所以还要考虑不同能级之间粒子数的交
换。交换 $N$ 个粒子的所有可能方式(全排列)共有 $\Pnum{N}{N}=N!$ 种。而单个能级上粒子的交换并不会造
成任何改变
\footnote{在能级 $\varepsilon_l$ 上,粒子交换带来微观状态的变化,已经计算在了
  ${\omega_l}^{a_l}$ 中。},
因此需要除去所有能级上粒子的交换数 $\prod_l a_l!$。

综上,分布 $\qty{a_l}$ 对应的总量子态数为
\begin{equation} \label{eq:number-of-states-in-boltzmann-dist}
  \Omega_\text{MB}(\qty{a_l}) = \frac{N!}{\prod_l a_l!} \prod_l {\omega_l}^{a_l}.
\end{equation}
下标“$\text{MB}$”表示粒子服从 Maxwell--Boltzmann 统计。

\subsection{Bose--Einstein 体系} \label{subsec:bose-einstein-system}

Bose--Einstein 体系由 Bose 子组成。相比定域子系,Bose 子具有不可分辨性,而且在每个量子态上可占据任意
数量。因而 $a_l$ 个 Bose 子在 $\omega_l$ 个量子态上的分布,就相当于 $a_l$ 个相同的小球放在
$\omega_l$ 个不同的盒子中。考虑使用“插空法”,即先将 $a_l$ 个小球和 $\omega_l-1$ 个挡板排列(此时已经
将球分成了 $\omega_l$ 份),再从中选出 $a_l$ 个位置放小球。可见,共有
\begin{equation}
  \Cnum{a_l+\omega_l-1}{a_l} = \frac{\qty(a_l+\omega_l-1)!}{a! \, \qty(\omega_l-1)!}
\end{equation}
种情况。

由于不可分辨性,这里不需要考虑不同能级上粒子的交换。所以总量子态数为
\begin{equation} \label{eq:number-of-states-in-bose-dist}
  \Omega_\text{BE}(\qty{a_l}) = \prod_l \frac{\qty(a_l+\omega_l-1)!}{a! \, \qty(\omega_l-1)!}.
\end{equation}
下标“$\text{BE}$”表示粒子服从 Bose--Einstein 统计。

\subsection{Fermi--Dirac 体系}

Fermi--Dirac 体系由 Fermi 子组成。根据 Pauli 不相容原理,每个量子态上最多只能占据一个 Fermi 子,这相
当于从 $\omega_l$ 个量子态中选出 $a_l$ 个态来让粒子分布。因此共有
\begin{equation}
  \Cnum{\omega_l}{a_l} = \frac{\omega_l!}{a_l! \, \qty(\omega_l-a_l)!}
\end{equation}
种情况。显然,此时要求 $k \leqslant n$。

与 Bose--Einstein 体系相同,这里同样不需要考虑粒子的交换。总量子态数为
\begin{equation} \label{eq:number-of-states-in-fermi-dist}
  \Omega_\text{FD}(\qty{a_l}) = \prod_l \frac{\omega_l!}{a_l! \, \qty(\omega_l-a_l)!}.
\end{equation}
下标“$\text{FD}$”表示粒子服从 Fermi--Dirac 统计。

\subsection{非简并条件(I)} \label{subsec:non-degenerate-condition-i}

当每个量子态上占据的粒子非常“稀薄”,即
\begin{equation} \label{eq:non-degenerate-condition}
  \frac{a_l}{\omega_l} \ll 1,
\end{equation}
且 $\omega_l$ 很大时,我们有
\begin{braced}[\label{eq:bose-and-fermi-dist-non-degenerate}]
  \Omega_\text{BE}(\qty{a_l}) &= \prod_l \frac{\qty(a_l+\omega_l-1)!}{a! \, \qty(\omega_l-1)!}
  = \prod_l \frac{\qty(\omega_l+a_l-1) \qty(\omega+a_l-2) \cdots \omega_l}{a_l!}
  \approx \prod_l \frac{{\omega_l}^{a_l}}{a_l!}
  = \frac{\Omega_\text{MB}(\qty{a_l})}{N!}, \\
  \Omega_\text{FD}(\qty{a_l}) &= \prod_l \frac{\omega_l!}{a_l! \, \qty(\omega_l-a_l)!}
  = \prod_l \frac{\omega_l \qty(\omega_l-1) \cdots \qty(\omega-a_l+1)}{a_l!}
  \approx \prod_l \frac{{\omega_l}^{a_l}}{a_l!}
  = \frac{\Omega_\text{MB}(\qty{a_l})}{N!}.
\end{braced}
这说明在\kwd{非简并条件}式~\eqref{eq:non-degenerate-condition} 下,Bose--Einstein 分布、Fermi--%
Dirac 分布与 Maxwell--Boltzmann 分布的差别将逐渐消失。

注意式~\eqref{eq:bose-and-fermi-dist-non-degenerate} 最后的结果中有一个 $1/N!$,这反映了粒子的全同性
要求。

\section{Maxwell--Boltzmann 分布}

根据等概率原理,平衡态下各微观态出现的概率相同。因此,对应微观态数目最多的分布,出现的概率最大,称
为\kwd{最概然分布}。

对于 Boltzmann 体系,即定域子系组成的系统,分布 $\qty{a_l}$ 对应的微观态(量子态)数目
$\Omega_\text{MB}(\qty{a_l})$ 由式~\eqref{eq:number-of-states-in-boltzmann-dist} 给出。现在我们做要
做的,就是找到使 $\Omega_\text{MB}(\qty{a_l})$ 取极大值的分布 $\qty{a_l}$。但需注意,系统的粒子数
$N$、能量 $E$ 还需满足如下关系:
\begin{braced*}
  & \sum_l a_l = N,               \tag*{\eqref{eq:constraint-condition-N-of-dist}} \\
  & \sum_l \varepsilon_l a_l = E. \tag*{\eqref{eq:constraint-condition-E-of-dist}}
\end{braced*}
因而这是一个带约束条件的极值问题,数学上的标准求解手段是 \kwd{Lagrange 乘子法}。下面给出具体的操作。

为处理方便,我们把求 $\Omega$
\footnote{本节以下部分将略去下标“MB”。}
的极大值改为求 $\ln\Omega$ 的极大值。利用
\kwd{Stirling 公式}
\begin{equation}
  n!     \approx \sqrt{2\pp n} \qty(\frac{n}{\ee})^n \qc
  \ln n! \approx n\,\qty(\ln n-1) + \frac{1}{2}\ln(2\pp n)
         \approx n\,\qty(\ln n-1) \qc
  n \gg 1,
\end{equation}
可有
\begin{align}
  \ln\Omega
  &= \ln(\frac{N!}{\prod_l a_l!} \prod_l {\omega_l}^{a_l})
   = \ln N! - \sum_l\ln a_l! + \sum_l a_l \ln\omega_l \notag \\
  &\approx N\,\qty(\ln N-1) - \sum_l a_l\,\qty(\ln a_l-1) + \sum_l a_l\ln\omega_l \notag
  \intertext{利用约束条件 $N=\sum_l a_l$:}
  &= N\ln N - \sum_l a_l\ln a_l + \sum_l a_l\ln\omega_l \notag \\
  &= N\ln N - \sum_l a_l\ln\frac{\omega_l}{a_l}.
\end{align}

极值点要求满足 $\var{\ln\Omega}=0$
\footnote{这里我们用了变分记号“$\var{}$”,用以强调 $\var{a_l}$ 并非真实变动。具体计算上,与普通微分
相同。}:
\begin{equation} \label{eq:delta-Omega=0}
  \var{\ln\Omega}
  = \var(N\ln N) - \sum_l \qty(\var{a_l}\ln\frac{\omega_l}{a_l}
                               -a_l\cdot\frac{1}{a_l}\cdot\var{a_l})
  = -\sum_l \qty(\ln\frac{a_l}{\omega_l}+1) \var{a_l} = 0.
\end{equation}
而根据约束条件则有
\begin{equation}
  \var{N} = \sum_l \var{a_l} = 0 \qc
  \var{E} = \sum_l \varepsilon_l \var{a_l} = 0.
\end{equation}
两式分别乘上系数 $\alpha-1$ 和 $\beta$(称为 \kwd{Lagrange 不定乘子}),并从式~%
\eqref{eq:delta-Omega=0} 中减去,可得:
\begin{equation}
  \var{\ln\Omega} - \qty(\alpha-1)\var{N} - \beta\var{E}
  = -\sum_l\qty(\ln\frac{a_l}{\omega_l}+\alpha+\beta\varepsilon_l) \var{a_l} = 0.
\end{equation}
根据 Lagrange 乘子法,令 $\var{a_l}$ 的系数为零,有
\begin{equation}
  \ln\frac{a_l}{\omega_l}+\alpha+\beta\varepsilon_l = 0 \qc l = 1, \, 2, \, \cdots
\end{equation}
这样就得到了最概然分布
\begin{equation} \label{eq:maxwell-boltzmann-dist}
  \tilde{a_l} = \omega_l \, \ee^{-\alpha-\beta\varepsilon_l}.
\end{equation}
式中的 Lagrange 不定乘子 $\alpha$ 与 $\beta$,可根据约束关系式~%
\eqref{eq:constraint-condition-of-dist} 确定,我们将在稍后下一章中计算。
% TODO: (2018-01-07) 下一章的计算,交叉引用

计算二阶变分:
\begin{align}
  \var[2]{\ln\Omega}
  &= -\sum_l \qty[  \qty(\frac{1}{a_l}\cdot\var{a_l})\var{a_l}
                  + \qty(\ln\frac{a_l}{\omega_l}+1) \var[2]{a_l}] \notag
  \intertext{忽略 $a_l$ 的二阶变分,得}
  &= -\sum_l \frac{\qty(\var{a_l})^2}{a_l} < 0.
\end{align}
$a_l>0$,所以该式恒小于零。这就验证了最概然分布 $\qty{\tilde{a_l}}$ 的确对应 $\ln\Omega$ 的极大值。
式~\eqref{eq:maxwell-boltzmann-dist} 给出的结果即为 \kwd{Maxwell--Boltzmann 分布}。

设粒子数分布相对最概然分布 $\qty{\tilde{a_l}}$ 有一微小偏离 $\var{a_l}$,并使得量子态数目变化了
$\var{\Omega}$。利用 Taylor 展开,可有
\begin{align}
  \ln(\Omega+\incr{\Omega})
  = \ln\Omega\qty(\qty{\tilde{a_l}+\var{a_l}})
  &= \ln\Omega(\qty{\tilde{a_l}}) + \eval{\var{\ln\Omega}}_\qty{\tilde{a_l}}
     + \frac{1}{2} \eval{\var[2]{\ln\Omega}}_\qty{\tilde{a_l}} + \cdots \notag \\
  &= \ln\Omega(\qty{\tilde{a_l}}) + 0 - \frac{1}{2} \sum_l \frac{\qty(\var{a_l})^2}{\tilde{a_l}},
\end{align}
即
\begin{equation}
  \ln\frac{\Omega+\incr{\Omega}}{\Omega}
  = -\frac{1}{2} \sum_l \qty(\frac{\var{a_l}}{\tilde{a_l}})^2 \tilde{a_l}
  \implies \frac{\Omega+\incr{\Omega}}{\Omega}
           = \exp[-\frac{1}{2} \sum_l \qty(\frac{\var{a_l}}{\tilde{a_l}})^2 \tilde{a_l}].
\end{equation}
在热力学极限下,$\sum_l a_l=N \sim 10^{23}$,即使 $\var{a_l}$ 很小(如 $10^{-5}$),也有
\begin{equation}
  \frac{\Omega+\incr{\Omega}}{\Omega}
  \sim \exp(-\frac{1}{2}\times\num{e-10}\times\num{e23}) \lll 1.
\end{equation}
这说明 $\Omega$ 具有尖锐成锋的特性,类似数学中的 Dirac-$\delta$ 函数。

\subsection{经典统计中 Boltzmann 分布的表达式}

之前的讨论建立在能级分立的量子情形上,现在将其推广到经典情形。我们需要在维度为 $r$ 的 μ 空间中划分
出等大的\kwd{相格},其体积为
\begin{equation}
  \qty\big(\var{q_1}\var{q_2}\cdots\var{q_r}) \, \qty\big(\var{p_1}\var{p_2}\cdots\var{p_r})
  = (\var{q_1}\var{p_1}) \, (\var{q_2}\var{p_2}) \, \cdots (\var{q_r}\var{p_r}) = h_0^r.
\end{equation}
式中的 $h_0=\incr{q_i}\incr{p_i}$ 为相格的尺度。当相格体积很小时,同一相格中不同代表点所代表的粒子,
其运动状态的差别可以忽略不计,即所有代表点都具有近似相同的广义坐标 $q_i$ 和广义动量 $p_i$。显然,相
格的尺度取得越小,对运动的刻画就越精确。但是,量子力学为相格的尺度给出了下限,为 \kwd{Planck 常数}
\begin{equation}
  h = \SI{6.626070040(81)e-34}{\joule\second}.
\end{equation}
这种近似通常被称为 μ 空间的\kwd{粗粒近似}。

再将 μ 空间中划分为许多体积元 $\incr{\omega_l}$。按照粗粒近似的观点,我们认为体积元
$\incr{\omega_l}$ 中有大量的代表点,它们均处在同一能级 $\varepsilon_l$ 上。能级的简并度 $\omega_l$
与 $\incr{\omega_l}$ 的体积成正比,取为
\begin{equation}
  \omega_l = \frac{\incr{\omega_l}}{h_0^r}.
\end{equation}
此时,类比式~\eqref{eq:number-of-states-in-boltzmann-dist} 和 \eqref{eq:maxwell-boltzmann-dist},我
们即可写出经典情形下的 Boltzmann 体系的微观态数目
\begin{equation}
  \Omega_\text{classical}
  = \frac{N!}{\prod_l a_l!} \prod_l \qty(\frac{\incr{\omega_l}}{h_0^r})^{a_l}
\end{equation}
以及 Maxwell--Boltzmann 分布
\begin{equation}
  a_{l,\,\text{classical}} = \frac{\incr{\omega_l}}{h_0^r} \, \ee^{-\alpha-\beta\varepsilon_l}.
\end{equation}
$a_{l,\,\text{classical}}$ 满足也约束 \eqref{eq:constraint-condition-of-dist}~式:
\begin{equation}
  \sum_l a_l = N \qc \sum_l \varepsilon_l a_l = E.
\end{equation}

\section{Bose--Einstein 分布与 Fermi--Dirac 分布}

\subsection{Bose--Einstein 分布}

在 \ref{subsec:bose-einstein-system}~小节中,我们已经得到了 Bose--Einstein 分布的量子态数目:
\begin{equation}
  \Omega_\text{BE}(\qty{a_l}) = \prod_l \frac{\qty(a_l+\omega_l-1)!}{a! \, \qty(\omega_l-1)!}.
  \tag*{\eqref{eq:number-of-states-in-bose-dist}}
\end{equation}
于是有
\begin{align}
  \ln\Omega_\text{BE}
  &=       \sum_l \qty\big[\ln(a_l+\omega_l-1)! - \ln a_l! - \ln(\omega_l-1)!] \notag
  \intertext{取 $a_l \gg 1$ 和 $\omega \gg 1$,忽略常数 1:}
  &\approx \sum_l \qty\big[\ln(a_l+\omega_l)! - \ln a_l! - \ln\omega_l!] \notag
  \intertext{利用 Stirling 公式:}
  &\approx \sum_l \qty\Big{  (a_l+\omega_l) \, \qty\big[\ln(a_l+\omega_l)-1]
                           - a_l \, \qty(\ln a_l-1) - \omega_l \, \qty(\ln\omega_l-1)} \notag \\
  &=       \sum_l \qty\big[  (a_l+\omega_l) \, \ln(a_l+\omega_l)
                           - a_l\ln a_l - \omega_l\ln\omega_l].
\end{align}
对其求变分,得
\begin{equation}
  \var{\ln\Omega_\text{BE}}
  = \sum_l \qty\big[\var{a_l}\ln(a_l+\omega_l) + \var{a_l} - \var{a_l}\ln a_l - \var{a_l}]
  = \sum_l \ln(\frac{\omega_l}{a_l}+1) \var{a_l}.
\end{equation}
根据极值点条件 $\var{\Omega_\text{BE}}=0$,并引入 Lagrange 乘子 $\alpha$、$\beta$,可有
\begin{equation}
  \var{\ln\Omega_\text{BE}} - \alpha\var{N} - \beta\var{E}
  = \sum_l \qty[\ln(\frac{\omega_l}{a_l}+1) - \alpha - \beta\varepsilon_l] \var{a_l} = 0.
\end{equation}
令 $\var{a_l}$ 的系数为零,立即得到 \kwd{Bose--Einstein 分布}
\begin{equation}
  a_l = \frac{\omega_l}{\ee^{\alpha+\beta\varepsilon_l}-1}.
\end{equation}
其中,$\alpha$、$\beta$ 需满足
\begin{equation}
  N = \sum_l a_l = \sum_l \frac{\omega_l}{\ee^{\alpha+\beta\varepsilon_l}-1} \qc
  E = \sum_l a_l \varepsilon_l = \sum_l \frac{\omega_l\varepsilon_l}%
                                        {\ee^{\alpha+\beta\varepsilon_l}-1}.
\end{equation}

\subsection{Fermi--Dirac 分布}

Fermi--Dirac 分布的推导是完全类似的。仍然从量子态数目
\begin{equation}
  \Omega_\text{FD}(\qty{a_l}) = \prod_l \frac{\omega_l!}{a_l! \, \qty(\omega_l-a_l)!}
  \tag*{\eqref{eq:number-of-states-in-fermi-dist}}
\end{equation}
出发,我们有
\begin{align}
  \ln\Omega_\text{FD}
  &=       \sum_l \qty\big[\ln\omega_l! - \ln a_l! - \ln(\omega_l-a_l)!] \notag \\
  &\approx \sum_l \qty\big[\omega_l\ln\omega_l - a_l\ln a_l
                           - (\omega_l-a_l) \, \ln(\omega_l-a_l)], \\
  \var{\ln\Omega_\text{FD}}
  &= \sum_l \ln(\frac{\omega_l}{a_l}-1) \var{a_l}.
\end{align}
根据极值点条件 $\var{\Omega_\text{FD}}=0$,并引入 Lagrange 乘子 $\alpha$、$\beta$,同理可有
\begin{equation}
  \var{\ln\Omega_\text{FD}} - \alpha\var{N} - \beta\var{E}
  = \sum_l \qty[\ln(\frac{\omega_l}{a_l}-1) - \alpha - \beta\varepsilon_l] \var{a_l} = 0.
\end{equation}
令 $\var{a_l}$ 的系数为零,得到 \kwd{Fermi--Dirac 分布}
\begin{equation}
  a_l = \frac{\omega_l}{\ee^{\alpha+\beta\varepsilon_l}+1}.
\end{equation}
其中,$\alpha$、$\beta$ 需满足
\begin{equation}
  N = \sum_l a_l = \sum_l \frac{\omega_l}{\ee^{\alpha+\beta\varepsilon_l}+1} \qc
  E = \sum_l a_l \varepsilon_l = \sum_l \frac{\omega_l\varepsilon_l}%
                                        {\ee^{\alpha+\beta\varepsilon_l}+1}.
\end{equation}
与 Bose--Einstein 分布相比,只有分母上的 “$-1$” 变成了 “$+1$”,其余完全一致。

值得注意的是,在推导 Fermi--Dirac 分布时,我们也需要近似 $a_l \gg 1$。这显然是荒谬的,因为 Pauli 不
相容原理要求每个量子态上最多占据一个粒子。可见,尽管这里给出的结果是正确的,但其理论基础却并不完善。
利用系综理论,才可以比较好地解决这一问题。

\subsection{非简并条件(II)}

我们这里再把三种分布全部列出:

\begin{table}[ht]
  \centering
  \caption{三种分布}
  \begin{tabular}{ccc}
    \toprule
      Boltzmann--Maxwell 分布 & Bose--Einstein 分布 & Fermi--Dirac 分布 \\
    \midrule \addlinespace[1.2ex]
      $\displaystyle a_l = \frac{\omega_l}{\ee^{\alpha+\beta\varepsilon_l}}$   &
      $\displaystyle a_l = \frac{\omega_l}{\ee^{\alpha+\beta\varepsilon_l}-1}$ &
      $\displaystyle a_l = \frac{\omega_l}{\ee^{\alpha+\beta\varepsilon_l}+1}$ \\[1.5ex]
    \bottomrule
  \end{tabular}
\end{table}

可以发现,当 $\ee^{\alpha} \gg 1$ 时,Bose--Einstein 分布 / Fermi--Dirac 分布中分母上的 $\pm 1$ 可以
忽略,即它们均趋同于 Boltzmann--Maxwell 分布。稍加变形可知,$\ee^{\alpha} \gg 1$ 时有
\begin{equation}
  \frac{a_l}{\omega_l} = \frac{1}{\ee^{\alpha+\beta\varepsilon_l}+\eta} \ll 1.
\end{equation}
式中的 $\eta$ 取 $0, \, \pm 1$ 正好对应三种分布。这正是我们在
\ref{subsec:non-degenerate-condition-i}~小节中提到的\kwd{非简并条件}。

    % 三大统计
    %% Copyright (C) 2016--2018 by Xiangdong Zeng <pssysrq@163.com>
%%
%% -* Notes on Thermodynamics and Statistical Physics *-
%%
%% This file may be distributed and/or modified under the
%% Creative Commons Attribution Share Alike 4.0 license.

\chapter{Boltzmann、Bose 与 Fermi 统计} \label{chap:three-distributions}

本章我们将从第\ref{chap:statistical-physics-basis}章获得的结果出发,推导一些重要的物理结论。

\section{定域子系热力学量的统计表述} \label{sec:thermodynamics-in-boltzmann-dist}

\subsection{单粒子配分函数}

定域子系遵循 Boltzmann 统计。其粒子数和内能由式~\eqref{eq:constraint-condition-of-dist} 给出:
\begin{braced}[\label{eq:constraint-condition-of-dist-restate}]
  N &=     \sum_l a_l = \sum_l \omega_l\ee^{-\alpha-\beta\varepsilon_l}, \\
  U &= E = \sum_l \varepsilon_l a_l = \sum_l \varepsilon_l\omega_l\ee^{-\alpha-\beta\varepsilon_l}.
\end{braced}
定义\kwd{单粒子配分函数}
\begin{equation}
  Z_1 \defeq \sum_l \omega_l\ee^{-\beta\varepsilon_l},
\end{equation}
则有
\begin{braced}[\label{eq:N-U-from-Z1}]
  N &= \sum_l a_l = \ee^{-\alpha} \sum_l \omega_l\ee^{-\beta\varepsilon_l}
     = \ee^{-\alpha} Z_1, \\
  U &= \sum_l \varepsilon_l a_l
     = \ee^{-\alpha} \, \qty(-\pdv{\beta} \sum_l \omega_l\ee^{-\beta\varepsilon_l})
     = \frac{N}{Z_1} \, \qty(-\pdv{\beta} Z_1) = -N\pdv{\beta} \ln Z_1.
\end{braced}

广义力 $Y$ 也可根据单粒子配分函数确定:
% TODO: (2018/01/27) 广义力
\begin{align}
  Y &= \sum_l \pdv{\varepsilon_l}{y} a_l
     = \sum_l \pdv{\varepsilon_l}{y} \omega_l\ee^{-\alpha-\beta\varepsilon_l} \notag \\
    &= \ee^{-\alpha} \, \qty(-\frac{1}{\beta} \pdv{y} \sum_l \omega_l\ee^{-\beta\varepsilon_l})
     = \frac{N}{Z_1} \, \qty(-\frac{1}{\beta} \pdv{y} Z_1) = -\frac{N}{\beta} \pdv{y} \ln Z_1.
\end{align}
在 $p$-$V$ 系统中,取 $Y=-p$ 和 $y=V$,可得
\begin{equation}
  p = \frac{N}{\beta} \pdv{V} \ln Z_1.
\end{equation}

\subsection{熵与 Boltzmann 关系}

下面我们来计算熵。根据热力学基本微分方程,我们有
\begin{equation}
  T\dd{S} = \dd{U} - Y\dd{y} = - N \dd(\pdv{\beta}\ln Z_1)
                               + \qty(\frac{N}{\beta}\pdv{y}\ln Z_1) \dd{y}.
\end{equation}
为了凑出全微分,需要在两边同乘 $\beta$:
\begin{align}
  \beta T\dd{S}
  &= N \, \qty[\qty(\pdv{y}\ln Z_1)\dd{y} - \beta\dd(\pdv{\beta}\ln Z_1)] \notag
  \intertext{注意到 $Z_1$ 只是 $\beta$ 和 $y$ 的函数,因此有}
  &= N \, \qty[\dd{\ln Z_1} - \qty(\pdv{\beta}\ln Z_1)\dd{\beta}
            - \dd(\beta\pdv{\beta}\ln Z_1) + \qty(\pdv{\beta}\ln Z_1)\dd{\beta}] \notag \\
  &= N \dd(\ln Z_1 - \beta\pdv{\beta}\ln Z_1).
\end{align}
令
% TODO: (2018/02/10) 与理想气体类比,如何操作
\begin{equation}
  \beta = \frac{1}{kT},
\end{equation}
立即得到
\begin{equation}
  \dd{S} = Nk \dd(\ln Z_1 - \beta\pdv{\beta}\ln Z_1).
\end{equation}
积分之后,可有
\begin{equation}
  S = Nk \, \qty(\ln Z_1 - \beta\pdv{\beta}\ln Z_1) + S_0
    = Nk \, \qty(\ln Z_1 - \beta\pdv{\beta}\ln Z_1).
\end{equation}
式中,我们取积分常数 $S_0=0$。
% TODO: (2018/02/10) 为什么这样取积分常数

代入粒子数和内能的表达式~\eqref{eq:N-U-from-Z1},可得
\begin{align}
  S &= Nk \, \qty(\ln N + \alpha + \beta\cdot\frac{U}{N}) \notag \\
    &= k \, \qty(N\ln N + \alpha N + \beta U) \notag
  \intertext{进一步代入式~\eqref{eq:constraint-condition-of-dist-restate}:}
    &= k \, \qty[N\ln N + \sum_l \qty(\alpha+\beta\varepsilon_l) a_l].
\end{align}
根据 Boltzmann 分布 \eqref{eq:maxwell-boltzmann-dist}~式
\begin{equation}
  a_l = \omega_l \, \ee^{-\alpha-\beta\varepsilon_l}
  \implies \alpha+\beta\varepsilon_l = -\ln\frac{a_l}{\omega_l},
\end{equation}
于是
\begin{equation} \label{eq:S-in-boltzmann-dist}
  S = k \, \qty(N\ln N - \sum_l a_l\ln\frac{a_l}{\omega_l}).
\end{equation}
另一方面,Boltzmann 体系中微观态数目为
\begin{equation}
  \Omega_\text{MB} = \frac{N!}{\prod_l a_l!} \prod_l {\omega_l}^{a_l}.
  \tag*{\eqref{eq:Omega-in-boltzmann-dist}}
\end{equation}
利用 Stirling 公式,计算得到
\footnote{推导可见 \ref{sec:maxwell-boltzmann-dist}~节。}
\begin{equation}
  \ln\Omega_\text{MB} = N\ln N - \sum_l a_l\ln\frac{a_l}{\omega_l}.
\end{equation}
与式~\eqref{eq:S-in-boltzmann-dist} 比较,立即有
\begin{equation}
  S = k\ln\Omega_\text{MB}.
\end{equation}
这称为 \kwd{Boltzmann 关系}。

在非简并条件下,Fermi--Dirac 分布与 Bose--Einstein 分布均趋同于 Boltzmann 分布(见
\ref{subsec:non-degenerate-condition-i}~小节):
\begin{equation}
  \Omega_\text{FD} = \Omega_\text{BE} = \frac{\Omega_\text{MB}}{N!}.
\end{equation}
此时,熵为
\begin{equation}
  S = Nk \, \qty(\ln Z_1 - \beta\pdv{\beta}\ln Z_1) - k\ln N! = k\ln\frac{\Omega_\text{MB}}{N!}.
\end{equation}
这里的 $N!$ 仍然是粒子全同性的要求。

利用熵和内能的表达式,还可从配分函数获得自由能:
\begin{equation}
  F = U-TS = -N\pdv{\beta}\ln Z_1 - NkT \, \qty(\ln Z_1 - \beta\pdv{\beta}\ln Z_1)
    = -NkT\ln Z_1.
\end{equation}
如果在非简并条件下(考虑粒子全同性),则为
\begin{equation}
  F = -NkT\ln Z_1 + k\ln N!.
\end{equation}

\subsection{经典过渡}

在 \ref{subsec:boltzmann-dist-classical}~小节中,我们给出了经典情形下的简并度:
\begin{equation}
  \omega_l = \frac{\incr{\omega_l}}{h_0^r}. \tag*{\eqref{eq:degeneracy-classical}}
\end{equation}
式中,$h_0$ 为相格尺度,而 $r$ 则为系统自由度。此时,单粒子配分函数可以写为
\begin{align}
  Z_1 &= \sum_l \omega_l\ee^{-\beta\varepsilon_l}
       = \sum_l \ee^{-\beta\varepsilon_l} \cdot \frac{\incr{\omega_l}}{h_0^r} \notag
  \intertext{相体积很小的时候可将求和化为积分:}
      &= \int \ee^{-\beta\varepsilon} \cdot \frac{\dd{\omega}}{h_0^r}
       = \frac{1}{h_0^r} \dotsint \ee^{-\beta\varepsilon(q,\,p)}
         \dd{q_1}\cdots\dd{q_r}\dd{p_1}\cdots\dd{p_r}.
\end{align}

在经典情形下,Boltzmann 分布过渡为
\begin{align}
  a_l &= \omega_l \, \ee^{-\alpha-\beta\varepsilon_l}
       = \ee^{-\alpha-\beta\varepsilon_l} \cdot \frac{\incr{\omega_l}}{h_0^r} \notag
  \intertext{代入 $N=\ee^{-\alpha}Z_1$,有}
      &= \frac{N}{Z_1} \cdot \ee^{-\beta\varepsilon_l} \cdot \frac{\incr{\omega_l}}{h_0^r}
       = N \cdot \frac{\ee^{-\beta\varepsilon_l}\incr{\omega_l}}%
                      {\sum_l\ee^{-\beta\varepsilon_l}\incr{\omega_l}}.
\end{align}
注意该表达式与 $h_0$ 无关。

\section{Maxwell 速度分布律}

    %\chapter{系综理论}
    %\include{chapters/7}
    %\chapter{相变的统计物理简介}
    %\include{chapters/8}

\end{document}

%========== 词汇 ==========%
% 绝热 adiabatic
% 等压 isobaric
% 等容 isochoric
% 等温 isothermal
% 等熵 isentropic
% 等焓 isenthalpic
% 多方 polytropic
% 准静 quasistatic
