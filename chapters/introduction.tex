\chapter{绪论}

本课程包含两部分内容:热力学、统计物理。

\kwd{热力学(thermodynamics)}是一种自上而下(top-down)的研究方式,它是形而上的、唯象的。用一句话可
以形容热力学:“知其然而不知其所以然”。它主要研究宏观物理量之间的关系。

对热力学有重大贡献的物理学家有 Carnot、Joule、Clausius、Kelvin 等。

\blankline

\kwd{统计物理(statistical Physics)}是一种自下而上(bottom-up)的研究方式,它是形而下的、微观的。

对统计物理有重大贡献的物理学家见表 \ref{tab:physicist-in-statistical-physics}。

\begin{table}[h]
  \centering
  \caption{对统计物理有重大贡献的物理学家}
  \label{tab:physicist-in-statistical-physics}
  \begin{tabular}{cc}
    \toprule
      \textbf{阶段} & \textbf{人物} \\
    \midrule
      经典统计 & Maxwell、Boltzmann、Gibbs、Einstein 等 \\
      量子概念 & Planck、Einstein、Fermi、Dirac、Pauli、Bose 等 \\
      量子统计 & von Neumann、Landau、Kramers、Pauli 等 \\
    \bottomrule
  \end{tabular}
\end{table}

统计物理可分为两个阶段。1860 年至 1902 年,人们主要研究近独立子体系(简单地说,就是理想气体);1902
年以后,出现系综理论,开始研究凝聚态系统。

统计物理的基础可以概括为\kwd{等概率原理}。
