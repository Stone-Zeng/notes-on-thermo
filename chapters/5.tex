\chapter{统计物理学基本概念}

\section{微观态的经典及量子描述}

\subsection{单粒子的经典描述}

微观态的经典描述以经典力学为基础,通常采用广义坐标与广义动量的形式。

对于一个有 $r$ 个自由度的系统,需要用 $2r$ 个变量来描述其运动状态,即 $r$ 个
广义坐标和 $r$ 个广义动量:
\begin{equation}
  (q_i, \, p_i) \quad \text{where} \quad i = 1, \, 2, \, \cdots, r.
\end{equation}
系统的 Hamilton 量为
\begin{equation}
  H = H (q_1, \, q_2, \, \cdots, q_r; \, p_1, \, p_2, \, \cdots, p_r; \, t),
\end{equation}
正则方程为
\begin{braced}
  \dot{q_i} &= \pdv{H}{p_i}, \\
  \dot{p_i} &= -\pdv{H}{q_i}.
\end{braced}

坐标和动量 $(q_1, \, q_2, \, \cdots, q_r; \, p_1, \, p_2, \, \cdots, p_r)$
张成了一个 $2r$ 维空间,称为 \kwd{μ 空间},每一组坐标和动量描述的点
称为\kwd{代表点}。

\begin{example}[自由粒子]
  对于一个 $r=3$ 的自由粒子,有
  \begin{braced}
    & p_x = m \dot{x}, \\
    & p_y = m \dot{y}, \\
    & p_z = m \dot{z}.
  \end{braced}
  其 μ 空间由 $(x, \, y, \, z; \, p_x, \, p_y, \, p_z)$ 张成。Hamilton 量为
  \begin{equation}
    H = \frac{1}{2m} \qty\big(p_x^2+p_y^2+p_z^2).
  \end{equation}
  %TODO:20160504 示意图
\end{example}

\begin{example}[一维谐振子]
  质量为 $m$ 的物体受力 $F=-Ax$ 的作用做简谐运动,其角频率等于
  \begin{equation}
    \omega = \sqrt{\frac{A}{m}},
  \end{equation}
  而 Hamilton 量为
  \begin{equation}
    H = \frac{p^2}{2m} + \frac{A}{2} x^2
      = \frac{p^2}{2m} + \frac{1}{2} m \omega^2 x^2.
  \end{equation}

  若总能量一定,即 $H=E$,则
  \begin{equation}
    \frac{p^2}{2mE} + \frac{x^2}{2E / m \omega^2} = 1,
  \end{equation}
  这在 μ 空间中表示一个椭圆(见图~\ref{fig:harmonic-oscillator-in-mu-space}),
  其面积为
  \begin{equation} \label{eq:ellipse-of-harmonic-oscillator-in-mu-space}
    S_\text{ellipse} = \pp ab
      = \pp \cdot \sqrt{\frac{2E}{m \omega^2}} \cdot \sqrt{2mE}
      = \frac{2 \pp E}{\omega}.
  \end{equation}
\end{example}

\begin{figure}[ht]
  \centering
  \FIGPLACEHOLDER
  %\begin{asy}
  %  pair O = (0, 0), x_axes = (8, 0), y_axes = (0, 5);
  %
  %  real a = 5.5, b = 3;
  %
  %  pair p1 = (0.4, 1.8), p2 = (5, 3.8);
  %  pair p3 = (2, -0.4), p4 = (6, -3);
  %
  %  draw(Label("$x$", EndPoint), (-x_axes)--x_axes, Arrow);
  %  draw(Label("$p$", EndPoint), (0, -4.2)--y_axes, Arrow);
  %
  %  draw(ellipse(O, a, b), linewidth(1) + color1);
  %
  %  draw(O--(a, 0), linewidth(1.2));
  %  draw(O--(0, b), linewidth(1.2));
  %
  %  draw(Label("$\sqrt{2mE}$", EndPoint), p1--p2);
  %  draw(Label("$\sqrt{\dfrac{\displaystyle 2E}{\displaystyle m\omega^2}}$", EndPoint), p3--p4);
  %
  %  label("$O$", O, SW);
  %\end{asy}
  \caption{μ 空间中的一维谐振子}
  \label{fig:harmonic-oscillator-in-mu-space}
\end{figure}

\begin{example}[转子]
  %\def\ST{\sin\theta} \def\SP{\sin\phi}
  %\def\CT{\cos\theta} \def\CP{\cos\phi}
  %\def\STsq{\sin^2\theta} \def\SPsq{\sin^2\phi}
  %\def\CTsq{\cos^2\theta} \def\CPsq{\cos^2\phi}
  %\def\DR{\dot{r}} \def\DT{\dot{\theta}} \def\DP{\dot{\phi}}
  如图~\ref{fig:rotator} 所示,质量为 $m$ 的物体被轻杆连接在 $O$ 点处,并可绕
  该点运动。其 Hamilton 量为
  \begin{equation} \label{eq:rotator-hamiltonian-a}
    H = \frac{m}{2} \qty\big(\dot{x}^2+\dot{y}^2+\dot{z}^2).
  \end{equation}
  取球坐标系,则有
  \begin{braced}
    x &= r \sin{\theta}\cos{\phi}, \\
    y &= r \sin{\theta}\sin{\phi}, \\
    z &= r \cos{\theta}.
  \end{braced}
  求导,得
  \begin{braced}
    \dot{x} &= \dot{r}\sin{\theta}\cos{\phi}
             + r\dot{\theta}\cos{\theta}\cos{\phi}
             - r\dot{\phi}\sin{\theta}\sin{\phi}, \\
    \dot{y} &= \dot{r}\sin{\theta}\sin{\phi}
             + r\dot{\theta}\cos{\theta}\sin{\phi}
             + r\dot{\phi}\sin{\theta}\cos{\phi}, \\
    \dot{z} &= \dot{r}\cos{\theta} - r\dot{\theta}\sin{\theta}.
  \end{braced}
  计算可知
  \begin{equation}
    \dot{x}^2 + \dot{y}^2 + \dot{z}^2
    = \dot{r}^2 + (r\dot{\theta})^2 + (r\dot{\phi})^2 \sin^2\theta.
  \end{equation}
  %\begin{align}
  %  \dot{x}^2 + \dot{y}^2 + \dot{z}^2 \\
  %  &= \qty\Big[\DR^2\STsq + (r\DT)^2\CTsq + (r\DP)^2\STsq] \\
  %  %\myTag{$\dot{x}^2$、$\dot{y}^2$ 平方项} \\
  %  &\mathrel{\phantom{=}} + \qty\Big[2r\DR\DT\ST\CT\CPsq - 2r\DR\DP\STsq\SP\CP - 2r^2\DT\DP\ST\CT\SP\CP] \\%\myTag{$\dot{x}^2$ 交叉项} \\
  %  &\mathrel{\phantom{=}} + \qty\Big[2r\DR\DT\ST\CT\SPsq + 2r\DR\DP\STsq\SP\CP + 2r^2\DT\DP\ST\CT\SP\CP] \\%\myTag{$\dot{y}^2$ 交叉项} \\
  %  &\mathrel{\phantom{=}} + \qty\Big[ \DR^2\CTsq - 2r\DR\DT\ST\CT + \qty\big(r\DT)^2\STsq] \myTag{$\dot{z}^2$} \\
  %  &= \DR^2 + (r\DT)^2 + (r\DP)^2\STsq.
  %\end{align}
  代入 \eqref{eq:rotator-hamiltonian-a}~式,得
  \begin{align}
    H &= \frac{m}{2} \qty\big(\dot{r}^2 + r^2 \dot{\theta}^2
                              + r^2 \dot{\phi}^2 \sin^2\theta).
  \end{align}
  由于物体已被轻杆连接在了 $O$ 点,因而 $r$ 不变、$\dot{r}=0$。

  引入\kwd{共轭动量}
  \begin{braced}
    p_\theta &= m r^2 \dot{\theta}, \\
    p_\phi   &= m r^2 \dot{\phi} \sin^2\theta,
  \end{braced}
  则系统的 Hamilton 量可写为
  \begin{equation}
    H = \frac{1}{2I} \qty(p_\theta^2 + \frac{1}{\sin^2\theta} p_\phi^2),
  \end{equation}
  其中的 $I = mr^2$ 是物体关于 $O$ 点的\kwd{转动惯量}。

  在本例中,μ 空间由广义坐标和广义动量
  $(\theta, \, \phi; \, p_\theta, \, p_\phi)$ 张成,它是四维的。
\end{example}

\begin{figure}[ht]
  \centering
  \FIGPLACEHOLDER
  %\begin{asy}
  %  import my3D;
  %
  %  triple O = (0, 0, 0), x_axes = (3.5, 0, 0), y_axes = (0, 5, 0), z_axes = (0, 0, 5);
  %  triple point_m = (2.5, 5, 6), point_n = (point_m.x, point_m.y, 0);
  %
  %  pair O2 = project(O), point_m2 = project(point_m), point_n2 = project(point_n);
  %
  %  draw(Label("$x$", EndPoint), O2--project(x_axes), Arrow);
  %  draw(Label("$y$", EndPoint), O2--project(y_axes), Arrow);
  %  draw(Label("$z$", EndPoint), O2--project(z_axes), Arrow);
  %
  %  draw(Label("$\bm{r}$", MidPoint, black), O2--point_m2, linewidth(1.5) + color2);
  %  draw(O2--point_n2--point_m2, dashed + color1);
  %  fill(circle(point_m2, 0.2), color1);
  %
  %  real angle_r = 0.7;
  %  draw(Label("$\varphi$", MidPoint, Relative(E)), angleMark(O, x_axes, point_n, angle_r));
  %  draw(Label("$\theta$", MidPoint, Relative((-1,-0.4))), angleMark(O, z_axes, point_m, angle_r));
  %
  %  label("$O$", O2, (-1.5, 0.5));
  %  label("$m$", point_m2, (2, 0.5));
  %\end{asy}
  \caption{转子的示意图}
  \label{fig:rotator}
\end{figure}

\subsection{单粒子的量子描述}

微观态的量子描述以量子力学为基础。粒子的动量 $\V{p}$、能量 $E$ 满足
\kwd{de Broglie关系}:
\begin{braced}
  \V{p} &= \hbar\V{k}, \\
  E     &= \hbar\omega,
\end{braced}
其中的 $\hbar$ 称为\kwd{(约化)Planck 常数},其值为
\begin{equation}
  \hbar = \frac{h}{2\pp} = \SI{1.0545718e-34}{\joule\second}.
\end{equation}
式中的 $h$ 也称 Planck 常数。

De Broglie 关系说明微观粒子具有\kwd{波粒二象性}。这就引出了另一个重要结果——%
\kwd{不确定关系}:
\begin{equation}
  \incr{p} \, \incr{q} \gtrsim h.
  \footnote{更精确的表述为 $\incr{p} \, \incr{q} \geqslant \hbar / 2 $。}
\end{equation}
该关系式说明粒子的动量和坐标不可能被同时精确测量,因而其运动也就无法用经典的
轨道概念来描述,必须改用\kwd{波函数}。

粒子波函数 $\Psi$ 满足的方程即\kwd{Schrödinger方程}:
\begin{equation}
  \ii \pdv{t} \Psi = \hat{H} \Psi,
\end{equation}
式中的 $\hat{H}$ 是 Hamilton 算符。在定态情况(即将时间变量分离后),
Schrödinger 方程化为
\begin{equation}
  \hat{H} \psi = E \psi.
\end{equation}

\begin{example}[箱中的自由粒子]
  设粒子在边长为 $L$ 的立方体容器内运动,则其量子态(即波函数)有平面波的形式:
  \begin{equation}
    \Psi_{n_1, \, n_2, \, n_3} (\V{r})
    \sim \ee^{\ii\V{p}\cdot\V{r} / \hbar} \qc
    n_i = \pm 1, \, \pm 2, \, \pm 3, \, \cdots
  \end{equation}
  求解 Schrödinger 方程,可以发现动量与能量的本征值都是量子化的:
  \begin{braced}
    \V{p} &= p_x\V{i} + p_y\V{j} + p_z\V{k}
           = \frac{2\pp\hbar}{L} n_1\V{i} + n_2\V{j} + n_3\V{k}, \\
        E &= \frac{1}{2m} \qty\big(p_x^2 + p_y^2 + p_z^2)
           = \frac{2\pp^2 \hbar^2}{mL^2} \qty\big(n_1^2 + n_2^2 + n_3^2).
  \end{braced}
  量子化的能量也成为\kwd{能级}。对于能级
  \begin{equation}
    E = \frac{2\pp^2\hbar^2}{mL^2},
  \end{equation}
  它对应6种量子态:
  \begin{equation}
    (\pm 1, \, 0, \, 0) \qc (0, \, \pm 1, \, 0) \qc (0, \, 0, \, \pm 1).
  \end{equation}
  这种现象称为能级\kwd{简并}。同一能级对应量子态的数目称为\kwd{简并度}。
  显然,这里的简并度为 6。而能量更高的一个能级
  \begin{equation}
    E = \frac{2\pp^2\hbar^2}{mL^2} \times 3
      = \frac{2\pp^2\hbar^2}{mL^2} \times (1+1+1)
  \end{equation}
  则对应 $2^3=8$ 个量子态,它的简并度为 8。
\end{example}

\begin{example}[一维谐振子]
  频率为 $\nu$ 的谐振子,其能量为
  \begin{equation}
    E_n = \qty(n+\frac{1}{2}) \, h\nu \qc n = 0, \, 1, \, 2, \, \cdots
  \end{equation}
  可见该系统的简并度 $g=1$。

  根据式~\eqref{eq:ellipse-of-harmonic-oscillator-in-mu-space},μ 空间中的椭圆
  面积为
  \begin{equation}
    S_n = \frac{2\pp E_n}{\omega} =\frac{E_n}{\nu} = \qty(n+\frac{1}{2}) \, h.
  \end{equation}
  因此两个相轨道之间的面积为 $h$,它对应一个量子态。
\end{example}

%     \begin{example}[转子]
%       blablabla
%     \end{example}%TODO:20160624 未完成
    
%   \subsection{多粒子体系的描述}
%     对于 $N$ 个粒子组成的体系,设每个粒子的自由度为 $r$,则每个粒子可用 $2r$ 个变量 $(q_1, \, q_2, \, \cdots, q_r; \allowbreak \, p_1, \, p_2, \, \cdots, p_r)$ 来描述。此时,系统的总自由度 %FIXME:20160624 公式换行
%     \begin{equation}
%       f = Nr,
%     \end{equation}
%     因而系统的运动需要用 $2f$ 个变量来刻画。它们张成了一个 $2f$ 维空间,称为\kwd{\itshape{Γ}空间},也叫\kwd{相空间}。
% %\section{}
% %\section{}
% %\section{}

% \raggedbottom%FIXME:20260325 交叉引用、脚注每页重新计数失效,必须加上该行
% \pagebreak
