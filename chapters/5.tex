%% Copyright (C) 2016--2018 by Xiangdong Zeng <pssysrq@163.com>
%%
%% -* Notes on Thermodynamics and Statistical Physics *-
%%
%% This file may be distributed and/or modified under the
%% Creative Commons Attribution Share Alike 4.0 license.

\chapter{统计物理学基本概念}

\section{微观态的经典及量子描述}

\subsection{单粒子的经典描述}

微观态的经典描述以经典力学为基础,通常采用广义坐标与广义动量的形式。

对于一个有 $r$ 个自由度的系统,需要用 $2r$ 个变量来描述其运动状态,即 $r$ 个广义坐标和 $r$ 个广义动
量:
\begin{equation}
  (q_i, \, p_i) \qq{where} i = 1, \, 2, \, \cdots, r.
\end{equation}
系统的 Hamilton 量为
\begin{equation}
  H = H (q_1, \, q_2, \, \cdots, q_r; \, p_1, \, p_2, \, \cdots, p_r; \, t),
\end{equation}
正则方程为
\begin{braced}
  \dot{q_i} &= \pdv{H}{p_i}, \\
  \dot{p_i} &= -\pdv{H}{q_i}.
\end{braced}

坐标和动量 $(q_1, \, q_2, \, \cdots, q_r; \, p_1, \, p_2, \, \cdots, p_r)$ 张成一个 $2r$ 维空间,称
为 \kwd{μ 空间}(或\kwd{子相空间}),每一组坐标和动量描述的点称为\kwd{代表点}。

\begin{example}[自由粒子]
  对于一个 $r=3$ 的自由粒子,有
  \begin{braced}
    & p_x = m \dot{x}, \\
    & p_y = m \dot{y}, \\
    & p_z = m \dot{z}.
  \end{braced}
  其 μ 空间由 $(x, \, y, \, z; \, p_x, \, p_y, \, p_z)$ 张成。Hamilton 量为
  \begin{equation}
    H = \frac{1}{2m} \qty\big(p_x^2+p_y^2+p_z^2).
  \end{equation}
  %TODO:20160504 示意图
\end{example}

\begin{example}[一维谐振子]
  质量为 $m$ 的物体受力 $F=-Ax$ 的作用做简谐运动,其角频率等于
  \begin{equation}
    \omega = \sqrt{\frac{A}{m}},
  \end{equation}
  而 Hamilton 量为
  \begin{equation}
    H = \frac{p^2}{2m} + \frac{A}{2} x^2
      = \frac{p^2}{2m} + \frac{1}{2} m \omega^2 x^2.
  \end{equation}

  若总能量一定,即 $H=E$,则
  \begin{equation}
    \frac{p^2}{2mE} + \frac{x^2}{2E / m \omega^2} = 1,
  \end{equation}
  这在 μ 空间中表示一个椭圆(见图~\ref{fig:harmonic-oscillator-mu-space}),其面积为
  \begin{equation} \label{eq:ellipse-of-harmonic-oscillator-mu-space}
    S_\text{ellipse} = \pp ab
      = \pp \cdot \sqrt{\frac{2E}{m \omega^2}} \cdot \sqrt{2mE}
      = \frac{2 \pp E}{\omega}.
  \end{equation}
\end{example}

\begin{figure}[ht]
  \centering
  \FIGPLACEHOLDER
  %\begin{asy}
  %  pair O = (0, 0), x_axes = (8, 0), y_axes = (0, 5);
  %
  %  real a = 5.5, b = 3;
  %
  %  pair p1 = (0.4, 1.8), p2 = (5, 3.8);
  %  pair p3 = (2, -0.4), p4 = (6, -3);
  %
  %  draw(Label("$x$", EndPoint), (-x_axes)--x_axes, Arrow);
  %  draw(Label("$p$", EndPoint), (0, -4.2)--y_axes, Arrow);
  %
  %  draw(ellipse(O, a, b), linewidth(1) + color1);
  %
  %  draw(O--(a, 0), linewidth(1.2));
  %  draw(O--(0, b), linewidth(1.2));
  %
  %  draw(Label("$\sqrt{2mE}$", EndPoint), p1--p2);
  %  draw(Label("$\sqrt{\dfrac{\displaystyle 2E}{\displaystyle m\omega^2}}$", EndPoint), p3--p4);
  %
  %  label("$O$", O, SW);
  %\end{asy}
  \caption{μ 空间中的一维谐振子}
  \label{fig:harmonic-oscillator-mu-space}
\end{figure}

\begin{example}[转子]
  %\def\ST{\sin\theta} \def\SP{\sin\phi}
  %\def\CT{\cos\theta} \def\CP{\cos\phi}
  %\def\STsq{\sin^2\theta} \def\SPsq{\sin^2\phi}
  %\def\CTsq{\cos^2\theta} \def\CPsq{\cos^2\phi}
  %\def\DR{\dot{r}} \def\DT{\dot{\theta}} \def\DP{\dot{\phi}}
  如图~\ref{fig:rotator} 所示,质量为 $m$ 的物体被轻杆连接在 $O$ 点处,并可绕该点运动。其 Hamilton
  量为
  \begin{equation} \label{eq:rotator-hamiltonian-a}
    H = \frac{m}{2} \qty\big(\dot{x}^2+\dot{y}^2+\dot{z}^2).
  \end{equation}
  取球坐标系,则有
  \begin{braced}
    x &= r \sin{\theta}\cos{\phi}, \\
    y &= r \sin{\theta}\sin{\phi}, \\
    z &= r \cos{\theta}.
  \end{braced}
  求导,得
  \begin{braced}
    \dot{x} &= \dot{r}\sin{\theta}\cos{\phi}
             + r\dot{\theta}\cos{\theta}\cos{\phi}
             - r\dot{\phi}\sin{\theta}\sin{\phi}, \\
    \dot{y} &= \dot{r}\sin{\theta}\sin{\phi}
             + r\dot{\theta}\cos{\theta}\sin{\phi}
             + r\dot{\phi}\sin{\theta}\cos{\phi}, \\
    \dot{z} &= \dot{r}\cos{\theta} - r\dot{\theta}\sin{\theta}.
  \end{braced}
  计算可知
  \begin{equation}
    \dot{x}^2 + \dot{y}^2 + \dot{z}^2
    = \dot{r}^2 + (r\dot{\theta})^2 + (r\dot{\phi})^2 \sin^2\theta.
  \end{equation}
  %\begin{align}
  %  \dot{x}^2 + \dot{y}^2 + \dot{z}^2 \\
  %  &= \qty\Big[\DR^2\STsq + (r\DT)^2\CTsq + (r\DP)^2\STsq] \\
  %  %\myTag{$\dot{x}^2$、$\dot{y}^2$ 平方项} \\
  %  &\mathrel{\phantom{=}} + \qty\Big[2r\DR\DT\ST\CT\CPsq - 2r\DR\DP\STsq\SP\CP - 2r^2\DT\DP\ST\CT\SP\CP] \\%\myTag{$\dot{x}^2$ 交叉项} \\
  %  &\mathrel{\phantom{=}} + \qty\Big[2r\DR\DT\ST\CT\SPsq + 2r\DR\DP\STsq\SP\CP + 2r^2\DT\DP\ST\CT\SP\CP] \\%\myTag{$\dot{y}^2$ 交叉项} \\
  %  &\mathrel{\phantom{=}} + \qty\Big[ \DR^2\CTsq - 2r\DR\DT\ST\CT + \qty\big(r\DT)^2\STsq] \myTag{$\dot{z}^2$} \\
  %  &= \DR^2 + (r\DT)^2 + (r\DP)^2\STsq.
  %\end{align}
  代入 \eqref{eq:rotator-hamiltonian-a}~式,得
  \begin{align}
    H &= \frac{m}{2} \qty\big(\dot{r}^2 + r^2 \dot{\theta}^2
                              + r^2 \dot{\phi}^2 \sin^2\theta).
  \end{align}
  由于物体已被轻杆连接在了 $O$ 点,因而 $r$ 不变、$\dot{r}=0$。

  引入\kwd{共轭动量}
  \begin{braced}
    p_\theta &= m r^2 \dot{\theta}, \\
    p_\phi   &= m r^2 \dot{\phi} \sin^2\theta,
  \end{braced}
  则系统的 Hamilton 量可写为
  \begin{equation}
    H = \frac{1}{2I} \qty(p_\theta^2 + \frac{1}{\sin^2\theta} p_\phi^2),
  \end{equation}
  其中的 $I = mr^2$ 是物体关于 $O$ 点的\kwd{转动惯量}。

  在本例中,μ 空间由广义坐标和广义动量
  $(\theta, \, \phi; \, p_\theta, \, p_\phi)$ 张成,它是四维的。
\end{example}

\begin{figure}[ht]
  \centering
  \FIGPLACEHOLDER
  %\begin{asy}
  %  import my3D;
  %
  %  triple O = (0, 0, 0), x_axes = (3.5, 0, 0), y_axes = (0, 5, 0), z_axes = (0, 0, 5);
  %  triple point_m = (2.5, 5, 6), point_n = (point_m.x, point_m.y, 0);
  %
  %  pair O2 = project(O), point_m2 = project(point_m), point_n2 = project(point_n);
  %
  %  draw(Label("$x$", EndPoint), O2--project(x_axes), Arrow);
  %  draw(Label("$y$", EndPoint), O2--project(y_axes), Arrow);
  %  draw(Label("$z$", EndPoint), O2--project(z_axes), Arrow);
  %
  %  draw(Label("$\bm{r}$", MidPoint, black), O2--point_m2, linewidth(1.5) + color2);
  %  draw(O2--point_n2--point_m2, dashed + color1);
  %  fill(circle(point_m2, 0.2), color1);
  %
  %  real angle_r = 0.7;
  %  draw(Label("$\varphi$", MidPoint, Relative(E)), angleMark(O, x_axes, point_n, angle_r));
  %  draw(Label("$\theta$", MidPoint, Relative((-1,-0.4))), angleMark(O, z_axes, point_m, angle_r));
  %
  %  label("$O$", O2, (-1.5, 0.5));
  %  label("$m$", point_m2, (2, 0.5));
  %\end{asy}
  \caption{转子的示意图}
  \label{fig:rotator}
\end{figure}

\subsection{单粒子的量子描述}

微观态的量子描述以量子力学为基础。粒子的动量 $\V{p}$、能量 $E$ 满足 \kwd{de Broglie关系}:
\begin{braced}
  \V{p} &= \hbar\V{k}, \\
  E     &= \hbar\omega,
\end{braced}
其中的 $\hbar$ 称为\kwd{(约化)Planck 常数},其值为
\begin{equation}
  \hbar = \frac{h}{2\pp} = \SI{1.0545718e-34}{\joule\second}.
\end{equation}
式中的 $h$ 也称 Planck 常数。

De Broglie 关系说明微观粒子具有\kwd{波粒二象性}。这就引出了另一个重要结果——\kwd{不确定关系}:
\begin{equation}
  \incr{p} \incr{q} \gtrsim h.
  \footnote{更精确的表述为 $\incr{p} \incr{q} \geqslant \hbar / 2 $。}
\end{equation}
可见,在 $\incr{p}\to 0$ 时,必有 $\incr{q}\to\infty$。这说明粒子的动量和坐标不可能被同时精确测量,
因而其运动也就无法用经典的轨道概念来描述,必须改用\kwd{波函数}。

粒子波函数 $\Psi$ 满足的方程即\kwd{Schrödinger 方程}:
\begin{equation}
  \ii \pdv{t} \Psi = \hat{H} \Psi,
\end{equation}
式中的 $\hat{H}$ 是 Hamilton 算符。在定态情况(即将时间变量分离后),Schrödinger 方程化为
\begin{equation}
  \hat{H} \psi = E \psi.
\end{equation}

\begin{example}[箱中的自由粒子]
  设粒子在边长为 $L$ 的立方体容器内运动,则其量子态(即波函数)有平面波的形式:
  \begin{equation}
    \Psi_{n_1, \, n_2, \, n_3} (\V{r})
    \sim \ee^{\ii\V{p}\cdot\V{r} / \hbar} \qc
    n_i = \pm 1, \, \pm 2, \, \pm 3, \, \cdots
  \end{equation}
  求解 Schrödinger 方程,可以发现动量与能量的本征值都是量子化的:
  \begin{braced}
    \V{p} &= p_x\V{i} + p_y\V{j} + p_z\V{k}
           = \frac{2\pp\hbar}{L} n_1\V{i} + n_2\V{j} + n_3\V{k}, \\
        E &= \frac{1}{2m} \qty\big(p_x^2 + p_y^2 + p_z^2)
           = \frac{2\pp^2 \hbar^2}{mL^2} \qty\big(n_1^2 + n_2^2 + n_3^2).
  \end{braced}
  量子化的能量也成为\kwd{能级}。对于能级
  \begin{equation}
    E = \frac{2\pp^2\hbar^2}{mL^2},
  \end{equation}
  它对应6种量子态:
  \begin{equation}
    (\pm 1, \, 0, \, 0) \qc (0, \, \pm 1, \, 0) \qc (0, \, 0, \, \pm 1).
  \end{equation}
  这种现象称为能级\kwd{简并}。同一能级对应量子态的数目称为\kwd{简并度}。显然,这里的简并度为 6。而能
  量更高的一个能级
  \begin{equation}
    E = \frac{2\pp^2\hbar^2}{mL^2} \times 3
      = \frac{2\pp^2\hbar^2}{mL^2} \times (1+1+1)
  \end{equation}
  则对应 $2^3=8$ 个量子态,它的简并度为 8。
\end{example}

\begin{example}[一维谐振子]
  频率为 $\nu$ 的谐振子,其能量为
  \begin{equation}
    E_n = \qty(n+\frac{1}{2}) \, h\nu \qc n = 0, \, 1, \, 2, \, \cdots
  \end{equation}
  可见该系统的简并度 $g=1$。

  根据式~\eqref{eq:ellipse-of-harmonic-oscillator-mu-space},μ 空间中的椭圆面积为
  \begin{equation}
    S_n = \frac{2\pp E_n}{\omega} =\frac{E_n}{\nu} = \qty(n+\frac{1}{2}) \, h.
  \end{equation}
  因此两个相轨道之间的面积为 $h$,它对应一个量子态。
\end{example}

\begin{example}[转子]
  %TODO:20160624 未完成
\end{example}

\subsection{多粒子系统}

对于 $N$ 个粒子组成的系统,设每个粒子的自由度为 $r$,则每个粒子可用 $2r$ 个变量
$(q_1, \, q_2, \, \cdots, q_r; \, \allowbreak p_1, \, p_2, \, \cdots, p_r)$ 来描述。此时,系统的总自
由度
\begin{equation}
  f = Nr,
\end{equation}
因而系统的运动需要用 $2f$ 个变量 $(q_1, \, q_2, \, \cdots, q_f; \, p_1, \, p_2, \, \cdots, p_f)$ 来
刻画。这些变量张成了一个 $2f$ 维空间,称为\kwd{Γ 空间},也叫\kwd{相空间}。Γ 空间中的一个点就表示系统
的一个微观状态,状态运动的微小范围可用体积元表示:
\begin{equation}
  \incr{\Omega} = \incr{q_1} \cdots \incr{q_f} \incr{p_1} \cdots \incr{p_f}.
\end{equation}
根据不确定关系,取 $\incr{p} \incr{q} \simeq h$,则有
\begin{equation}
  \incr{\Omega} = \incr{q_1} \incr{p_1} \cdots \incr{q_f} \incr{p_f} \simeq h^f.
\end{equation}
这是多粒子系统的相格大小。

统计物理研究的系统往往由大量\kwd{全同粒子}组成。全同粒子是指内禀性质,如质量、电荷、自旋等均完全相同
的粒子。根据量子力学,全同粒子具有不可分辨性。换句话说,全同粒子的交换不引起新的量子态。

设系统的波函数为 $\Psi$。引入\kwd{交换算符} $\hat{P}$,根据全同粒子的不可分辨性,可有
\begin{equation}
  \abs{\hat{P}\Psi}^2 = \abs{\Psi}^2 \implies \hat{P}\Psi = \pm\Psi.
\end{equation}

波函数交换对称,即 $\hat{P}\Psi=+\Psi$ 的粒子,称为\kwd{Bose 子}。它们遵循%
\kwd{Bose--Einstein 统计},并且自旋为整数。光子、π 介子、胶子以及 Higgs 粒子等都是 Bose 子。单一量子
态上可占据任意数目的 Bose 子。

波函数交换反对称,即 $\hat{P}\Psi=-\Psi$ 的粒子,称为\kwd{Fermi 子}。它们遵循%
\kwd{Fermi--Dirac 统计},自旋为半整数。电子、质子、中子以及夸克等都是 Fermi 子。全同 Fermi 子组成的
系统满足\kwd{Pauli 不相容原理},即单一量子态上只可占据 0 个或 1 个 Fermi 子。

就基本粒子而言,Bose 子传递相互作用,而 Fermi 子组成物质。对于复合粒子,如果含有偶数个 Fermi 子,则
为 Bose 子;如果含有奇数个 Fermi 子,则为 Fermi 子。例如氦的同位素 \ce{^4He},它包含两个质子、两个中
子和两个电子,为 Bose 子;而 \ce{^3He} 则包含两个质子、一个中子和两个电子,为 Fermi 子。

对于多粒子系统,如果各粒子的波函数分别局限在空间不同范围内,彼此交叠很少,则称为\kwd{定域子系}。此
时,交换两个粒子的波函数(对应于量子态),可以看出系统的微观状态发生了变化(见图~%
\ref{fig:localized-sub-system}),因而即使是全同粒子,在定域子系中也可分辨。定域子系遵循%
\kwd{Boltzmann 统计}。

\begin{figure}[ht]
  \centering
  \FIGPLACEHOLDER
  \caption{定域子系}
  \label{fig:localized-sub-system}
\end{figure}

下面我们举例说明单粒子量子态与多体量子态之间的关系。设系统由两个粒子组成,每个粒子可以处在四个量子态
下。不同情况下,粒子的所有分布方式列于表~\ref{tab:two-particles-distribution}。由此可知,对于定域子
系、非定域 Bose 子和非定域 Fermi 子,系统分别有 16 个、10 个和 6 个量子态。

\begin{table}[ht]
  \def\B{{\Large\symbol{"25CB}}}
  \let\TC=\textcircled
  \newcommand\STATE[4]{%
    \CJKunderline{\makebox[2em][c]{#1}}\kern1em%
    \CJKunderline{\makebox[2em][c]{#2}}\kern1em%
    \CJKunderline{\makebox[2em][c]{#3}}\kern1em%
    \CJKunderline{\makebox[2em][c]{#4}}}
  \newcommand\TITLE[1]{\makebox[11em][c]{#1}}
  \centering
  \caption{两个粒子在四个量子态中的分布情况}
  \label{tab:two-particles-distribution}
  \begin{tabular}{c|c|c}
    \toprule
      \multicolumn{3}{c}{%
        \begin{tabular}{@{}ccc@{}}
          \TITLE{定域子系} & \TITLE{非定域 Bose 子} & \TITLE{非定域 Fermi 子}
        \end{tabular}} \\
    \midrule
      \STATE{\TC1\TC2}{}{}{} & \STATE{\B\B}{}{}{} & \\
      \STATE{}{\TC1\TC2}{}{} & \STATE{}{\B\B}{}{} & \\
      \STATE{}{}{\TC1\TC2}{} & \STATE{}{}{\B\B}{} & \\
      \STATE{}{}{}{\TC1\TC2} & \STATE{}{}{}{\B\B} & \\
      \STATE{\TC1}{\TC2}{}{} & \STATE{\B}{\B}{}{} & \STATE{\B}{\B}{}{} \\
      \STATE{\TC2}{\TC1}{}{} &                    &                    \\
      \STATE{\TC1}{\TC2}{}{} & \STATE{\B}{}{\B}{} & \STATE{\B}{}{\B}{} \\
      \STATE{\TC2}{}{\TC1}{} &                    &                    \\
      \STATE{\TC1}{}{\TC2}{} & \STATE{\B}{}{}{\B} & \STATE{\B}{}{}{\B} \\
      \STATE{\TC2}{}{}{\TC1} &                    &                    \\
      \STATE{}{\TC1}{\TC2}{} & \STATE{}{\B}{\B}{} & \STATE{}{\B}{\B}{} \\
      \STATE{}{\TC2}{\TC1}{} &                    &                    \\
      \STATE{}{\TC1}{}{\TC2} & \STATE{}{\B}{}{\B} & \STATE{}{\B}{}{\B} \\
      \STATE{}{\TC2}{}{\TC1} &                    &                    \\
      \STATE{}{}{\TC1}{\TC2} & \STATE{}{}{\B}{\B} & \STATE{}{}{\B}{\B} \\
      \STATE{}{}{\TC2}{\TC1} &                    &                    \\
    \bottomrule
  \end{tabular}
\end{table}

至于一般情况,我们将在 \ref{sec:distribution-and-microstate}~节中讨论。

\section{宏观量;等概率原理}

\subsection{宏观量的统计性质}

统计物理学的基本观点是:宏观量是微观量的统计平均。宏观观测有如下特点:

\begin{itemize}
  \item 空间尺度上,宏观小、微观大;
  \item 时间尺度上,宏观短、微观长。
\end{itemize}

由此我们可以知道,任意宏观态都对应着非常多的微观态。

\subsection{统计规律性}

就微观层面而言,粒子的运动遵循\kwd{力学规律}。无论是经典力学中的 Newton 方程,还是量子力学中的
Schrödinger 方程,都具有\kwd{时间反演}对称性,因而是可逆的。但在宏观层面,考虑到热力学第二定律,热现
象过程具有不可逆性,即时间有确定方向。这是\kwd{统计规律}的体现。

力学规律是确定性的,一旦运动方程确定,系统在任意时刻的状态就可以确定;而统计规律则具有不确定性,在一
定宏观条件下,系统总是以一定概率处于某一微观状态。

造成统计规律的原因主要有下面两点:

\begin{itemize}
  \item 宏观态对应着数量极其巨大的微观态,这些微观态不可能由宏观状态唯一决定;
  \item 系统与环境之间总是不可避免地存在相互作用,而且这些相互作用又有一定的随机性。
\end{itemize}

\subsection{等概率原理}

\kwd{等概率原理}(equal a priori probability postulate)最早由 Boltzmann 提出,它可以表述为:对于平
衡态下的孤立系,各微观态出现的概率相同。这是统计物理学的基本假设。

\section{分布和微观状态} \label{sec:distribution-and-microstate}

在本节以及第\ref{ch:}章中,我们均要求粒子之间相互作用可以忽略。因而总能量等于各粒子的能量之和。这样
的系统称为\kwd{近独立粒子系统}。

设粒子的能级为 $\varepsilon_1, \, \varepsilon_2, \, \cdots, \, \varepsilon_l, \, \cdots$,各能级的简
并度为 $\omega_1, \, \omega_2, \, \cdots, \, \omega_l, \, \cdots$。我们把每个能级上占据的粒子数
$a_1, \, a_2, \, \cdots, \, a_l, \, \cdots$ 称为一个\kwd{(微观)分布},简记为 $\qty{a_l}$。对于平衡
态下的孤立系,能量 $E$、体积 $V$、粒子数 $N$ 都是给定的。因此分布 $\qty{a_l}$ 需要满足
\begin{braced}
  & \sum_l a_l = N, \\
  & \sum_l \varepsilon_l a_l = E.
\end{braced}

注意分布与量子态是不同的概念。

\subsection{Boltzmann 体系}

Boltzmann 体系由定域子系组成。在能级 $\varepsilon_l$ 上,有 $\omega_l$ 个量子态(简并度),并占据着
$a_l$ 个粒子。粒子所处的量子态互不影响,且粒子可以编号。根据乘法原理,所有分布情况数等于每个粒子的可
能量子态数目之积,即 ${\omega_l}^{a_l}$。对于整个体系而言,需将所有能级上的量子态数目乘起来,即
$\prod_l {\omega_l}^{a_l}$。

分布 $\qty{a_l}$ 仅仅指定了粒子的数目,而没有指定具体是哪些粒子,所以还要考虑不同能级之间粒子数的交
换。交换 $N$ 个粒子的所有可能方式(全排列)共有 $\Pnum{N}{N}=N!$ 种。而单个能级上粒子的交换并不会造
成任何改变
\footnote{在能级 $\varepsilon_l$ 上,粒子交换带来微观状态的变化,已经计算在了
  ${\omega_l}^{a_l}$ 中。},
因此需要除去所有能级上粒子的交换数 $\prod_l a_l!$。

综上,分布 $\qty{a_l}$ 对应的总量子态数为
\begin{equation}
  \Omega_\text{MB}(\qty{a_l}) = \frac{N!}{\prod_l a_l!} \prod_l {\omega_l}^{a_l}.
\end{equation}
下标“$\text{MB}$”表示粒子服从 Maxwell--Boltzmann 统计。

\subsection{Bose--Einstein 体系}

Bose--Einstein 体系由 Bose 子组成。相比定域子系,Bose 子具有不可分辨性,而且在每个量子态上可占据任意
数量。因而 $a_l$ 个 Bose 子在 $\omega_l$ 个量子态上的分布,就相当于 $a_l$ 个相同的小球放在
$\omega_l$ 个不同的盒子中。考虑使用“插空法”,即先将 $a_l$ 个小球和 $\omega_l-1$ 个挡板排列(此时已经
将球分成了 $\omega_l$ 份),再从中选出 $a_l$ 个位置放小球。可见,共有
\begin{equation}
  \Cnum{a_l+\omega_l-1}{a_l} = \frac{\qty(a_l+\omega_l-1)!}{a! \, \qty(\omega_l-1)!}
\end{equation}
种情况。

由于不可分辨性,这里不需要考虑不同能级上粒子的交换。所以总量子态数为
\begin{equation}
  \Omega_\text{BE}(\qty{a_l}) = \prod_l \frac{\qty(a_l+\omega_l-1)!}{a! \, \qty(\omega_l-1)!}.
\end{equation}
下标“$\text{BE}$”表示粒子服从 Bose--Einstein 统计。

\subsection{Fermi--Dirac 体系}

Fermi--Dirac 体系由 Fermi 子组成。根据 Pauli 不相容原理,每个量子态上最多只能占据一个 Fermi 子,这相
当于从 $\omega_l$ 个量子态中选出 $a_l$ 个态来让粒子分布。因此共有
\begin{equation}
  \Cnum{\omega_l}{a_l} = \frac{\omega_l!}{a_l! \, \qty(\omega_l-a_l)!}
\end{equation}
种情况。显然,此时要求 $k \leqslant n$。

与 Bose--Einstein 体系相同,这里同样不需要考虑粒子的交换。总量子态数为
\begin{equation}
  \Omega_\text{FD}(\qty{a_l}) = \prod_l \frac{\omega_l!}{a_l! \, \qty(\omega_l-a_l)!}.
\end{equation}
下标“$\text{FD}$”表示粒子服从 Fermi--Dirac 统计。

\subsection{非简并条件}

当每个量子态上占据的粒子非常“稀薄”,即
\begin{equation} \label{eq:non-degenerate-condition}
  \frac{a_l}{\omega_l} \ll 1,
\end{equation}
且 $\omega_l$ 很大时,我们有
\begin{braced} \label{eq:bose-and-fermi-dist-under-non-degenerate-condition}
  \Omega_\text{BE}(\qty{a_l}) &= \prod_l \frac{\qty(a_l+\omega_l-1)!}{a! \, \qty(\omega_l-1)!}
  = \prod_l \frac{\qty(\omega_l+a_l-1) \qty(\omega+a_l-2) \cdots \omega_l}{a_l!}
  \approx \prod_l \frac{{\omega_l}^{a_l}}{a_l!}
  = \frac{\Omega_\text{MB}(\qty{a_l})}{N!}, \\
  \Omega_\text{FD}(\qty{a_l}) &= \prod_l \frac{\omega_l!}{a_l! \, \qty(\omega_l-a_l)!}
  = \prod_l \frac{\omega_l \qty(\omega_l-1) \cdots \qty(\omega-a_l+1)}{a_l!}
  \approx \prod_l \frac{{\omega_l}^{a_l}}{a_l!}
  = \frac{\Omega_\text{MB}(\qty{a_l})}{N!}.
\end{braced}
即 Bose--Einstein 分布与 Fermi--Dirac 分布均趋同于 Maxwell--Boltzmann 分布。式~%
\eqref{eq:non-degenerate-condition} 称为\kwd{非简并条件}。

注意式~\eqref{eq:bose-and-fermi-dist-under-non-degenerate-condition} 最后的结果中有一个 $1/N!$,这反
映了粒子的全同性要求。
