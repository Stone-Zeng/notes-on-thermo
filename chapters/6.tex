%% Copyright (C) 2016--2018 by Xiangdong Zeng <pssysrq@163.com>
%%
%% -* Notes on Thermodynamics and Statistical Physics *-
%%
%% This file may be distributed and/or modified under the
%% Creative Commons Attribution Share Alike 4.0 license.

\chapter{Boltzmann、Bose 与 Fermi 统计} \label{chap:three-distributions}

本章我们将从第\ref{chap:statistical-physics-basis}章获得的结果出发,推导一些重要的物理结论。

\section{定域子系热力学量的统计表述} \label{sec:thermodynamics-in-boltzmann-dist}

\subsection{单粒子配分函数}

定域子系遵循 Boltzmann 统计。其粒子数和内能由式~\eqref{eq:constraint-condition-of-dist} 给出:
\begin{braced}[\label{eq:constraint-condition-of-dist-restate}]
  N &=     \sum_l a_l = \sum_l \omega_l\ee^{-\alpha-\beta\varepsilon_l}, \\
  U &= E = \sum_l \varepsilon_l a_l = \sum_l \varepsilon_l\omega_l\ee^{-\alpha-\beta\varepsilon_l}.
\end{braced}
定义\kwd{单粒子配分函数}
\begin{equation}
  Z_1 \defeq \sum_l \omega_l\ee^{-\beta\varepsilon_l},
\end{equation}
则有
\begin{braced}[\label{eq:N-U-from-Z1}]
  N &= \sum_l a_l = \ee^{-\alpha} \sum_l \omega_l\ee^{-\beta\varepsilon_l}
     = \ee^{-\alpha} Z_1, \\
  U &= \sum_l \varepsilon_l a_l
     = \ee^{-\alpha} \, \qty(-\pdv{\beta} \sum_l \omega_l\ee^{-\beta\varepsilon_l})
     = \frac{N}{Z_1} \, \qty(-\pdv{\beta} Z_1) = -N\pdv{\beta} \ln Z_1.
\end{braced}

广义力 $Y$ 也可根据单粒子配分函数确定:
% TODO: (2018/01/27) 广义力
\begin{align}
  Y &= \sum_l \pdv{\varepsilon_l}{y} a_l
     = \sum_l \pdv{\varepsilon_l}{y} \omega_l\ee^{-\alpha-\beta\varepsilon_l} \notag \\
    &= \ee^{-\alpha} \, \qty(-\frac{1}{\beta} \pdv{y} \sum_l \omega_l\ee^{-\beta\varepsilon_l})
     = \frac{N}{Z_1} \, \qty(-\frac{1}{\beta} \pdv{y} Z_1) = -\frac{N}{\beta} \pdv{y} \ln Z_1.
\end{align}
在 $p$-$V$ 系统中,取 $Y=-p$ 和 $y=V$,可得
\begin{equation}
  p = \frac{N}{\beta} \pdv{V} \ln Z_1.
\end{equation}

\subsection{熵与 Boltzmann 关系}

下面我们来计算熵。根据热力学基本微分方程,我们有
\begin{equation}
  T\dd{S} = \dd{U} - Y\dd{y} = - N \dd(\pdv{\beta}\ln Z_1)
                               + \qty(\frac{N}{\beta}\pdv{y}\ln Z_1) \dd{y}.
\end{equation}
为了凑出全微分,需要在两边同乘 $\beta$:
\begin{align}
  \beta T\dd{S}
  &= N \, \qty[\qty(\pdv{y}\ln Z_1)\dd{y} - \beta\dd(\pdv{\beta}\ln Z_1)] \notag
  \intertext{注意到 $Z_1$ 只是 $\beta$ 和 $y$ 的函数,因此有}
  &= N \, \qty[\dd{\ln Z_1} - \qty(\pdv{\beta}\ln Z_1)\dd{\beta}
            - \dd(\beta\pdv{\beta}\ln Z_1) + \qty(\pdv{\beta}\ln Z_1)\dd{\beta}] \notag \\
  &= N \dd(\ln Z_1 - \beta\pdv{\beta}\ln Z_1).
\end{align}
令
% TODO: (2018/02/10) 与理想气体类比,如何操作
\begin{equation}
  \beta = \frac{1}{kT},
\end{equation}
立即得到
\begin{equation}
  \dd{S} = Nk \dd(\ln Z_1 - \beta\pdv{\beta}\ln Z_1).
\end{equation}
积分之后,可有
\begin{equation}
  S = Nk \, \qty(\ln Z_1 - \beta\pdv{\beta}\ln Z_1) + S_0
    = Nk \, \qty(\ln Z_1 - \beta\pdv{\beta}\ln Z_1).
\end{equation}
式中,我们取积分常数 $S_0=0$。
% TODO: (2018/02/10) 为什么这样取积分常数

代入粒子数和内能的表达式~\eqref{eq:N-U-from-Z1},可得
\begin{align}
  S &= Nk \, \qty(\ln N + \alpha + \beta\cdot\frac{U}{N}) \notag \\
    &= k \, \qty(N\ln N + \alpha N + \beta U) \notag
  \intertext{进一步代入式~\eqref{eq:constraint-condition-of-dist-restate}:}
    &= k \, \qty[N\ln N + \sum_l \qty(\alpha+\beta\varepsilon_l) a_l].
\end{align}
根据 Boltzmann 分布 \eqref{eq:maxwell-boltzmann-dist}~式
\begin{equation}
  a_l = \omega_l \, \ee^{-\alpha-\beta\varepsilon_l}
  \implies \alpha+\beta\varepsilon_l = -\ln\frac{a_l}{\omega_l},
\end{equation}
于是
\begin{equation} \label{eq:S-in-boltzmann-dist}
  S = k \, \qty(N\ln N - \sum_l a_l\ln\frac{a_l}{\omega_l}).
\end{equation}
另一方面,Boltzmann 体系中微观态数目为
\begin{equation}
  \Omega_\text{MB} = \frac{N!}{\prod_l a_l!} \prod_l {\omega_l}^{a_l}.
  \tag*{\eqref{eq:Omega-in-boltzmann-dist}}
\end{equation}
利用 Stirling 公式,计算得到
\footnote{推导可见 \ref{sec:maxwell-boltzmann-dist}~节。}
\begin{equation}
  \ln\Omega_\text{MB} = N\ln N - \sum_l a_l\ln\frac{a_l}{\omega_l}.
\end{equation}
与式~\eqref{eq:S-in-boltzmann-dist} 比较,立即有
\begin{equation}
  S = k\ln\Omega_\text{MB}.
\end{equation}
这称为 \kwd{Boltzmann 关系}。

在非简并条件下,Fermi--Dirac 分布与 Bose--Einstein 分布均趋同于 Boltzmann 分布(见
\ref{subsec:non-degenerate-condition-i}~小节):
\begin{equation}
  \Omega_\text{FD} = \Omega_\text{BE} = \frac{\Omega_\text{MB}}{N!}.
\end{equation}
此时,熵为
\begin{equation}
  S = Nk \, \qty(\ln Z_1 - \beta\pdv{\beta}\ln Z_1) - k\ln N! = k\ln\frac{\Omega_\text{MB}}{N!}.
\end{equation}
这里的 $N!$ 仍然是粒子全同性的要求。

利用熵和内能的表达式,还可从配分函数获得自由能:
\begin{equation}
  F = U-TS = -N\pdv{\beta}\ln Z_1 - NkT \, \qty(\ln Z_1 - \beta\pdv{\beta}\ln Z_1)
    = -NkT\ln Z_1.
\end{equation}
如果在非简并条件下(考虑粒子全同性),则为
\begin{equation}
  F = -NkT\ln Z_1 + k\ln N!.
\end{equation}

\subsection{经典过渡}

在 \ref{subsec:boltzmann-dist-classical}~小节中,我们给出了经典情形下的简并度:
\begin{equation}
  \omega_l = \frac{\incr{\omega_l}}{h_0^r}. \tag*{\eqref{eq:degeneracy-classical}}
\end{equation}
式中,$h_0$ 为相格尺度,而 $r$ 则为系统自由度。此时,单粒子配分函数可以写为
\begin{align}
  Z_1 &= \sum_l \omega_l\ee^{-\beta\varepsilon_l}
       = \sum_l \ee^{-\beta\varepsilon_l} \cdot \frac{\incr{\omega_l}}{h_0^r} \notag
  \intertext{相体积很小的时候可将求和化为积分:}
      &= \int \ee^{-\beta\varepsilon} \cdot \frac{\dd{\omega}}{h_0^r}
       = \frac{1}{h_0^r} \dotsint \ee^{-\beta\varepsilon(q,\,p)}
         \dd{q_1}\cdots\dd{q_r}\dd{p_1}\cdots\dd{p_r}.
\end{align}

在经典情形下,Boltzmann 分布过渡为
\begin{align}
  a_l &= \omega_l \, \ee^{-\alpha-\beta\varepsilon_l}
       = \ee^{-\alpha-\beta\varepsilon_l} \cdot \frac{\incr{\omega_l}}{h_0^r} \notag
  \intertext{代入 $N=\ee^{-\alpha}Z_1$,有}
      &= \frac{N}{Z_1} \cdot \ee^{-\beta\varepsilon_l} \cdot \frac{\incr{\omega_l}}{h_0^r}
       = N \cdot \frac{\ee^{-\beta\varepsilon_l}\incr{\omega_l}}%
                      {\sum_l\ee^{-\beta\varepsilon_l}\incr{\omega_l}}.
\end{align}
注意该表达式与 $h_0$ 无关。

\section{Maxwell 速度分布律}
