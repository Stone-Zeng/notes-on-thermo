%% Copyright (C) 2016--2018 by Xiangdong Zeng <pssysrq@163.com>
%%
%% -* Notes on Thermodynamics and Statistical Physics *-
%%
%% This file may be distributed and/or modified under the
%% Creative Commons Attribution Share Alike 4.0 license.

\chapter{热力学基础}

\section{平衡态及其描述} \label{sec:equilibrium-state}

\subsection{热力学系统}

热力学研究的对象是\kwd{热力学系统}(简称系统)。它是宏观体系,粒子数的量级约为
\num{e23}。与系统相对应的是\kwd{外界},也称为环境或热库。按照与外界的关系,
可将系统分为三种:\emph{孤立系}、\emph{封闭系}和\emph{开放系},
见表~\ref{tab:definition-of-systems}。

\begin{table}[h]
  \centering
  \begin{tabular}{cccc}
    \toprule
    & \kwd{孤立系 (isolated)} & \kwd{封闭系 (closed)} & \kwd{开放系 (open)} \\
    \midrule
    物质交换 & \xmark & \xmark & \cmark \\
    能量交换 & \xmark & \cmark & \cmark \\
    \bottomrule
  \end{tabular}
  \caption{三种热力学体系} \label{tab:definition-of-systems}
\end{table}

以后我们将会了解到,这三种系统,分别对应着\emph{微正则系综}、\emph{正则系综}%
与\emph{巨正则系综}。

\subsection{平衡态}

经典热力学研究\kwd{平衡态}。它有两个要素:状态不随时间变化、孤立系。若不是
孤立系,但满足状态不随时间变化的条件,则称\emph{稳恒状态},如日光灯
(显然有能量交换)。

平衡态只说明宏观性质不随时间变化,而微观态仍可有变化(微观粒子不断变化)。
因此称为\kwd{动态平衡}(就微观状态而言,也称为\emph{细致平衡})。

\subsection{平衡态的描述} \label{subsec:平衡态的描述}

平衡态可以利用\kwd{状态变量}(可以测量)来描述,如压强 $p$、体积 $V$、温度 $T$
等。需注意,这些量未必互相独立。因此需要选取\emph{独立}的\kwd{状态参量}。
当某一状态变量可以用其他状态参量来描述时,则称它为一个\kwd{状态函数}。

状态变量可分为两种:\kwd{强度量},如压强 $p$、温度 $T$、密度 $\rho$ 等,
它们不随粒子数的增加而增加;\kwd{广延量},如体积 $V$、内能 $U$、熵 $S$ 等,
它们随粒子数的增加而增加。

\section{温度;状态方程}

\subsection{热平衡定律}

\begin{theorem}[热平衡定律(热力学第零定律)]
  若物体 $A$ 分别与物体 $B$ 和 $C$ 处于热平衡,那么如果让 $B$ 与 $C$ 热接触,
  它们一定也处于热平衡。
\end{theorem}

该定律是温度测量的基础:互为热平衡的物体存在一个属于其固有属性的物理量,
即\kwd{温度}。一切互为热平衡的物体温度相等。

具体确定温度,需要选定\kwd{温标}。除了常见的摄氏、华氏温标,还有热膨胀温标、
热电阻温标、理想气体温标和热力学温标等。

\subsection{物态方程}

温度与其他状态参量 \footnote{即独立的状态变量。} 的函数关系称为\kwd{物态方程}:
\begin{equation}
  T = f(p, \, V)
\end{equation}
或
\begin{equation}
  g(p, \, V, \, T) = 0.
\end{equation}

Boyle 和 Mariotte 分别于 1662 年和 1676 年各自确立了\kwd{Boyle--Mariotte定律}:
\begin{equation}
  pV = p_0 V_0 \qc\qif* T = \const;
\end{equation}
1802年,Gay-Lussac 确立了\kwd{Gay-Lussac定律}
\footnote{事实上,在 1787 年,Charles 就已经发现了这一定律,只是当时未发表,
  也未被人注意。}:
\begin{equation}
  \frac{V}{T} = \frac{V_0}{T_0} \qc\qif* p = \const
\end{equation}
综合两式,可得
\footnote{该式中常数的值显然与物质的量 $n$ 有关,故用带引号的
  $\text{``const.''}$ 表示。}
\begin{equation} \label{eq:ideal-gas-law-a}
  \frac{p V}{T} = \frac{p_0 V_0}{T_0} = \text{``const.''} = n R,
\end{equation}
即\kwd{理想气体物态方程}
\begin{equation} \label{eq:ideal-gas-law-b}
  \boxed{pV = nRT = N \kB T}
\end{equation}

\begin{proof}
  式~\eqref{eq:ideal-gas-law-a} 的推导如下。
  设某气体初状态为 $p_0$、$V_0$、$T_0$,末状态为 $p$、$V$、$T$。

  先保持温度 $T_0$ 不变,则有
  \begin{equation} \label{eq:ideal-gas-law-proof}
    p_0 V_0 = p V';
  \end{equation}
  再假设压强 $p$ 不变,于是
  \begin{equation}
    \frac{V'}{T_0} = \frac{V}{T},
  \end{equation}
  即
  \begin{equation}
    V' = \frac{V T_0}{T}.
  \end{equation}
  代入 \eqref{eq:ideal-gas-law-proof}~式,即有
  \begin{equation}
    \frac{p V}{T} = \frac{p_0 V_0}{T_0} = \text{``const.''}.
  \end{equation}
\end{proof}

除了通过统计物理推导,也可通过测量膨胀系数、压强系数和等温压缩系数
(亦可简称压缩系数)来得到物态方程。

\kwd{膨胀系数} $\alpha$ 定义为
\begin{equation}
  \alpha \defeq \frac{1}{V} \qty(\pdv{V}{T})_p,
\end{equation}
\kwd{压强系数} $\beta$ 定义为
\begin{equation}
  \beta \defeq \frac{1}{p} \qty(\pdv{p}{T})_V,
\end{equation}
\kwd{等温压缩系数} $\kappa_T$ 定义为
\begin{equation}
  \kappa_T \defeq -\frac{1}{V} \qty(\pdv{V}{p})_T.
\end{equation}

可以证明,以上三个常数满足下面的关系:
\begin{equation} \label{eq:relation-of-alpha-beta-gamma}
  \alpha = \beta \kappa_T p.
\end{equation}
该式说明三者并非独立。通常会直接测量 $\alpha$ 和 $\kappa_T$,而通过计算得到
$\beta$。

\begin{proof}
  先证明两个结论。若 $x, \, y, \, z$ 满足 $F(x, y, z) = \const$,其中
  $F(x, y, z)$ 是一个可微函数。那么
  \begin{braced}
    \dd{x} &= \qty(\pdv{x}{y})_z \dd{y} + \qty(\pdv{x}{z})_y \dd{z}, \\
    \dd{z} &= \qty(\pdv{z}{x})_y \dd{x} + \qty(\pdv{z}{y})_x \dd{y}.
  \end{braced}
  把第一式代入第二式,得
  \begin{align}
    \dd{z} &= \qty(\pdv{z}{x})_y
      \qty[\qty(\pdv{x}{y})_z \dd{y} + \qty(\pdv{x}{z})_y \dd{z}]
      + \qty(\pdv{z}{y})_x \dd{y} \notag \\
    &= \qty[\qty(\pdv{z}{x})_y \qty(\pdv{x}{y})_z + \qty(\pdv{z}{y})_x] \dd{y}
      + \qty(\pdv{z}{x})_y \qty(\pdv{x}{z})_y \dd{z}.
  \end{align}
  于是
  \begin{equation} \label{eq:coefficient-of-dy-and-dz}
    \qty[\qty(\pdv{z}{x})_y \qty(\pdv{x}{y})_z + \qty(\pdv{z}{y})_x] \dd{y}
    + \qty[\qty(\pdv{z}{x})_y \qty(\pdv{x}{z})_y - 1] \dd{z} = 0.
  \end{equation}
  因为 $y$ 与 $z$ 是独立的变量,所以 $\dd{y}$ 与 $\dd{z}$ 也是独立的。这就要求
  它们的系数均为零。由 $\dd{z}$ 的系数为零,可得到\kwd{倒数关系}:
  \begin{equation} \label{eq:reciprocal-relation}
    \qty(\pdv{z}{x})_y \qty(\pdv{x}{z})_y = 1;
  \end{equation}
  由 $\dd{y}$ 的系数为零,可得
  \begin{equation} \label{eq:cyclic-relation}
    \qty(\pdv{z}{x})_y \qty(\pdv{x}{y})_z = -\qty(\pdv{z}{y})_x.
  \end{equation}
  利用倒数关系式~\eqref{eq:reciprocal-relation},即有\kwd{偏导数三乘积法则}:
  \begin{equation}
    \qty(\pdv{x}{y})_z \qty(\pdv{y}{z})_x \qty(\pdv{z}{x})_y = -1.
  \end{equation}

  根据式~\eqref{eq:cyclic-relation},用热力学中的变量将其写成
  \begin{equation}
    \qty(\pdv{V}{T})_p = -\qty(\pdv{V}{p})_T \qty(\pdv{p}{T})_V.
  \end{equation}
  代入前文中的定义,就得到了 \eqref{eq:relation-of-alpha-beta-gamma}~式:
  \begin{equation}
    \alpha = \beta \kappa_T p.
  \end{equation}
\end{proof}

考虑分子之间的相互作用后,对理想气体进行修正,这就是\kwd{van der Waals 气体}。
其物态方程为
\begin{equation} \label{eq:van-der-waals-gas}
  \qty(p + \frac{n^2 a}{V^2}) (V - nb) = nRT,
\end{equation}
其中 $n^2 a / V^2$ 代表分子之间吸引力所引起的修正,而 $nb$ 则代表排斥力所引起
的修正。

\section{功;热力学第一定律}

\subsection{准静态过程的功}

静态过程即平衡态。

\kwd{准静态过程}指过程中的每一步都是平衡态。这就要求外界的条件变化地足够缓慢。
令 $\tau$ 为外界条件变化的特征时间,$\incr{t}$ 为系统趋于与外界条件对应的平衡态
的特征时间(即\kwd{弛豫时间}),那么准静态过程相当于
\begin{equation}
  \frac{\tau}{\incr{t}} \to 0
\end{equation}
的极限。

在一个过程中,如果每一步都可以在相反的方向进行而不引起外界的变化,则称为%
\kwd{可逆过程}。可逆过程的实质是没有耗散。

下面举两个准静态过程的例子。

第一个例子是 $p$-$\!V$ 系统。考虑盛于带活塞容器内的气体。气体体积变化 $\dd{V}$ 时,
\emph{外界对系统}做功
\begin{equation}
  \dbar{W} = -p \dd{V}.
\end{equation}
这里的 $p$ 指外界对系统的压强。由于是准静态过程,它又等于容器内气体对器壁的
压强。外界压力作用下体积减小,$\dd{V} < 0$;而外界却对系统做正功,因此会出现
负号。如果令 $W'$ 为\emph{系统对外界}做的功,则有
\begin{equation}
  \dbar{W'} = - \dbar{W} = p \dd{V}.
\end{equation}

体积由 $V_1$ 变化到 $V_2$,外界对系统做的总功为
\begin{equation}
  W = -\int_{V_1}^{V_2} p \dd{V}.
\end{equation}

% \begin{figure}[h]
%   \centering
%   \begin{asy}
%     pair O = (0, 0), x_axes = (10, 0), y_axes = (0, 10);
%     
%     pair p1 = (2.2, 7.5), p2 = (8, 2.5), p3 = (4, 4.5);
%     pair p1_x = (p1.x, 0), p2_x = (p2.x, 0);
%     
%     fill(p1_x--p1..p3..p2--p2_x--cycle, color1 + opacity(0.2));
%     
%     draw(Label("$V$", EndPoint), O--x_axes, Arrow);
%     draw(Label("$p$", EndPoint), O--y_axes, Arrow);
%     
%     draw(p1..p3..p2, linewidth(1) + color1);
%     draw(p1_x--p1, dashed + color1);
%     draw(p2_x--p2, dashed + color1);
%     
%     label("$O$", O, SW);
%     label("状态 $1$", p1, (0, 2));
%     label("状态 $2$", p2, (0.5, 2));
%     label("$V_1$", p1_x, S);
%     label("$V_2$", p2_x, S);
%   \end{asy}
%   \caption{$p \text{-} V$ 系统的状态空间}
%   \label{fig:pV-diagram}
% \end{figure}

在 $p$-$\!V$ 图中标出状态 1 和状态 2,则连接它们的曲线就表示一个准静态过程,曲线
下的面积等于 $-W$,如图~\ref{fig:pV-diagram} 所示。显然,$W$ 与路径有关,即不是
一个全微分,因此我们用“$\dbar{W}$”表示微功。

另一个例子是电介质极化系统。考虑均匀电场 $\V{E}$ 中的均匀电介质。电位移 $\V{D}$
变化 $\dd{\V{D}}$ 时,电场做功
\begin{equation}
  \dbar{W} = V \V{E} \cdot \dd{\V{D}},
\end{equation}
其中的 $V$ 表示电介质的体积。这里,$\V{E}$ 是变化的外因,$\V{D}$ 则是内果。
令 $\V{P}$ 为极化强度,则
\begin{equation}
  \V{D} = \pfs \V{E} + \V{P}.
\end{equation}
于是
\begin{equation}
  \dbar{W} = V \pfs \V{E} \cdot \dd{\V{E}} + V \V{E} \cdot \dd{\V{P}}
  = V \dd(\frac{1}{2} \pfs \V{E}^2) + V \V{E} \cdot \dd{\V{P}}.
\end{equation}
式中的第一项代表电场能量的变化,第二项则代表\emph{极化功}。

对以上两种情况进行推广,可得
\begin{equation} \label{eq:work-general}
  \dbar{W} = Y_1 \dd{y_1} + Y_2 \dd{y_2} + \cdots + Y_n \dd{y_n}
  = \sum_{i=1}^{n} Y_i \dd{y_i}.
\end{equation}
这里的 $Y_i$ 是\kwd{广义力},如压强 $p$、电场强度 $\V{E}$、磁场强度 $\V{H}$ 等;
$y_i$ 是\kwd{广义坐标},如体积 $V$、电位移 $\V{D}$、磁感应强度 $\V{B}$ 等。

\subsection{非静态过程}

等容过程是一种非静态过程。由于体积不变,即 $\dd{V} = 0$,因此
\begin{equation} \label{eq:work-fixed-V}
  \dbar{W} = 0.
\end{equation}

另一种非静态过程是等压过程。由于\emph{外界}压强不变,即 $p_\text{ext}=\const$,
因此体积从 $V_1$ 变化到 $V_2$ 时外界对系统做功
\begin{equation}
  W = -p_\text{ext} \qty(V_2 - V_1) = -p_\text{ext} \incr{V}.
\end{equation}
若系统初、终态压强相等,并等于外部压强,即
\begin{equation}
  p_1 = p_2 \defeq p = p_\text{ext} .
\end{equation}
则有
\begin{equation} \label{eq:work-fixed-p}
  W = -p \qty(V_2 - V_1) = -p \incr{V}.
\end{equation}
这里的 $p$ 指系统内的压强。注意,我们并没有要求\emph{整个过程}中系统内的压强
都等于 $p$,只需要初、终态压强等于 $p$ 即可。

\subsection{热力学第一定律}

\kwd{热力学第一定律}的主要建立者有 Mayer、Joule、Helmholtz、Carnot 等。它描述
了\emph{功}、\emph{热量}、\emph{内能}三者之间的关系,是\kwd{能量守恒定律}在宏观
热现象过程中的表现形式。

\begin{theorem}[能量守恒定律]
  自然界中的一切物质都具有能量。能量有各种不同的形式,能够从一种形式转化为
  另一种形式,从一个物体传递给另一个物体。在转化和传递中,能量的总量不变。
\end{theorem}

对绝热过程而言,从状态 1 变化到状态 2 的过程中,只有外界做功,因此
\begin{equation}
  U_2 - U_1 = W_\text{a}.
\end{equation}
下标“a”表示绝热。

而对于非绝热过程,由于既有外界做功,又有热量传递,因此
\begin{equation} \label{eq:first-law-integral-form}
  U_2 - U_1 = W + Q.
\end{equation}
写成微分形式,为
\begin{equation} \label{eq:first-law-differential-form}
  \boxed{\dd{U} = \dbar{W} + \dbar{Q}}
\end{equation}
这实际上就是热力学第一定律最常用的表述。

设两个全同系统内能分别为 $U_1$、$U_2$。则总内能 $U_\text{total}$ 除了
$U_1+U_2$,还需包括由于界面效应所导致的 $U_{12}$。但在热力学中,界面效应基本
可以忽略,因此近似有
\begin{equation}
  U_\text{total} = U_1 + U_2.
\end{equation}
这说明内能是一个广延量。

\section{热容与焓;理想气体的性质} \label{sec:heat-capacity-enthalpy-ideal-gas}

\subsection{热容与焓} \label{subsec:heat-capacity-and-enthalpy}

对于过程 $y$,定义\kwd{热容}
\begin{equation}
  C_y \defeq \frac{\dbar{Q_y}}{\dd{T}},
\end{equation}
式中的 $\dbar{Q_y}$ 是温度升高 $\dd{T}$ 时系统所吸收的热量。

对于\emph{等容过程},有\kwd{定容热容}
\begin{equation}
  C_V \defeq \frac{\dbar{Q_V}}{\dd{T}}.
\end{equation}
根据式~\eqref{eq:work-fixed-V},可知
\begin{equation}
  \dbar{Q_V} = \dd{U} - \dbar{W} = \dd{U} - 0 = \dd{U},
\end{equation}
因此
\begin{equation} \label{eq:heat-capacity-fixed-V}
  C_V = \qty(\pdv{U}{T})_V.
\end{equation}

对于\emph{等压过程},有\kwd{定压热容}
\begin{equation}
  C_p \defeq \frac{\dbar{Q_p}}{\dd{T}}.
\end{equation}
根据式~\eqref{eq:work-fixed-p},可知
\begin{equation} \label{eq:dQ-fixed-p}
  \dbar{Q_p} = \dd{U} - \dbar{W} = \dd{U} - p \dd{V},
\end{equation}
因此
\begin{equation} \label{eq:heat-capacity-fixed-p}
  C_V = \qty(\pdv{U}{T})_p + p \, \qty(\pdv{V}{T})_p.
\end{equation}

% 可以发现,
% \begin{equation}
%   C_p = C_V + p \, \qty(\pdv{V}{T})_p.
% \end{equation}
% 这是它们的关系之一。

下面引入一个新的物理量——\kwd{焓},其定义为
\begin{equation}
  H \defeq U + p V.
\end{equation}
由于 $U$、$p$ 和 $V$ 均是状态函数,因此焓也是状态函数。利用焓,
可将式~\eqref{eq:heat-capacity-fixed-p} 改写为
\begin{equation} \label{eq:heat-capacity-fixed-p-with-H}
  C_p = \qty(\pdv{H}{T})_p,
\end{equation}
同时将式~\eqref{eq:dQ-fixed-p} 改写为
\begin{equation}
  \dbar{Q_p} = \dd{H}.
\end{equation}
这就是说,\emph{在等压过程中,物体吸收的热量等于焓的增加量}。

显然,热容与焓均是广延量。单位质量下的热容称为\kwd{比热容}或\kwd{比热},
它是强度量。

\subsection{理想气体的性质} \label{subsec:ideal-gas-property} %\label{subsec:理想气体的性质}

% \begin{figure}[h]
%   \begin{asy}
%     pair p1 = (0, 7), p2 = (0, 0), p3 = (12, 0), p4 = (p3.x, p1.y), p5 = (0, 6), p6 = (p3.x, p5.y);
%     pair m1 = (p1+p4)/2, m2 = (p2+p3)/2;
%     transform myReflect = reflect(m1, m2);
%     
%     real boxWidth = 3, boxHeight = 2;
%     pair pA1 = (1.8, 2.5), pA2 = (pA1.x+boxWidth, pA1.y), pA3 = (pA2.x, pA1.y+boxHeight), pA4 = (pA1.x, pA3.y);
%     path boxA = pA1--pA2--pA3--pA4--cycle;
%     path boxB = myReflect*boxA;
%     
%     real pipeSize = 0.2, pipeHeight = 3.5;
%     pair ppA1 = (pA4.x+(boxWidth-pipeSize)/2, pA4.y),
%          ppA2 = (ppA1.x, ppA1.y+pipeHeight),
%          ppA3 = (ppA1.x+pipeSize, ppA1.y),
%          ppA4 = (ppA3.x, ppA2.y-pipeSize);
%     pair ppB1 = myReflect*ppA1, ppB2 = myReflect*ppA2, ppB3 = myReflect*ppA3, ppB4 = myReflect*ppA4;
%     path pipe = ppA1--ppA2--ppB2--ppB1--ppB3--ppB4--ppA4--ppA3--cycle;
%     
%     real valveHeight = 0.6;
%     pair pv1 = (m1.x, ppA2.y-(valveHeight+pipeSize)/2), pv2 = (pv1.x, pv1.y+valveHeight);
%     
%     real thermometerHeight = 5, thermometerR = 0.2;
%     pair pt1 = (1, 3), pt2 = (pt1.x, pt1.y+thermometerHeight);
%     
%     fill(p5--p2--p3--p6--cycle, color1+opacity(0.2));
%     draw(p1--p2--p3--p4, linewidth(1)+color1);
%     draw(boxA, linewidth(1)+color2);
%     draw(boxB, linewidth(1)+color2);
%     draw(pipe, linewidth(1)+color2);
%     draw(pv1--pv2, linewidth(2)+color3);
%     draw(pt1--pt2, linewidth(2)+color4);
%     fill(circle(pt1, thermometerR), color4);
%     
%     label("A", (ppA1+ppA3)/2, (0, -3));
%     label("B", (ppB1+ppB3)/2, (0, -3));
%     
%     pair pwLabel1 = (p3.x-1, p3.y+1.4), pwLabel2 = (p3.x+1, p3.y+2.5);
%     draw(pwLabel1--pwLabel2);
%     label("水槽", pwLabel2, E);
%     
%     pair ptLabel1 = (pt1.x-0.25, pt2.y-1), ptLabel2 = (pt1.x-1.5, pt2.y+0.4);
%     draw(ptLabel1--ptLabel2);
%     label("温度计", ptLabel2, N);
%     
%     pair pvLabel1 = (pv1.x+0.25, pv2.y-0.1), pvLabel2 = (pv1.x+1.5, pv2.y+1);
%     draw(pvLabel1--pvLabel2);
%     label("阀门", pvLabel2, N);
%   \end{asy}
%   \caption{Joule实验的装置}
%   \label{fig-Joule-experiment}
% \end{figure}

先介绍 Joule 实验(1845 年)。设有一容器,分为 A、B 两个相同的部分。将它们置于
水槽内,并通过带阀门的细管相连,水温可以通过温度计测量,
如图~\ref{fig-Joule-experiment} 所示。

首先,在 A 内充满气体,而使 B 保持真空。然后打开阀门,气体将\emph{自由膨胀},
并充满整个容器。Joule 的实验结果是前后水温不变。

由于是等容过程,因此 $W=0$;水温不变,说明 $Q=0$。根据热力学第一定律,有
$\incr{U}=0$。假设内能是温度和体积的函数,即
\begin{equation}
  U = U(T, \, V).
\end{equation}
根据\emph{偏导数三乘积法则},可知
\begin{equation}
  \qty(\pdv{U}{V})_T \qty(\pdv{V}{T})_U \qty(\pdv{T}{U})_V = -1.
\end{equation}
于是
\begin{equation}
  \qty(\pdv{U}{V})_T
  = -\qty(\pdv{T}{U})_V^{-1} \qty(\pdv{V}{T})_U^{-1}
  = -\qty(\pdv{U}{T})_V \qty(\pdv{T}{V})_U.
\end{equation}
注意到系统满足 $\incr{U}=0$ 以及 $\incr{T}=0$,因此有
\begin{equation}
  \qty(\pdv{T}{V})_U = 0 \implies \qty(\pdv{U}{V})_T = 0.
\end{equation}
这就说明 $U = U(T)$,即内能只和温度有关。

不过,限于当时的技术水平,Joule 实验较为粗糙。更精确的实验表明,实际气体的内能
不仅与 $T$ 有关,还与 $V$ 有关。但对于理想气体,以上结论仍然成立,因此理想气体
具有如下两条性质:
\begin{braced}
  & pV = nRT,  \label{eq:ideal-gas-property-equation-of-state}\\
  & U  = U(T). \label{eq:ideal-gas-property-internal-energy}
\end{braced}
就目前而言,这两条性质彼此是独立的。但利用热力学第二定律或统计物理,可以由
\eqref{eq:ideal-gas-property-equation-of-state}~式推导
\eqref{eq:ideal-gas-property-internal-energy}~式。
%[见\secref{sec:Maxwell关系}\subsecref{subsec:简单应用_OF_MAXWELL关系}中的\egref{EG_pd_U/pd_V_WITH_FIXED_T}]

对于理想气体,根据焓的定义,可得
\begin{equation}
  H = U + pV = U(T) + nRT = H(T),
\end{equation}
即焓也仅是温度的函数。从而根据式~\eqref{eq:heat-capacity-fixed-V} 和
\eqref{eq:heat-capacity-fixed-p-with-H},又有
\begin{equation}
  C_p - C_V = \qty(\pdv{H}{T})_p - \qty(\pdv{U}{T})_V
  = \dv{H}{T} - \dv{U}{T} = \dv{T} \qty\Big(pV) = \dv{T} \qty\Big(nRT) = nR.
  \label{eq:C_p-C_V-for-ideal-gas}
\end{equation}

定义\kwd{热容比}
\footnote{也称为\kwd{绝热指数},原因见第\ref{subsec:adiabatic-process}小节。}
$\gamma$ 为 $C_p$ 与 $C_V$ 之比:
\begin{equation} \label{eq:heat-capacity-ratio}
  \gamma \defeq \frac{C_p}{C_V}.
\end{equation}
对于理想气体,显然也有
\begin{equation}
  \gamma = \gamma(T).
\end{equation}

由式~\eqref{eq:C_p-C_V-for-ideal-gas} 和 \eqref{eq:heat-capacity-ratio},
可以解得
\begin{braced}
  C_V &= \frac{1}{\gamma-1} nR,      \label{eq:C_V-in-heat-capacity-ratio} \\
  C_p &= \frac{\gamma}{\gamma-1} nR. \label{eq:C_p-in-heat-capacity-ratio}
\end{braced}
因为 $\gamma$ 可以通过实验测量,由此算出热容后,就可以确定理想气体的内能与焓:
\begin{braced}
  U(T) &= \int C_V(T) \dd{T} + U_0, \\
  H(T) &= \int C_p(T) \dd{T} + H_0,
\end{braced}
其中的 $U_0$ 和 $H_0$ 是积分常数。将上式写成微分形式,为
\begin{braced}
  & \dd{U} = C_V(T) \dd{T}, \label{eq:dU-for-ideal-gas} \\
  & \dd{H} = C_p(T) \dd{T}. \label{eq:dH-for-ideal-gas}
\end{braced}

\subsection{绝热过程的过程方程} \label{subsec:adiabatic-process}

\emph{本节叙述均只针对理想气体。}

根据理想气体状态方程
\begin{equation}
  pV = nRT,
\end{equation}
有
\begin{equation}
  \dd{T} = \frac{p\dd{V}+V\dd{p}}{nR}.
\end{equation}
对于绝热过程,根据热力学第一定律,有
\begin{align}
  0 = \dbar{Q} &= \dd{U} + p\dd{V} = C_V\dd{T} + p\dd{V} \notag \\
  &= C_V \, \frac{p\dd{V}+V\dd{p}}{nR} + p\dd{V} \notag \\
  &= \qty(\frac{C_V}{nR} + 1) p\dd{V} + \frac{C_V}{nR} V\dd{p} \notag
  \intertext{利用式~\eqref{eq:C_V-in-heat-capacity-ratio},可得}
  &= \frac{\gamma}{\gamma-1} p \dd{V} + \frac{1}{\gamma - 1} V \dd{p},
\end{align}
即
\begin{equation}
  \frac{\dd{p}}{p} + \gamma \frac{\dd{V}}{V} = 0.
\end{equation}
假定 $\gamma$ 为常数,积分可得
\begin{equation} \label{eq:equation-of-adiabatic-process-in-p-V}
  \ln p + \gamma \ln V = \ln(pV^\gamma) = \const \implies pV^\gamma = \const
\end{equation}
改用其他变量,可把该式写成
\begin{equation} \label{eq:equation-of-adiabatic-process-in-p-T}
  p^{(1 - \gamma) / \gamma} T = \const
\end{equation}
或
\begin{equation} \label{eq:equation-of-adiabatic-process-in-T-V}
  T V^{\gamma - 1} = \const
\end{equation}
的形式。

\begin{example}[海拔与气温的关系]
  下面推导气温垂直递减率。首先考虑干燥空气的温度\emph{绝热}递减率。假设空气是
  理想气体。

  因为是绝热过程,我们有
  \begin{equation}
    p^{(1-\gamma) / \gamma} T = \const
  \end{equation}
  两边求微分,得
  \begin{equation}
    \frac{1-\gamma}{\gamma} p^{(1-\gamma)/\gamma-1} T \dd{p}
    + p^{(1-\gamma)/\gamma} \dd{T} = 0,
  \end{equation}
  即
  \begin{equation} \label{eq:dT-over-dp-in-example-of-lapse-rate}
    \dv{T}{p} = \frac{1-\gamma}{\gamma} \frac{T}{p}.
  \end{equation}

  假设大气处于平衡状态,则有
  \begin{equation}
    p = \rho gz \implies \dd{p} = -\rho g \dd{z},
  \end{equation}
  其中的 $g$ 是重力加速度,而 $\rho$ 是空气密度:
  \begin{equation}
    \rho = \frac{m}{V} = \frac{nM}{nRT/p} = \frac{pM}{RT},
  \end{equation}
  这里,$M$ 是空气的平均摩尔质量,$m$、$V$ 则分别是一定量空气的质量和体积。
  代入 \eqref{eq:dT-over-dp-in-example-of-lapse-rate}~式,可得
  \begin{equation}
    \dv{T}{z} = -\frac{1-\gamma}{\gamma} \frac{\rho gT}{p}
    = -\frac{1-\gamma}{\gamma} \frac{Mg}{R}.
  \end{equation}
  代入空气的热容比 $\gamma=1.4$、平均摩尔质量 $M=\SI{28.8e-3}{\kg\per\mol}$
  等数值,可得
  \begin{equation}
    \dv{T}{z} = \SI{-9.7}{\kelvin\per\km}.
  \end{equation}

%  若为水汽饱和的湿空气,有下面的近似公式:%TODO:20160315 参考文献
%  \begin{equation} \label{EQ_SATURATED_ADIABATIC_LAPSE_RATE}
%    \frac{\dd T}{\dd z}
%    = -g \, \frac{1 + \dfrac{L_\text{vap} r}{R_\text{s,\,dry} T}}{c_{p,\,\text{dry}} + \dfrac{L_\text{vap}^2 r}{R_\text{s,\,water} T^2}}
%    = -g \, \frac{1 + \dfrac{L_\text{vap} r}{R_\text{s,\,dry} T}}{c_{p,\,\text{dry}} + \dfrac{L_\text{vap}^2 r \e}{R_\text{s,\,dry} T^2}} \comma
%  \end{equation}
%  式中的各符号见表~\ref{TAB_SYMBOLS_IN_SATURATED_ADIABATIC_LAPSE_RATE}。%FIXME:20160401 qed位置
%
%  \begin{myTable}{Mcc}{式~\eqref{EQ_SATURATED_ADIABATIC_LAPSE_RATE} 中所用到的符号}{TAB_SYMBOLS_IN_SATURATED_ADIABATIC_LAPSE_RATE}
%    \toprule
%    \text{\kwd{符号}} & \kwd{说明} & \kwd{数值} \\%HACK:20160330 表格首行加粗
%    \midrule
%    g & 重力加速度 & \SI{9.8076}{\metre\per\second\squared} \\
%    L_\text{vap} & 水的汽化热 & \SI{2257}{\kilo\joule\per\kg} \\
%    c_{p,\,\text{dry}} & 干燥空气的定压比热容 & \SI{1003.5}{\joule\per\kg\per\kelvin} \\
%    R_\text{s,\,dry} & 干燥空气的气体常数 & \SI{287}{\joule\per\kg\per\kelvin} \\
%    R_\text{s,\,water} & 水蒸气的气体常数 & \SI{461.5}{\joule\per\kg\per\kelvin} \\
%    \e = R_\text{s,\,dry} / R_\text{s,\,water} & 干燥空气与水蒸气的气体常数之比 & 0.622 \\
%    e & 饱和空气的水蒸气分压 & —— \\
%    p & 饱和空气的气压 & —— \\
%    r = \e e / (p - e) & 水蒸气的质量与干燥空气质量的混合比例 & —— \\
%    T & 饱和空气的温度 & —— \\
%    \bottomrule
%  \end{myTable}
\end{example}

\section{理想气体与 Carnot 循环;热力学第二定律}
\label{sec:ideal-gas-carnot-cycle-and-second-law}

\subsection{Carnot 循环} \label{subsec:carnot-cycle}

\kwd{Carnot 循环}分为四个过程,如图~\ref{fig:carnot-cycle} 所示。
\begin{itemize}
  \item 等温膨胀 (I)  :$(T_\text{H}, \, V_1) \to (T_\text{H}, \, V_2)$,
  \item 绝热膨胀 (II) :$(T_\text{H}, \, V_2) \to (T_\text{C}, \, V_3)$,
  \item 等温压缩 (III):$(T_\text{C}, \, V_3) \to (T_\text{C}, \, V_4)$,
  \item 绝热压缩 (IV) :$(T_\text{C}, \, V_4) \to (T_\text{H}, \, V_1)$。
\end{itemize}
这里的 $T_\text{H}$ 和 $T_\text{C}$ 分别指高温和低温,并且还有 $V_1 < V_2$,
$V_4 < V_3$。

% \begin{figure}[h]
%   \begin{asy}
%     import graph;
%     pair O = (0, 0), x_axes = (10, 0), y_axes = (0, 10);
%     draw(Label("$V$", EndPoint), O--x_axes, Arrow);
%     draw(Label("$p$", EndPoint), O--y_axes, Arrow);
%     
%     real gamma = 2.5;
%     real x1 = 2, x4 = 9;
%     real c1 = 10, c2 = 18;
%     real c3 = c2*x1^(gamma-1), c4 = c1*x4^(gamma-1);
%     real x2 = x1*(c1/c2)^(1/(1-gamma)), x3 = x4*(c2/c1)^(1/(1-gamma));
%     
%     path path1 = graph(new real(real x) {return c2/x;}, x1, x3);
%     path path2 = graph(new real(real x) {return c4/x^gamma;}, x3, x4);
%     path path3 = reverse(graph(new real(real x) {return c1/x;}, x2, x4));
%     path path4 = reverse(graph(new real(real x) {return c3/x^gamma;}, x1, x2));
%     //path path_TH = graph(new real(real x) {return c2/x;}, x3, 8);
%     //path path_TC = graph(new real(real x) {return c1/x;}, 1.7, x2);
%     
%     pair p1 = (x1, c2/x1), p2 = (x3, c2/x3), p3 = (x4, c1/x4), p4 = (x2, c1/x2);
%     
%     pen pen1 = linewidth(1)+color1;
%     
%     fill(path1 & path2 & path3 & path4 & cycle, color1+opacity(0.2));
%     
%     //draw(path_TH, dashed + color1);
%     //draw(path_TC, dashed + color1);
%     draw(path1, pen1, Arrow(position = Relative(0.7), arrowhead = HookHead, size = 4));
%     draw(path2, pen1, Arrow(position = Relative(0.5), arrowhead = HookHead, size = 4));
%     draw(path3, pen1, Arrow(position = Relative(0.7), arrowhead = HookHead, size = 4));
%     draw(path4, pen1, Arrow(position = Relative(0.45), arrowhead = HookHead, size = 4));
%     
%     draw(Label("$V_1$", EndPoint, black), p1--(p1.x, 0), dashed + color1);
%     draw(Label("$V_2$", EndPoint, black), p2--(p2.x, 0), dashed + color1);
%     draw(Label("$V_3$", EndPoint, black), p3--(p3.x, 0), dashed + color1);
%     draw(Label("$V_4$", EndPoint, black), p4--(p4.x, 0), dashed + color1);
%     
%     label("状态1", p1, N);
%     label("状态2", p2, NE);
%     label("状态3", p3, E);
%     label("状态4", p4, SW, Fill(white));
%     
%     label("I", path1, align = Relative(W));
%     label("II", path2, align = Relative(W));
%     label("III", path3, align = Relative(W));
%     label("IV", path4, align = Relative(W), Fill(white));
%   \end{asy}
%   \caption{Carnot循环示意图}
%   \label{fig:carnot-cycle}
% \end{figure}

根据热力学第一定律,有
\begin{equation}
  \oint \dd{U} = \oint \dbar{Q} + \oint \dbar{W} = 0.
\end{equation}
其中的“$\oint$”代表沿循环过程的积分。

整个过程中对外做的净功(它等于图~\ref{fig:carnot-cycle} 中曲线包围起来的面积)
\begin{equation}
  W' = -\oint \dbar{W} = \oint \dbar{Q} = Q_\text{H} + Q_\text{C},
\end{equation}
其中的 $Q_\text{H}$ 和 $Q_\text{C}$ 分别为高温和低温时吸收的热量(可以有正负)。

若工作物质为理想气体,则有
\begin{equation} \label{eq:Q_H-in-carnot-cycle}
  Q_\text{H} = \incr{U_\text{I}} - W_\text{I} = -W_\text{I}
  = \int_{V_1}^{V_2} p \dd{V}
  = nRT_\text{H} \int_{V_1}^{V_2} \frac{\dd{V}}{V}
  = nRT_\text{H} \ln \frac{V_2}{V_1} > 0.
\end{equation}
同理,还有
\begin{equation} \label{eq:Q_C-in-carnot-cycle}
  Q_\text{C} = -nRT_\text{C} \ln \frac{V_3}{V_4} < 0.
\end{equation}

根据绝热过程的过程方程~\eqref{eq:equation-of-adiabatic-process-in-T-V},有
\begin{braced}
  & T_\text{H} V_2^{\gamma-1} = T_\text{C} V_3^{\gamma-1}, \\
  & T_\text{H} V_1^{\gamma-1} = T_\text{C} V_4^{\gamma-1},
\end{braced}
两边分别相除,得
\begin{equation} \label{eq:V_2-over-V_1-in-carnot-cycle}
  \frac{V_2}{V_1} = \frac{V_3}{V_4}.
\end{equation}

定义\kwd{热机效率}
\begin{equation}
  \eta \defeq \frac{W'}{Q_\text{H}}
  = \frac{Q_\text{H} - \abs{Q_\text{C}}}{Q_\text{H}}
  = 1 - \frac{\abs{Q_\text{C}}}{Q_\text{H}}.
\end{equation}
对于理想气体,代入式~\eqref{eq:Q_H-in-carnot-cycle} 和
\eqref{eq:Q_C-in-carnot-cycle},并利用式~%
\eqref{eq:V_2-over-V_1-in-carnot-cycle},可得
\begin{equation}\label{eq:efficiency-of-carnot-cycle-for-ideal-gas}
  \eta = 1 - \frac{T_\text{C}}{T_\text{H}} \frac{\ln(V_3/V_4)}{\ln(V_2/V_1)}
       = 1 - \frac{T_\text{C}}{T_\text{H}}.
\end{equation}
%TODO:20160320 非理想气体的情况,见作业

若 Carnot 循环反向进行,就成为\kwd{Carnot 制冷机}。其\emph{制冷效率}定义为
\begin{equation}
  \varepsilon = \frac{Q_\text{C}}{W}
  = \frac{Q_\text{C}}{Q_\text{H} - Q_\text{C}}
  = \frac{T_\text{C}}{T_\text{H} - T_\text{C}},
\end{equation}
它通常是大于 1 的。

\subsection{热力学第二定律}

\kwd{热力学第二定律}解决了有关过程\emph{方向性}的问题,它的主要建立者有
Carnot、Clausius、Kelvin 等。

\begin{theorem}[热力学第二定律(Kelvin 表述)]
  不可能从单一热源吸热使之完全变为有用的功,而不产生其他影响。即第二类永动机
  不可能实现。
\end{theorem}

\begin{theorem}[热力学第二定律(Clausius 表述)]
  不可能把热量从低温物体传到高温物体,而不产生其他影响。
\end{theorem}

热力学第二定律的 Kelvin 表述与 Clausius 表述是等价的。

\begin{proof} %TODO:20160318 图片
  \def\Clausius{(\text{Clausius 表述})} \def\Kelvin{(\text{Kelvin 表述})}
  \def\TH{T_\text{H}} \def\TC{T_\text{C}}
  \def\QH{Q_\text{H}} \def\QC{Q_\text{C}}

  首先证明 $\Kelvin\implies\Clausius$。我们采用反证法,即证明
  $\neg\,\Clausius \implies \neg\,\Kelvin$。

  设有一个 Carnot 热机工作于高温热源 $\TH$ 和低温热源 $\TC$ 之间。它从 $\TH$ 处
  吸收热量 $\QH$,向 $\TC$ 放出热量 $\QC$,并做功 $W=\QH-\QC$。假设 Clausius
  表述不成立,就可以在不产生其他影响的前提下,使低温热源获得的热量 $\QC$ 重新
  回到高温热源。净结果便是从单一热源 $\TH$ 吸收了热量 $\QH-\QC$,并将其完全转化
  为功,这就违背了 Kelvin 表述。因此原假设不成立,即有
  $\neg\,\Clausius \implies \neg\,\Kelvin$。

  接下来证明 $\Clausius\implies\Kelvin$。同样采用反证法,即证明
  $\neg\,\Kelvin \implies \neg\,\Clausius$。

  假设 Kelvin 表述不成立,就可以在不产生其他影响的前提下,从单一热源 $\TH$ 吸热
  $\QH$ 并将其完全转化为有用功 $W=\QH$。它可以推动 Carnot 制冷机从低温热源
  $\TC$ 吸收 $\QC$ 的热量,并传给高温热源 $\QH+\QC$ 的热量。净结果是热量 $\QC$
  从低温热源传给了高温热源,却没有产生其他影响,这就违背了 Clausius 表述。
  因此原假设不成立,即有
  $\neg\,\Kelvin \implies \neg\,\Clausius$。
\end{proof}

热力学第二定律的核心内容可以概括为:\emph{自然界一切热现象过程都是不可逆的。}

\section{热力学第二定律的数学解释;熵}

\subsection{Carnot 定理}

\begin{theorem}[Carnot 定理]
  工作于两个确定温度之间的所有热机中,可逆热机效率最高。
\end{theorem}

设两个热机 A、B 工作于高温热源 $\theta_\text{H}$ 和低温热源
$\theta_\text{C}$ 之间
\footnote{这里用 $\theta$ 表示温度,而没有使用 $T$,原因见下一小节。},
它们分别从 $\theta_\text{H}$ 吸收 $Q_\text{H,\,A}$ 与 $Q_\text{H,\,B}$ 的热量,
向 $\theta_\text{C}$ 放出 $Q_\text{C,\,A}$ 与 $Q_\text{C,\,B}$ 的热量,并对外
做功 $W_\text{A}$ 与 $W_\text{B}$
\footnote{前文用撇号表示系统(热机)对外界做功,这里方便起见直接用 $W$。
  但需注意,$W_\text{A}$ 与 $W_\text{B}$ 均大于 $0$。}。
根据定义,其效率分别为
\begin{equation}
  \eta_\text{A} = \frac{W_\text{A}}{Q_\text{H,\,A}} \qc
  \eta_\text{B} = \frac{W_\text{B}}{Q_\text{H,\,B}}.
\end{equation}
设 A 是一个可逆热机。因此不妨把 $\eta_\text{A}$ 写成 $\eta_\text{rev,\,A}$。
根据 Carnot 定理,有
\begin{equation}
  \eta_\text{rev,\,A} \geqslant \eta_\text{B}.
\end{equation}

\begin{proof}
  下面利用反证法证明 Carnot 定理,即假设 $\eta_\text{rev,\,A} < \eta_\text{B}$。
  因此有
  \begin{equation}
    \frac{W_\text{A}}{Q_\text{H,\,A}} < \frac{W_\text{B}}{Q_\text{H,\,B}}.
  \end{equation}
  令 A、B 从高温热源 $\theta_\text{H}$ 处吸收相同的热量,即
  $Q_\text{H,\,A} = Q_\text{H,\,B}$,那么就有 $W_\text{A} < W_\text{B}$。因为
  A 是可逆热机,所以不妨让 B 热机输出功的一部分 $W_\text{A}$ 推动 A 热机逆向
  运行(此时 A 就是一个制冷机)。此时,B 热机还可以输出功
  $W_\text{B} - W_\text{A}$。

  根据热力学第一定律,有
  \begin{braced}
    & W_\text{A} = Q_\text{H,\,A} - Q_\text{C,\,A}, \\
    & W_\text{B} = Q_\text{H,\,B} - Q_\text{C,\,B}.
  \end{braced}
  所以有
  \begin{equation}
    W_\text{B} - W_\text{A} = Q_\text{C,\,A} - Q_\text{C,\,B}.
  \end{equation}

  若 A、B 联合运行,其净结果便是从低温热源 $\theta_\text{C}$ 处吸收
  $Q_\text{C,\,A} - Q_\text{C,\,B}$ 的热量,并对外做 $W_\text{B} - W_\text{A}$
  的功,即在不产生其他影响的情况下完全把热转化为了功。这显然违背了热力学第二
  定律的 Kelvin 表述。因此原假设不成立,Carnot 定理得证。
\end{proof}

由 Carnot 定理,可以得到如下推论:

\begin{theorem}
  所有工作于两个确定温度之间的可逆热机,其效率均相等。
\end{theorem}

\subsection{热力学温标}

\kwd{温标},是以量化数值,配以温度单位来表示温度的方法。它包含三个要素:
\begin{enumerate}
  \item \emph{测温质}与\emph{测温参量};
  \item 测温参量与温度的\emph{函数关系};
  \item \emph{温度标准点}的选定。
\end{enumerate}

常用的经验温标有摄氏温标、华氏温标等。利用理想气体状态方程,可以定义%
\kwd{理想气体温标}:
\begin{equation}
  T \defeq \frac{1}{nR} \lim_{p \to 0} pV,
\end{equation}
同时需要规定水的三相点温度 $T_\text{tr} \defeq \SI{273.16}{\kelvin}$。

在 \S\ref{sec:ideal-gas-carnot-cycle-and-second-law}
第\ref{subsec:carnot-cycle}小节中,我们使用 $T$ 表示温度。实际上,那里的
“温度”正是用理想气体温标表示的值。

根据 Carnot 定理,可逆热机的效率只与两个热源的温度有关,而与工作物质的性质、
吸放热多少、做功多少均无关。换句话说,可逆热机的效率是两个温度
$\theta_\text{H}$、$\theta_\text{C}$ 的\emph{普适函数}。根据定义,热机的效率
\begin{equation}
  \eta = \frac{W}{Q_\text{H}} = 1 - \frac{Q_\text{C}}{Q_\text{H}}.
\end{equation}
因此有
\begin{equation}
  \frac{Q_\text{C}}{Q_\text{H}} = F\qty(\theta_\text{H}, \, \theta_\text{C}),
\end{equation}
其中的 $F\qty(\theta_\text{H}, \, \theta_\text{C})$ 是 $\theta_\text{H}$ 与
$\theta_\text{C}$ 的普适函数。

下面证明
\begin{equation}
  F(\theta_\text{H}, \, \theta_\text{C})
  = \frac{f\qty(\theta_\text{C})}{f(\theta_\text{H})},
\end{equation}
其中的 $f$ 是另一个普适函数。%TODO:20160322 热力学温标证明

由式~\eqref{eq:efficiency-of-carnot-cycle-for-ideal-gas},以理想气体为工作物质的
Carnot 热机效率为
\begin{equation}
  \eta = 1 - \frac{T^*_\text{C}}{T^*_\text{H}},
\end{equation}
这里用 $T^*$ 表示理想气体温标下的温度。这与式 是相同的,即温度尺度相同。
又因为理想气体温标也规定在水的三相点处 $T^*_\text{tr} = \SI{273.16}{\kelvin}$,
因此,理想气体温标与热力学温标是相同的。

\subsection{Clausius 不等式}

根据 Carnot 定理,工作于两个确定温度之间的所有热机,其效率均满足
\footnote{以后均直接用 $T$ 表示温度。}
\begin{equation}
  \eta = 1 - \frac{Q_2}{Q_1} \leqslant 1 - \frac{T_2}{T_1}.
\end{equation}
对于可逆热机,取等号;对于不可逆热机,则取小于号。

上式稍作变形,可得
\begin{equation}
  \frac{Q_2}{Q_1} \geqslant \frac{T_2}{T_1}
  \implies \frac{Q_1}{T_1} - \frac{Q_2}{T_2} \leqslant 0.
\end{equation}
约定 $Q$ 始终表示吸收的热量,则放热应写作 $-Q$。于是
\begin{equation}
  \frac{Q_1}{T_1} + \frac{Q_2}{T_2} \leqslant 0.
\end{equation}
假设系统先后与温度分别为 $T_1, \, T_2 \, \dots, \, T_n$ 的 $n$ 个热源接触,
又分别吸热 $Q_1, \, Q_2 \, \dots, \, Q_n$,则可以证明 \kwd{Clausius 不等式}:
\begin{equation}
  \sum_{i=1}^n \frac{Q_i}{T_i} \leqslant 0.
\end{equation}

\begin{proof}
  设有 %TODO:20160322 证明过程没写
\end{proof}

在 $n\to\infty$ 的极限下,Clausius 不等式过渡到积分形式:
\begin{equation} \label{eq:clausius-inequality-integral-form}
  \lim_{n\to\infty} \sum_{i=1}^n \frac{Q_i}{T_i} \leqslant 0
  = \oint \frac{\dbar Q}{T} \leqslant 0.
\end{equation}

\subsection{熵的定义}

对于可逆循环,根据式~\eqref{eq:clausius-inequality-integral-form},有
\begin{equation}
  \oint \frac{\dbar{Q_\text{rev}}}{T} = 0.
\end{equation}
如图,%TODO:20160323 图片
可以表示成两段路径之和:
\begin{equation}
  \underset{C_1\phantom{M}}{\int_{(P_0)}^{(P)}} \, \frac{\dbar{Q_\text{rev}}}{T}
  + \underset{C_2\phantom{M}}{\int_{(P)}^{(P_0)}} \, \frac{\dbar{Q_\text{rev}}}{T} = 0,
\end{equation}
即
\begin{equation}
  \underset{C_1\phantom{M}}{\int_{(P_0)}^{(P)}} \, \frac{\dbar{Q_\text{rev}}}{T}
  = \underset{C_2\phantom{M}}{\int_{(P_0)}^{(P)}} \, \frac{\dbar{Q_\text{rev}}}{T} = \const
\end{equation}
可以看出,$\dbar{Q_\text{rev}}/T$ 是一个与路径无关的量。由此,定义一个新的
状态函数——\kwd{熵}:
\begin{equation}
  S - S_0 = \int_{(P_0)}^{(P)} \, \frac{\dbar{Q_\text{rev}}}{T}.
\end{equation}

\subsection{不可逆过程的数学表述}

如果初、终态均是平衡态,根据 Clausius 不等式,我们有
\begin{equation}
  \underset{\text{irrev}+\text{rev}}{\oint} \frac{\dbar{Q}}{T}
  = \int_{(P_0)}^{(P)} \frac{\dbar{Q_\text{irrev}}}{T}
    + \int_{(P)}^{(P_0)} \frac{\dbar{Q_\text{rev}}}{T} < 0
  \implies S - S_0 > \int_{(P_0)}^{(P)} \frac{\dbar{Q_\text{irrev}}}{T}.
\end{equation}

如果初、终态均是非平衡态,采用\emph{局域平衡近似},仍旧可以推得
\begin{equation}
  S - S_0 > \int_{(P_0)}^{(P)} \, \frac{\dbar{Q_\text{irrev}}}{T}.
\end{equation} %TODO:20160323 局域平衡近似的证明

把对可逆过程与不可逆过程的表述合起来,就有
\begin{equation} \label{eq:secone-law-integral-form}
  \incr S = S - S_0 \geqslant \int_{(P_0)}^{(P)} \, \frac{\dbar{Q}}{T};
\end{equation}
写成微分形式,为
\begin{equation} \label{eq:secone-law-differential-form}
  \boxed{\dd{S} \geqslant \frac{\dbar{Q}}{T}}
\end{equation}
以上两式中,“$=$”适用于可逆过程,“$>$”适用于不可逆过程。这两式实际上便是热力学第二定律的数学表述。

\subsection{熵的性质}

这里小结一下熵的性质。

\begin{itemize}
  \item 熵是\emph{状态函数}。
  \item 熵是\emph{广延量}。
  \item 对微小的\emph{可逆}过程,$\dd{S} = \dbar{Q} / T$。因此有
  \begin{equation} \label{eq:dQ-TdS}
    \dbar{Q} = T \dd{S}.
  \end{equation}
  对于绝热过程,有 $\dbar{Q} = 0$,因此有
  \begin{equation} \label{eq:dS-for-adiabatic-reversible-process}
    \dd{S} = 0.
  \end{equation}
\end{itemize} %TODO:20160323 卡诺循环的 TS 表述

\subsection{热力学基本方程}

热力学第一定律式~\eqref{eq:first-law-differential-form}:
\begin{equation}
  \dd{U} = \dbar{Q} + \dbar{W};
\end{equation}
由热力学第二定律,得可逆过程微热量的表达式 \eqref{eq:dQ-TdS}:
\begin{equation}
  \dbar{Q} = T \dd{S};
\end{equation}
根据式~\eqref{eq:work-general},又有微功的一般表示:
\begin{equation}
  \dbar{W} = \sum_{i=1}^{n} Y_i \dd{y_i}.
\end{equation}
联立以上三式,可得
\begin{equation} \label{eq:fundamental-equation-of-thermodynamics}
  \dd{U} = T \dd{S} + \sum_{i=1}^{n} Y_i \dd{y_i}.
\end{equation}
这就是\kwd{热力学基本微分方程}。

对于 $p$-$\!V\!$-$\!T$ 系统,上式可简化为
\begin{equation} \label{eq:fundamental-equation-for-PVT-system}
  \dd{U} = T \dd{S} - p \dd{V}.
\end{equation}

\begin{example}[理想气体的熵]
  下面推导不同过程下理想气体的熵。

  首先考虑等容过程。根据式~\eqref{eq:dU-for-ideal-gas},有
  \begin{equation}
    \dd{U} = C_V \dd{T}.
  \end{equation}
  根据热力学基本微分方程式~\eqref{eq:fundamental-equation-for-PVT-system},
  \begin{equation}
    T \dd{S} = \dd{U} + p \dd{V}
    \implies \dd{S} = \frac{\dd{U}}{T} + \frac{p \dd{V}}{T}
                    = C_V \frac{\dd{T}}{T} + nR \frac{\dd{V}}{V}.
  \end{equation} %TODO:20160323 有问题?
  积分可得
  \begin{align}
    S &= \int C_V \frac{\dd{T}}{T} + nR \ln V + S_0 \\
  \intertext{若 $C_V$ 为常数,则有}
      &= C_V \ln T + nR \ln V + S_0.
  \end{align}
\end{example}

\section{熵增加原理;最大功} \label{sec:principle-of-the-increase-of-entropy}

\subsection{熵增加原理}

%    根据热力学第二定律[式~\eqref{EQ_2ND_LAW_IN_INTEGRAL_FORM}],
%    \begin{equation}
%      \incr S \geqslant \int_{\text{I}}^{\text{II}} \frac{\dbar Q}{T} \fullstop
%    \end{equation}
%    对于\emph{绝热}过程(或孤立体系),有 $\dbar Q = 0$。因此
%    \begin{equation} \label{EQ_PRINCIPLE_OF_ENTROPY_INCREASE}
%      \incr S \geqslant 0 \fullstop \footnote{
%        注意与 \eqref{eq:dS-for-adiabatic-reversible-process} 式对比,它还要求\kwd{可逆}过程。
%      }
%    \end{equation}
%    这就是\kwd{熵增加原理},它说明绝热体系的熵永不减少。
%  \subsection{不可逆过程的熵变}
%    没写%TODO:20160323 没写
%  \subsection{最大功}
%    根据热力学第一定律[式~\eqref{EQ_1ST_LAW_IN_DIFFERENTIAL_FORM}],
%    \begin{equation}
%      \dd U = \dbar Q + \dbar W \fullstop
%    \end{equation}
%    令 $\dbar W' = - \dbar W$ 为系统对外界做的功,则
%    \begin{equation} \label{EQ_dU=dQ-dW'_IN_SECTION_MAX_WORK}
%      \dbar W' = \dbar Q - \dd U \fullstop
%    \end{equation}
%    根据热力学第二定律[式~\eqref{EQ_2ND_LAW_IN_DIFFERENTIAL_FORM}],
%    \begin{equation}
%      \dbar Q \leqslant T_\text{e} \dd S \fullstop
%    \end{equation}
%    代入式~\eqref{EQ_dU=dQ-dW'_IN_SECTION_MAX_WORK},可得
%    \begin{equation}
%      \dbar W' \leqslant T_\text{e} \dd S - \dd U \fullstop
%    \end{equation}
%    因此系统对外做的\kwd{最大功}为
%    \begin{equation}
%      \dbar W'_{\text{max}} = \dbar W'_{\text{rev}} = T \dd S - \dd U \comma
%    \end{equation}
%    这里的 $T = T_\text{e}$ 为系统的温度(因为是可逆过程)。
%    
%    对于不可逆过程,显然有
%    \begin{equation}
%      \dbar W'_{\text{irrev}} < \dbar W'_{\text{rev}} \fullstop
%    \end{equation}
%    
%    \begin{example}[水的混合]
%      两杯等量的水初始温度分别为 $T_1$、$T_2$。在等压、绝热条件下将它们混合均匀,求该过程的熵变。%TODO:20160323 没写
%      %TODO:20160330 T取平均值:假设热容为常数
%    \end{example}
%    
%    \begin{example}[制冷机所需的最小功]
%      两物体初始温度均为 $T_1$。一台制冷机工作于其间,使一物体温度升高至 $T_2$。假设这是一个等压过程,并且不考虑相变。证明:制冷机所需的最小功
%      \begin{equation}
%        W_{\text{min}} = C_p \left( \frac{T_1^2}{T_2} + T_2 - 2 T_1 \right) \fullstop%TODO:20160323 没写
%      \end{equation}
%    \end{example}
%  
%\section{自由能与Gibbs函数}
%%	根据熵增加原理[式~\eqref{EQ_PRINCIPLE_OF_ENTROPY_INCREASE}],对于孤立系统,有
%%	\begin{equation}
%%		\incr S \geqslant 0 \fullstop
%%	\end{equation}
%  \subsection{自由能}
%    考虑这样的\emph{等温过程}:热源维持恒定温度 $T$;系统初终态温度 $T_1$、$T_2$ 与热源温度相同,即 $T_1 = T_2 =T$。对于可逆过程,在全程中系统温度均为 $T$;%而对于不可逆过程,仅满足 $T_1 = T_2 =T$。
%    
%    由Clausius不等式%[\eqref{}]
%    \begin{equation}
%      \incr S = S_2 - S_1 \geqslant \int_{\text{I}}^{\text{II}} \frac{\dbar Q}{T} = \frac{1}{T} \int_{\text{I}}^{\text{II}} \dbar Q = \frac{Q}{T}\comma 
%    \end{equation}
%    即
%    \begin{equation}
%      Q \leqslant T (S_2 - S_1) \fullstop
%    \end{equation}
%    根据热力学第一定律[式~\eqref{EQ_1ST_LAW_IN_INTEGRAL_FORM}],
%    \begin{equation}
%      U_2 - U_1 = W + Q \comma
%    \end{equation}
%    因此
%    \begin{align}
%      -W &= (U_1 - U_2) + Q \notag \\
%      &\leqslant (U_1 - U_2) - T (S_2 - S_1) \notag \\
%      &=(U_1 - T S_1) - (U_2 - T S_2) \fullstop
%    \end{align}
%    定义\kwd{自由能} $F = U - T S$,则
%    \begin{equation}
%      -W \leqslant F_1 - F_2 \fullstop
%    \end{equation}
%    
%    如果该过程除了保持等温,还保持等容,即 $W = 0$,则有%TODO:20160325 怎么会有等温等容?
%    \begin{equation}
%      \incr F = F_2 - F_1 \leqslant 0 \comma
%    \end{equation}
%    这说明在等温等容过程中,系统向自由能减小的方向前进。
%    
%    自由能具有以下的性质:
%    \begin{myEnum2}
%      \item 态函数 %TODO:没写
%    \end{myEnum2}
%  \subsection{Gibbs函数}
%    考虑等温等压过程