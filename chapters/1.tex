\section{平衡态及其描述} \label{sec:平衡态及其描述}
\subsection{热力学系统}
热力学研究的对象是\emphA{热力学系统}(简称系统)。它是宏观体系,粒子数的量级至少
约 $10^{20}$。与系统相对应的是\emphA{外界},也称为环境或热库。按照与外界的关系,
可将系统分为三种:\emphB{孤立系}、\emphB{封闭系}和\emphB{开放系},
见表~\ref{TAB_DEFINITION_OF_SYSTEMS}。

\begin{myTable}{cccc}{三种热力学体系}{TAB_DEFINITION_OF_SYSTEMS}
  \toprule
  & \emphA{孤立系} & \emphA{封闭系} & \emphA{开放系} \\%HACK:20160330 表格首行加粗
  \midrule
  物质交换 & \xmark & \xmark & \cmark \\
  能量交换 & \xmark & \cmark & \cmark \\
  \bottomrule
\end{myTable}

  \subsection{平衡态}
    传统热力学研究\emphA{平衡态}。它有两个要素:状态不随时间变化、孤立系。若非孤立系,则称\emphB{稳恒状态},如日光灯。
    
    平衡态只是宏观性质不随时间变化,而微观态仍有变化(微观粒子不断变化)。因此称为\emphA{动态平衡}(就微观状态而言,也称为\emphB{细致平衡})。
    
  \subsection{平衡态的描述} \label{subsec:平衡态的描述}
    平衡态可以利用\emphA{状态变量}(可以测量)来描述,如 $p$、$V$、$T$ 等。需注意,这些量未必互相独立。因此需要选取\emphB{独立}的\emphA{状态参量}。
    
    状态变量可分为两种:\emphA{强度量},如 $p$、$T$、$\r$ 等,它们不随粒子数的增加而增加;\emphA{广延量},如 $V$、$U$、$S$ 等,它们随粒子数的增加而增加。
    
    当某一状态变量可以用其他状态参量来描述时,则称其为一个\emphA{状态函数}。
    
\section{温度;状态方程}
  \subsection{热平衡定律}
    \begin{myThm}{热平衡定律(热力学第零定律)}
      若物体 $A$ 分别与物体 $B$ 和 $C$ 处于热平衡,那么如果让 $B$ 与 $C$ 热接触,它们一定也处于热平衡。
    \end{myThm}
    
    该定律是温度测量的基础:互为热平衡的物体存在一个属于其固有属性的物理量,即\emphA{温度}。一切互为热平衡的物体温度相等。
    
    具体确定温度,需要选定\emphA{温标}。除了常见的摄氏、华氏温标,还有热膨胀温标、热电阻温标、理想气体温标和热力学温标等。
    
  \subsection{物态方程}
    温度与其他状态参量\footnote{
      即独立的状态变量。
    }的函数关系称为\emphA{物态方程}:
    \begin{equation}
      T = f(p, V)
    \end{equation}
    或
    \begin{equation}
      g(p, V, T) = 0 \fullstop
    \end{equation}
    
    Boyle和Mariotte分别于1662年和1676年各自确立了\emphA{Boyle–Mariotte定律}:
    \begin{equation}
      p V = p_0 V_0, \quad \text{若 $T = \const \semicomma$}
    \end{equation}
    1802年,Gay-Lussac确立了\emphA{Gay-Lussac定律}\footnote{
      事实上,在1787年,Charles就已经发现了这一定律,只是当时未发表,也未被人注意。
    }:
    \begin{equation}
      \frac{V}{T} = \frac{V_0}{T_0}, \quad \text{若 $p = \const \fullstop$}
    \end{equation}
    综合两式,可得
    \begin{equation} \label{EQ_IDEAL_GAS_LAW_A}
      \frac{p V}{T} = \frac{p_0 V_0}{T_0} = \text{``const.''} = n R \comma \footnote{
        该式中常数的值显然与物质的量 $n$ 有关,故用带引号的 $\text{``const.''}$ 表示。
      }
    \end{equation}
    即\emphA{理想气体物态方程}
    \begin{boxedEq} \label{EQ_IDEAL_GAS_LAW_B}
      p V = n R T = N \kB T
    \end{boxedEq}
    
    \begin{myProof}
      式~\eqref{EQ_IDEAL_GAS_LAW_A} 推导如下。设某气体初状态为 $p_0$、$V_0$、$T_0$,末状态为 $p$、$V$、$T$。
      
      先保持温度 $T_0$ 不变,则有
      \begin{equation} \label{EQ_PROOF_OF_IDEAL_GAS_LAW_1}
        p_0 V_0 = p V' \semicomma
      \end{equation}
      再假设压强 $p$ 不变,于是
      \begin{equation}
        \frac{V'}{T_0} = \frac{V}{T} \comma
      \end{equation}
      即
      \begin{equation}
        V' = \frac{V T_0}{T} \fullstop
      \end{equation}
      代入 \eqref{EQ_PROOF_OF_IDEAL_GAS_LAW_1}~式,即有
      \begin{equation}
        \frac{p V}{T} = \frac{p_0 V_0}{T_0} = \text{``const.''} \fullstop
      \end{equation}
    \end{myProof}
    
    \blankline
    
    除了通过统计物理推导,也可通过测量\emphB{膨胀系数}、\emphB{压强系数}和\emphB{等温压缩系数}(亦可简称\emphB{压缩系数})来得到物态方程。
    
    膨胀系数 $\a$ 定义为
    \begin{equation} \label{EQ_DEFINITION_OF_EXPANSION_COEFFICIENT}
      \a \eqdef \frac{1}{V} \left( \frac{\pd V}{\pd T} \right)_p \comma
    \end{equation}
    压强系数 $\b$ 定义为
    \begin{equation}
      \b \eqdef \frac{1}{p} \left( \frac{\pd p}{\pd T} \right)_V \comma
    \end{equation}
    等温压缩系数 $\k_T$ 定义为
    \begin{equation}
      \k_T \eqdef -\frac{1}{V} \left( \frac{\pd V}{\pd p} \right)_T \fullstop
    \end{equation}
    
    可以证明,以上三个常数满足下面的关系:
    \begin{equation} \label{EQ_RELATION_OF_ALPHA_BETA_KAPPA}
      \a = \b \k_T p \fullstop
    \end{equation}
    该式说明三者并非独立。通常会直接测量 $\a$ 和 $\k_T$,而通过计算得到 $\b$。
    
    \begin{myProof}
      先证明两个结论。若 $x, \, y, \, z$ 满足 $F(x, y, z) = \const$,其中 $F(x, y, z)$ 是一个可微函数。那么
      \begin{mySubEq}
        \begin{align}
          \dd x &= \left( \frac{\pd x}{\pd y} \right)_z \dd y + \left( \frac{\pd x}{\pd z} \right)_y \dd z \comma \\
          \dd z &= \left( \frac{\pd z}{\pd x} \right)_y \dd x + \left( \frac{\pd z}{\pd y} \right)_x \dd y \fullstop
        \end{align}
      \end{mySubEq}
      把第一式代入第二式,得
      \begin{equation}
        \begin{aligned}
          \dd z &= \left( \frac{\pd z}{\pd x} \right)_y \left[ \left( \frac{\pd x}{\pd y} \right)_z \dd y + \left( \frac{\pd x}{\pd z} \right)_y \dd z \right] + \left( \frac{\pd z}{\pd y} \right)_x \dd y \\
          &= \left[ \left( \frac{\pd z}{\pd x} \right)_y \left( \frac{\pd x}{\pd y} \right)_z + \left( \frac{\pd z}{\pd y} \right)_x \right] \dd y
          + \left( \frac{\pd z}{\pd x} \right)_y \left( \frac{\pd x}{\pd z} \right)_y \dd z \fullstop
        \end{aligned}
      \end{equation}
      于是
      \begin{equation} \label{EQ_***_dz_EQUAL_***_dy}
         \left[ 1 - \left( \frac{\pd z}{\pd x} \right)_y \left( \frac{\pd x}{\pd z} \right)_y \right] \dd z
        = \left[ \left( \frac{\pd z}{\pd x} \right)_y \left( \frac{\pd x}{\pd y} \right)_z + \left( \frac{\pd z}{\pd y} \right)_x \right] \dd y \fullstop
      \end{equation}
      因为 $y$ 与 $z$ 是独立的变量,所以 $\dd y$ 与 $\dd z$ 也是独立的。这就要求上式左右两边括号内的项均为零。由左边括号内的项,可得到\emphA{倒数关系}:
      \begin{equation}
        \left( \frac{\pd z}{\pd x} \right)_y \left( \frac{\pd x}{\pd z} \right)_y = 1 \fullstop
      \end{equation}
      类似地,还有
      \begin{equation} \label{EQ_RECIPROCITY_RELATION}
        \left( \frac{\pd z}{\pd y} \right)_x \left( \frac{\pd y}{\pd z} \right)_x = 1 \fullstop
      \end{equation}
      根据 \eqref{EQ_***_dz_EQUAL_***_dy}~式的右侧,有
      \begin{equation} \label{EQ_CYCLIC_RELATION}
        \left( \frac{\pd z}{\pd x} \right)_y \left( \frac{\pd x}{\pd y} \right)_z = -\left( \frac{\pd z}{\pd y} \right)_x \fullstop
      \end{equation}
      代入式~\eqref{EQ_RECIPROCITY_RELATION},便得到\emphA{偏导数三乘积法则}:
      \begin{equation}
        \left( \frac{\pd x}{\pd y} \right)_z \left( \frac{\pd y}{\pd z} \right)_x \left( \frac{\pd z}{\pd x} \right)_y = -1 \fullstop
      \end{equation}
      
      \blankline
      
      然后我们利用式~\eqref{EQ_CYCLIC_RELATION},用热力学中的变量将其写成
      \begin{equation}
        \left( \frac{\pd V}{\pd T} \right)_p = -\left( \frac{\pd V}{\pd p} \right)_T \left( \frac{\pd p}{\pd T} \right)_V \fullstop
      \end{equation}
      代入前文中的定义,就得到了 \eqref{EQ_RELATION_OF_ALPHA_BETA_KAPPA}~式:
      \begin{equation}
        \a = \b \k_T p \fullstop
      \end{equation}
    \end{myProof} 
    
    \blankline
    
    考虑分子之间的相互作用后,对理想气体进行修正,这就是\emphA{van der Waals气体}。其物态方程为
    \begin{equation} \label{EQ_VAN_DER_WAALS_GAS_STATE_EQUATION}
      \left( p + \frac{n^2 a}{V^2} \right) (V - n b) = n R T \comma
    \end{equation}
    其中 $n^2 a / V^2$ 代表分子之间吸引力所引起的修正,而 $n b$ 则代表排斥力所引起的修正。
    
\section{功;热力学第一定律}
  \subsection{准静态过程的功}
    静态过程即平衡态。
    
    \emphA{准静态过程}指过程中的每一步都是平衡态。这就要求外界的条件变化地足够缓慢。令 $\t$ 为外界条件变化的特征时间,$\incr t$ 为系统趋于与外界条件对应的平衡态的特征时间(即\emphA{弛豫时间}),那么准静态过程相当于
    \begin{equation}
      \frac{\t}{\incr t} \approach 0
    \end{equation}
    的极限。
    
    一个过程中,若每一步都可以在相反的方向进行而不引起外界的变化,则称为\emphA{可逆过程}。可逆过程的实质是没有耗散。
    
    下面举两个准静态过程的例子。
    
    \begin{myEnum1}
      \myItem{$p V$ 系统}
        考虑盛于带活塞容器内的气体。气体体积变化 $\dd V$ 时,\emphB{外界对系统}做功
        \begin{equation}
          \db W = -p \dd V \fullstop
        \end{equation}
        这里的 $p$ 指外界对系统的压强。由于是准静态过程,它又等于容器内气体对器壁的压强。外界压力作用下体积减小,$\dd V < 0$;而外界却对系统做正功,因此会出现负号。如果令 $W'$ 为\emphB{系统对外界}做的功,则有
        \begin{equation}
          \db W' = - \dd W = p \dd V \fullstop
        \end{equation}
        
        体积由 $V_1$ 变化到 $V_2$,外界对系统做的总功
        \begin{equation}
          W = -\int_{V_1}^{V_2} p \dd V \fullstop
        \end{equation}
        
        \begin{figure}[h]
          \centering
          
          \begin{asy}
            pair O = (0, 0), x_axes = (10, 0), y_axes = (0, 10);
            
            pair p1 = (2.2, 7.5), p2 = (8, 2.5), p3 = (4, 4.5);
            pair p1_x = (p1.x, 0), p2_x = (p2.x, 0);
            
            fill(p1_x--p1..p3..p2--p2_x--cycle, color1 + opacity(0.2));
            
            draw(Label("$V$", EndPoint), O--x_axes, Arrow);
            draw(Label("$p$", EndPoint), O--y_axes, Arrow);
            
            draw(p1..p3..p2, linewidth(1) + color1);
            draw(p1_x--p1, dashed + color1);
            draw(p2_x--p2, dashed + color1);
            
            label("$O$", O, SW);
            label("状态 $1$", p1, (0, 2));
            label("状态 $2$", p2, (0.5, 2));
            label("$V_1$", p1_x, S);
            label("$V_2$", p2_x, S);
          \end{asy}
          \caption{$p \text{-} V$ 系统的状态空间}
          \label{FIG_pV_DIAGRAM}
        \end{figure}
        
        在 $p \text{-} V$ 图中标出状态1和状态2,则连接它们的曲线就表示一个准静态过程,曲线下的面积就等于 $-W$,如图~\ref{FIG_pV_DIAGRAM} 所示。显然,$W$ 与路径有关,即不是一个全微分,因此用“$\db W$”表示微功。
        
      \myItem{电介质极化系统}
        
        考虑均匀电场 $\vecb{E}$ 中的均匀电介质。电位移 $\vecb{D}$ 变化 $\dd \vecb{D}$ 时,电场做功
        \begin{equation}
          \db W = V \vecb{E} \dotTimes \dd \vecb{D} \comma
        \end{equation}
        其中的 $V$ 表示电介质的体积。这里,$\vecb{E}$ 是变化的外因,$\vecb{D}$ 则是内果。令 $\vecb{P}$ 为极化强度,则
        \begin{equation}
          \vecb{D} = \pfs \vecb{E} + \vecb{P} \fullstop
        \end{equation}
        于是
        \begin{equation}
          \begin{aligned}
            \db W &= V \pfs \vecb{E} \dotTimes \dd \vecb{E} + V \vecb{E} \dotTimes \dd \vecb{P} \\
            &= V \dd \left( \frac{1}{2} \pfs \vecb{E}^2 \right) + V 
            \vecb{E} \dotTimes \dd \vecb{P} \fullstop
          \end{aligned}
        \end{equation}
        式中的第一项代表电场能量的变化,第二项则代表\emphB{极化功}。
    \end{myEnum1}
    
    对以上两种情况进行推广,可得
    \begin{align} \label{EQ_GENERAL_WORK}
      \db W &= Y_1 \dd y_1 + Y_2 \dd y_2 + \cdots Y_n \dd y_n \notag \\
      &= \iTonSum Y_i \dd y_i \fullstop
    \end{align}
    这里的 $Y_i$ 是\emphA{广义力},如压强 $p$、电场强度 $\vecb{E}$、磁场强度 $\vecb{H}$ 等;$y_i$ 是\emphA{广义坐标},如体积 $V$、电位移 $\vecb{D}$、磁感应强度 $\vecb{B}$ 等。
    
  \subsection{非静态过程}
    \begin{myEnum1}
      \myItem{等容过程}
        由于体积不变,$\dd V = 0$,因此
        \begin{equation} \label{EQ_WORK_IN_ISOCHORIC_PROCESS}
          \db W = 0 \fullstop
        \end{equation}
        
      \myItem{等压过程}
        \emphB{外界}压强不变,即 $p_{\text{ext}} = \const$,因此体积从 $V_1$ 变化到 $V_2$ 时外界对系统做功
        \begin{equation}
          W = -p_{\text{ext}} (V_2 - V_1) = -p_{\text{ext}} \incr V \fullstop
        \end{equation}
        若系统初、终态压强相等都处于平衡态,有
        \begin{equation}
          p_1 = p_2 = p_{\text{ext}} \eqdef p \fullstop
        \end{equation}
        于是功
        \begin{equation} \label{EQ_WORK_IN_ISOBARIC_PROCESS}
          W = -p (V_2 - V_1) = -p \incr V \fullstop
        \end{equation}
    \end{myEnum1}
    
  \subsection{热力学第一定律}
    \emphA{热力学第一定律}的主要建立者有Mayer、Joule、Helmholtz、Carnot等。它描述了\emphB{功}、\emphB{热量}、\emphB{内能}三者之间的关系,是\emphA{能量守恒定律}在宏观热现象过程中的表现形式。
    \begin{myThm}{能量守恒定律}
      自然界中的一切物质都具有能量。能量有各种不同的形式,能够从一种形式转化为另一种形式,从一个物体传递给另一个物体。在转化和传递中,能量的总量不变。
    \end{myThm}
    
    \begin{myEnum1}
      \myItem{绝热过程}
        从状态1变化到状态2的过程中,只有外界做功,因此
        \begin{equation}
          U_2 - U_1 = W_\text{a} \comma
        \end{equation}
        其中的 $W_\text{a}$ 表示绝热功。
        
      \myItem{非绝热过程}
      由于既有外界做功,又有热量传递,因此
      \begin{equation} \label{EQ_1ST_LAW_IN_INTEGRAL_FORM}
        U_2 - U_1 = W + Q \fullstop
      \end{equation}
      写成微分形式,为
      \begin{boxedEq} \label{EQ_1ST_LAW_IN_DIFFERENTIAL_FORM}
        \dd U = \db W + \db Q
      \end{boxedEq}
    \end{myEnum1}
    这实际上就是热力学第一定律最常用的表述。
    
    设两个全同系统内能分别为 $U_1$、$U_2$。则总内能 $U_\text{total}$ 除了 $U_1 + U_2$,还需包括由于界面效应所导致的 $U_{12}$。但在热力学中,界面效应基本可以忽略,因此近似有
    \begin{equation}
      U_\text{total} = U_1 + U_2 \fullstop
    \end{equation}
    这就是说,内能也是一个广延量。
    
\section{热容与焓;理想气体的性质} \label{sec:热容与焓;理想气体的性质}
  \subsection{热容与焓} \label{subsec:热容与焓}
    对于过程 $y$,定义\emphA{热容(heat capacity)}
    \begin{equation}
      C_y \eqdef \frac{\db Q_y}{\dd T} \comma
    \end{equation}
    式中的 $\db Q_y$ 是温度升高 $\dd T$ 时系统所吸收的热量。
    
    对于\emphB{等容过程},有\emphA{定容热容}
    \begin{equation}
      C_V \eqdef \frac{\db Q_V}{\dd T} \fullstop
    \end{equation}
    根据式~\eqref{EQ_WORK_IN_ISOCHORIC_PROCESS},可知
    \begin{equation}
      \db Q_V = \dd U - \db W = \dd U - 0 = \dd U \comma
    \end{equation}
    因此
    \begin{equation} \label{EQ_HEAT_CAPACITY_IN_CONST_V}
      C_V = \left( \frac{\pd U}{\pd T} \right)_V \fullstop
    \end{equation}
    
    对于\emphB{等压过程},有\emphA{定压热容}
    \begin{equation}
      C_p \eqdef \frac{\db Q_p}{\dd T} \fullstop
    \end{equation}
    根据式~\eqref{EQ_WORK_IN_ISOBARIC_PROCESS},可知
    \begin{equation} \label{EQ_dQ_IN_CONST_p}
      \db Q_p = \dd U - \db W = \dd U + p \dd V \comma
    \end{equation}
    因此
    \begin{equation} \label{EQ_HEAT_CAPACITY_IN_CONST_p}
      C_p = \left( \frac{\pd U}{\pd T} \right)_p + p \left( \frac{\pd V}{\pd T} \right)_p \fullstop
    \end{equation}
    
    可以发现,
    \begin{equation}
      C_p = C_V + p \left( \frac{\pd V}{\pd T} \right)_p \fullstop
    \end{equation}
    这是它们的关系之一。\footnote{
      $\pd U / \pd T$ 的下标,在式~\eqref{EQ_HEAT_CAPACITY_IN_CONST_V} 中是 $V$,而在式~\eqref{EQ_HEAT_CAPACITY_IN_CONST_p} 中则是 $p$。%TODO:20160312 偏微分关系
    }%TODO:20160330 下标的更换是不对的,或者有其他要求
    
    \blankline
    
    引入一个新的物理量——\emphA{焓(enthalpy)},其定义为
    \begin{equation}
      H \eqdef U + p V \fullstop
    \end{equation}
    由于 $U$、$p$ 和 $V$ 均是状态函数,因此焓也是状态函数。利用焓,可将式~\eqref{EQ_HEAT_CAPACITY_IN_CONST_p} 改写为
    \begin{equation} \label{EQ_HEAT_CAPACITY_IN_CONST_p_WITH_H}
      C_p = \left( \frac{\pd H}{\pd T} \right)_p \comma
    \end{equation}
    同时将式~\eqref{EQ_dQ_IN_CONST_p} 改写为
    \begin{equation}
      \db Q_p = \dd H \fullstop
    \end{equation}
    这就是说,\emphB{在等压过程中,物体吸收的热量等于焓的增加量}。
    
    很明显,热容与焓均是广延量。单位质量的热容称为\emphA{比热容(specific heat capacity)}或\emphA{比热(specific heat)},它是强度量。
    
  \subsection{理想气体的性质} \label{subsec:理想气体的性质}
    \begin{figure}[h]
      \begin{asy}
        pair p1 = (0, 7), p2 = (0, 0), p3 = (12, 0), p4 = (p3.x, p1.y), p5 = (0, 6), p6 = (p3.x, p5.y);
        pair m1 = (p1+p4)/2, m2 = (p2+p3)/2;
        transform myReflect = reflect(m1, m2);
        
        real boxWidth = 3, boxHeight = 2;
        pair pA1 = (1.8, 2.5), pA2 = (pA1.x+boxWidth, pA1.y), pA3 = (pA2.x, pA1.y+boxHeight), pA4 = (pA1.x, pA3.y);
        path boxA = pA1--pA2--pA3--pA4--cycle;
        path boxB = myReflect*boxA;
        
        real pipeSize = 0.2, pipeHeight = 3.5;
        pair ppA1 = (pA4.x+(boxWidth-pipeSize)/2, pA4.y), ppA2 = (ppA1.x, ppA1.y+pipeHeight), ppA3 = (ppA1.x+pipeSize, ppA1.y), ppA4 = (ppA3.x, ppA2.y-pipeSize);
        pair ppB1 = myReflect*ppA1, ppB2 = myReflect*ppA2, ppB3 = myReflect*ppA3, ppB4 = myReflect*ppA4;
        path pipe = ppA1--ppA2--ppB2--ppB1--ppB3--ppB4--ppA4--ppA3--cycle;
        
        real valveHeight = 0.6;
        pair pv1 = (m1.x, ppA2.y-(valveHeight+pipeSize)/2), pv2 = (pv1.x, pv1.y+valveHeight);
        
        real thermometerHeight = 5, thermometerR = 0.2;
        pair pt1 = (1, 3), pt2 = (pt1.x, pt1.y+thermometerHeight);
        
        fill(p5--p2--p3--p6--cycle, color1+opacity(0.2));
        draw(p1--p2--p3--p4, linewidth(1)+color1);
        draw(boxA, linewidth(1)+color2);
        draw(boxB, linewidth(1)+color2);
        draw(pipe, linewidth(1)+color2);
        draw(pv1--pv2, linewidth(2)+color3);
        draw(pt1--pt2, linewidth(2)+color4);
        fill(circle(pt1, thermometerR), color4);
        
        label("A", (ppA1+ppA3)/2, (0, -3));
        label("B", (ppB1+ppB3)/2, (0, -3));
        
        pair pwLabel1 = (p3.x-1, p3.y+1.4), pwLabel2 = (p3.x+1, p3.y+2.5);
        draw(pwLabel1--pwLabel2);
        label("水槽", pwLabel2, E);
        
        pair ptLabel1 = (pt1.x-0.25, pt2.y-1), ptLabel2 = (pt1.x-1.5, pt2.y+0.4);
        draw(ptLabel1--ptLabel2);
        label("温度计", ptLabel2, N);
        
        pair pvLabel1 = (pv1.x+0.25, pv2.y-0.1), pvLabel2 = (pv1.x+1.5, pv2.y+1);
        draw(pvLabel1--pvLabel2);
        label("阀门", pvLabel2, N);
      \end{asy}
      \caption{Joule实验的装置}
      \label{FIG_JOULE_EXPERIMENT}
    \end{figure}
    
    先介绍Joule实验(1845年)。设有一容器,分为A、B两个相同的部分。将它们置于水槽内,并通过带阀门的细管相连,水温可以通过温度计测量,如图~\ref{FIG_JOULE_EXPERIMENT} 所示。
    
    首先,在A内充满气体,而使B保持真空。然后打开阀门,气体将\emphB{自由膨胀},并充满整个容器。Joule的实验结果是前后水温不变。
    
    由于是等容过程,因此 $W = 0$;水温不变,说明 $Q = 0$。根据热力学第一定律,有 $\incr U = 0$。假设内能是温度和体积的函数,即
    \begin{equation}
      U = U(T, \, V) \fullstop
    \end{equation}%TODO:20160312 此处推导有问题?
    根据\emphB{偏导数三乘积法则},可知
    \begin{equation}
      \left( \frac{\pd U}{\pd V} \right)_T \left( \frac{\pd V}{\pd T} \right)_U \left( \frac{\pd T}{\pd T} \right)_V = -1 \fullstop
    \end{equation}
    于是
    \begin{equation}
      \left( \frac{\pd U}{\pd V} \right)_T 
      = -\left( \frac{\pd U}{\pd T} \right)_V \left( \frac{\pd T}{\pd V} \right)_U 
      = 0 \fullstop
    \end{equation}
    这就说明 $U = U(T)$,即内能只和温度有关。
    
    不过,Joule实验过于粗糙,更精确的实验表明实际气体的内能不仅与 $T$ 有关,还与 $V$ 有关。但对于理想气体,以上结论仍然成立,因此理想气体具有如下两条性质:
    \begin{mySubEq}
      \begin{empheq}[left=\empheqlbrace]{align}
        & p V = n R T \comma \label{EQ_IDEAL_GAS_PROPERTY_STATE_EQUATION} \\
        & U = U(T) \label{EQ_IDEAL_GAS_PROPERTY_INTERNAL_ENERGY} \fullstop
      \end{empheq}
    \end{mySubEq}
    就目前而言,这两条性质彼此是独立的。但利用热力学第二定律或统计物理,可以由 \eqref{EQ_IDEAL_GAS_PROPERTY_STATE_EQUATION}~式 推导 \eqref{EQ_IDEAL_GAS_PROPERTY_INTERNAL_ENERGY}~式。[见\secref{sec:Maxwell关系}\subsecref{subsec:简单应用_OF_MAXWELL关系}中的\egref{EG_pd_U/pd_V_WITH_FIXED_T}]%FIXME:20160401 后续交叉引用
    
    对于理想气体,根据焓的定义,可得
    \begin{equation}
      H = U + p V = U(T) + n R T = H(T) \comma
    \end{equation}
    即焓也仅是温度的函数。从而根据式~\eqref{EQ_HEAT_CAPACITY_IN_CONST_V} 和 \eqref{EQ_HEAT_CAPACITY_IN_CONST_p_WITH_H},又有
    \begin{align}
      C_p - C_V 
      &= \left( \frac{\pd H}{\pd T} \right)_p - \left( \frac{\pd U}{\pd T} \right)_V \notag \\
      &= \frac{\dd H}{\dd T} - \frac{\dd U}{\dd T} \notag \\
      &= \frac{\dd \; (p V)}{\dd T} 
      = \frac{\dd \; (n R T)}{\dd T} 
      = n R \fullstop \label{EQ_C_p-C_V_FOR_IDEAL_GAS}
    \end{align}
    
    定义\emphA{热容比(heat capacity ratio)} $\g$ \footnote{
      也称为\emphA{绝热指数(adiabatic index)},原因见\subsecref{subsec:绝热过程的过程方程}。
    }
    为 $C_p$ 与 $C_V$ 之比:
    \begin{equation} \label{EQ_DEF_OF_HEAT_CAPACITY_RATIO}
      \g \eqdef \frac{C_p}{C_V} = \g(T) \fullstop
    \end{equation}
    这里 $\g = \g(T)$ 是显而易见的。
    
    由式~\eqref{EQ_C_p-C_V_FOR_IDEAL_GAS} 和 \eqref{EQ_DEF_OF_HEAT_CAPACITY_RATIO},可以解得
    \begin{mySubEq}
      \begin{empheq}[left=\empheqlbrace]{align}
        & C_V = \frac{1}{\g - 1} n R \comma \label{EQ_C_V_BY_HEAT_CAPACITY_RATIO}\\
        & C_p = \frac{\g}{\g - 1} n R \fullstop \label{EQ_C_p_BY_HEAT_CAPACITY_RATIO}
      \end{empheq}
    \end{mySubEq}
    因为 $\g$ 可以通过实验测量,由此算出热容后,就可以确定理想气体的内能与焓:
    \begin{mySubEq}
      \begin{empheq}[left=\empheqlbrace]{align}
        & U(T) = \int C_V(T) \dd T + U_0 \comma \\
        & H(T) = \int C_p(T) \dd T + H_0 \comma
      \end{empheq}
    \end{mySubEq}
    其中的 $U_0$ 和 $H_0$ 是积分常数。写成微分形式,为
    \begin{mySubEq}
      \begin{empheq}[left=\empheqlbrace]{align}
      & \dd U = C_V(T) \dd T \comma \label{EQ_dU=Cv*dT_IDEAL_GAS} \\
      & \dd H = C_p(T) \dd T \fullstop \label{EQ_dH=Cp*dT_IDEAL_GAS}
      \end{empheq}
    \end{mySubEq}
    
  \subsection{绝热过程的过程方程} \label{subsec:绝热过程的过程方程}
    本节叙述均针对理想气体。
    
    根据理想气体状态方程,有
    \begin{equation}
      p V = n R T \fullstop
    \end{equation}
    于是
    \begin{equation}
      \dd T = \frac{p \dd V + V \dd p}{n R} \fullstop
    \end{equation}
    对于绝热过程,根据热力学第一定律,有
    \begin{align}
      \db Q &= \dd U + p \dd V \notag \\
      &= C_V \dd T + p \dd V \notag \\
      &= C_V \, \frac{p \dd V + V \dd p}{n R} + p \dd V \notag \\
      &= \frac{\g}{\g - 1} p \dd V + \frac{1}{\g - 1} V \dd p \myTag{见式~\eqref{EQ_C_V_BY_HEAT_CAPACITY_RATIO}} \\
      &= 0 \comma
    \end{align}
    因此
    \begin{align}
      &\mathrel{\phantom{\implies}} V \dd p + \g p \dd V = 0 \\
      &\implies \frac{\dd p}{p} + \g \frac{\dd V}{V} = 0 \notag \\
      &\implies \ln p + \g \ln V = \ln \left(p V^{\g} \right) = 0 \notag \\
      &\implies p V^{\g} = \label{EQ_STATE_EQUATION_OF_ADIABATIC_PROCESS_IN_p_V} \const \footnotemark
    \end{align} \footnotetext{这一步推导假定 $\g$ 为常数,因此可以直接积分。} \enlargethispage{\baselineskip}%HACK:禁止脚注分页
    改用其他变量,可把该式写成
    \begin{equation} \label{EQ_STATE_EQUATION_OF_ADIABATIC_PROCESS_IN_p_T}
      p^{(1 - \g) / \g} T = \const
    \end{equation}
    或
    \begin{equation} \label{EQ_STATE_EQUATION_OF_ADIABATIC_PROCESS_IN_T_V}
      T V^{\g - 1} = \const
    \end{equation}
    的形式。
    
    \begin{myExample}[海拔与气温的关系]
      下面推导\emphB{气温垂直递减率}。首先考虑\emphB{干燥空气}的温度\emphB{绝热}递减率。假设空气是理想气体。
      
      因为是绝热过程,因此
      \begin{equation}
        p^{(1 - \g) / \g} T = \const
      \end{equation}
      两边求微分,得
      \begin{equation}
        \frac{1 - \g}{\g} p^{\frac{1 - \g}{\g} - 1} T \dd p + p^{\frac{1 - \g}{\g} - 1} \dd T = 0 \comma
      \end{equation}
      即
      \begin{equation} \label{EQ_dT/dp_IN_EXAMPLE_OF_LAPSE_RATE}
        \frac{\dd T}{\dd p} = \frac{1 - \g}{\g} \frac{T}{p} \fullstop
      \end{equation}
      
      假设大气处于平衡状态,则有
      \begin{equation}
        \dd p = -\r g \dd z \comma
      \end{equation}
      其中的 $\r$ 是空气密度,
      \begin{equation}
        \r = \frac{m}{V} = \frac{n M}{n R T / p} = \frac{p M}{R T} \comma
      \end{equation}
      这里的 $M$ 是空气的平均摩尔质量, $g$ 是重力加速度,$m$、$V$ 分别是一定量空气的质量和体积。代入 \eqref{EQ_dT/dp_IN_EXAMPLE_OF_LAPSE_RATE} 式,可得
      \begin{equation}
        \frac{\dd T}{\dd z} = -\frac{1 - \g}{\g} \frac{\r g T}{p} = -\frac{1 - \g}{\g} \frac{M g}{R} \fullstop
      \end{equation}
      代入空气的热容比 $\g = 1.4$、平均摩尔质量 $M = \SI{28.8e-3}{\kg\per\mol}$ 等数值,可得
      \begin{equation}
        \frac{\dd T}{\dd z} = \SI{-9.7}{\kelvin\per\km} \fullstop
      \end{equation}
      
      \blankline
      
      若为水汽饱和的湿空气,有下面的近似公式:%TODO:20160315 参考文献
      \begin{equation} \label{EQ_SATURATED_ADIABATIC_LAPSE_RATE}
        \frac{\dd T}{\dd z}
        = -g \, \frac{1 + \dfrac{L_\text{vap} r}{R_\text{s,\,dry} T}}{c_{p,\,\text{dry}} + \dfrac{L_\text{vap}^2 r}{R_\text{s,\,water} T^2}}
        = -g \, \frac{1 + \dfrac{L_\text{vap} r}{R_\text{s,\,dry} T}}{c_{p,\,\text{dry}} + \dfrac{L_\text{vap}^2 r \e}{R_\text{s,\,dry} T^2}} \comma
      \end{equation}
      式中的各符号见表~\ref{TAB_SYMBOLS_IN_SATURATED_ADIABATIC_LAPSE_RATE}。%FIXME:20160401 qed位置
      
      \begin{myTable}{Mcc}{式~\eqref{EQ_SATURATED_ADIABATIC_LAPSE_RATE} 中所用到的符号}{TAB_SYMBOLS_IN_SATURATED_ADIABATIC_LAPSE_RATE}
        \toprule
        \text{\emphA{符号}} & \emphA{说明} & \emphA{数值} \\%HACK:20160330 表格首行加粗
        \midrule
        g & 重力加速度 & \SI{9.8076}{\metre\per\second\squared} \\
        L_\text{vap} & 水的汽化热 & \SI{2257}{\kilo\joule\per\kg} \\
        c_{p,\,\text{dry}} & 干燥空气的定压比热容 & \SI{1003.5}{\joule\per\kg\per\kelvin} \\
        R_\text{s,\,dry} & 干燥空气的气体常数 & \SI{287}{\joule\per\kg\per\kelvin} \\
        R_\text{s,\,water} & 水蒸气的气体常数 & \SI{461.5}{\joule\per\kg\per\kelvin} \\
        \e = R_\text{s,\,dry} / R_\text{s,\,water} & 干燥空气与水蒸气的气体常数之比 & 0.622 \\
        e & 饱和空气的水蒸气分压 & —— \\
        p & 饱和空气的气压 & —— \\
        r = \e e / (p - e) & 水蒸气的质量与干燥空气质量的混合比例 & —— \\
        T & 饱和空气的温度 & —— \\
        \bottomrule
      \end{myTable}
    \end{myExample}
    
\section{理想气体与Carnot循环;热力学第二定律} \label{sec:理想气体与Carnot循环_热力学第二定律}
  \subsection{Carnot循环} \label{subsec:Carnot循环}
    \emphA{Carnot循环}分为四个过程,如图~\ref{FIG_CARNOT_CYCLE} 所示。
    \begin{myEnum2}
      \item 等温膨胀:$(T_\text{H}, \, V_1) \rightarrow (T_\text{H}, \, V_2)$,
      \item 绝热膨胀:$(T_\text{H}, \, V_2) \rightarrow (T_\text{C}, \, V_3)$,
      \item 等温压缩:$(T_\text{C}, \, V_3) \rightarrow (T_\text{C}, \, V_4)$,
      \item 绝热压缩:$(T_\text{C}, \, V_4) \rightarrow (T_\text{H}, \, V_1)$,
    \end{myEnum2}
    这里的 $T_\text{H}$ 和 $T_\text{C}$ 分别指高温和低温,并且还有 $V_1 < V_2$,$V_4 < V_3$。
    
    \begin{figure}[h]
      \begin{asy}
        import graph;
        pair O = (0, 0), x_axes = (10, 0), y_axes = (0, 10);
        draw(Label("$V$", EndPoint), O--x_axes, Arrow);
        draw(Label("$p$", EndPoint), O--y_axes, Arrow);
        
        real gamma = 2.5;
        real x1 = 2, x4 = 9;
        real c1 = 10, c2 = 18;
        real c3 = c2*x1^(gamma-1), c4 = c1*x4^(gamma-1);
        real x2 = x1*(c1/c2)^(1/(1-gamma)), x3 = x4*(c2/c1)^(1/(1-gamma));
        
        path path1 = graph(new real(real x) {return c2/x;}, x1, x3);
        path path2 = graph(new real(real x) {return c4/x^gamma;}, x3, x4);
        path path3 = reverse(graph(new real(real x) {return c1/x;}, x2, x4));
        path path4 = reverse(graph(new real(real x) {return c3/x^gamma;}, x1, x2));
        //path path_TH = graph(new real(real x) {return c2/x;}, x3, 8);
        //path path_TC = graph(new real(real x) {return c1/x;}, 1.7, x2);
        
        pair p1 = (x1, c2/x1), p2 = (x3, c2/x3), p3 = (x4, c1/x4), p4 = (x2, c1/x2);
        
        pen pen1 = linewidth(1)+color1;
        
        fill(path1 & path2 & path3 & path4 & cycle, color1+opacity(0.2));
        
        //draw(path_TH, dashed + color1);
        //draw(path_TC, dashed + color1);
        draw(path1, pen1, Arrow(position = Relative(0.7), arrowhead = HookHead, size = 4));
        draw(path2, pen1, Arrow(position = Relative(0.5), arrowhead = HookHead, size = 4));
        draw(path3, pen1, Arrow(position = Relative(0.7), arrowhead = HookHead, size = 4));
        draw(path4, pen1, Arrow(position = Relative(0.45), arrowhead = HookHead, size = 4));
        
        draw(Label("$V_1$", EndPoint, black), p1--(p1.x, 0), dashed + color1);
        draw(Label("$V_2$", EndPoint, black), p2--(p2.x, 0), dashed + color1);
        draw(Label("$V_3$", EndPoint, black), p3--(p3.x, 0), dashed + color1);
        draw(Label("$V_4$", EndPoint, black), p4--(p4.x, 0), dashed + color1);
        
        label("状态1", p1, N);
        label("状态2", p2, NE);
        label("状态3", p3, E);
        label("状态4", p4, SW, Fill(white));
        
        label("I", path1, align = Relative(W));
        label("II", path2, align = Relative(W));
        label("III", path3, align = Relative(W));
        label("IV", path4, align = Relative(W), Fill(white));
      \end{asy}
      \caption{Carnot循环示意图}
      \label{FIG_CARNOT_CYCLE}
    \end{figure}
    
    利用热力学第一定律,有
    \begin{equation}
      \oint \dd U = \oint \db Q + \oint \db W = 0 \comma
    \end{equation}
    其中的“$\oint$”代表沿循环过程的积分。
    
    整个过程中对外做的净功(它等于图~\ref{FIG_CARNOT_CYCLE} 中曲线包围起来的面积)
    \begin{equation}
      W' = -\oint \db W = \oint \db Q = Q_\text{H} + Q_\text{C} \comma
    \end{equation}
    其中的 $Q_\text{H}$ 和 $Q_\text{C}$ 分别为高温和低温时吸收的热量(可以有正负)。
    
    若工作物质为理想气体,则有
    \begin{align}
      Q_\text{H} &= \incr U_\text{I} - W_\text{I} \myTag{热力学第一定律}\\
      &= 0 - W_\text{I} \notag \\
      &= \int_{V_1}^{V_2} p \dd V \notag \\
      &= n R T_\text{H} \int_{V_1}^{V_2} \frac{\dd V}{V} \myTag{根据 $p V = n R T$} \\
      &= n R T_\text{H} \ln \frac{V_2}{V_1}
      > 0 \label{EQ_Q_H_IN_CARNOT_CYCLE} \fullstop
    \end{align}
    同理,还有
    \begin{equation} \label{EQ_Q_C_IN_CARNOT_CYCLE}
      Q_\text{C} = -n R T_\text{C} \ln \frac{V_3}{V_4} < 0 \fullstop
    \end{equation}
    
    根据绝热过程的过程方程~\eqref{EQ_STATE_EQUATION_OF_ADIABATIC_PROCESS_IN_T_V},有
    \begin{mySubEq}
      \begin{empheq}[left=\empheqlbrace]{align}
        & T_\text{H} V_2^{\g - 1} = T_\text{C} V_3^{\g - 1} \comma \\
        & T_\text{H} V_1^{\g - 1} = T_\text{C} V_4^{\g - 1} \comma
    \end{empheq}
    \end{mySubEq}
    两边分别相除,得
    \begin{equation} \label{EQ_V2/V1_IN_CARNOT_CYCLE}
      \frac{V_2}{V_1} = \frac{V_3}{V_4} \fullstop
    \end{equation}
    
    定义\emphA{热机效率}
    \begin{equation}
      \h \eqdef \frac{W'}{Q_\text{H}} 
      = \frac{Q_\text{H} - \abs{Q_\text{C}}}{Q_\text{H}} 
      = 1 - \frac{\abs{Q_\text{C}}}{Q_\text{H}} \fullstop
    \end{equation}
    对于理想气体,代入式~\eqref{EQ_Q_H_IN_CARNOT_CYCLE} 和 \eqref{EQ_Q_C_IN_CARNOT_CYCLE},并利用式~\eqref{EQ_V2/V1_IN_CARNOT_CYCLE},可得
    \begin{align}
      \h &= 1 - \frac{T_\text{C}}{T_\text{H}} \frac{\ln (V_3 / V_4)}{\ln (V_2 / V_1)} \notag \\
      &= 1 - \frac{T_\text{C}}{T_\text{H}} \label{EQ_CARNOT_CYCLE_EFFICIENCY_WITH_IDEAL_GAS} \fullstop
    \end{align}%TODO:20160320 非理想气体的情况,见作业
    
    若Carnot循环反向进行,就成为\emphA{Carnot制冷机}。其\emphB{制冷效率}定义为
    \begin{equation}
      \ve = \frac{Q_\text{C}}{W}
      = \frac{Q_\text{C}}{Q_\text{H} - Q_\text{C}}
      = \frac{T_\text{C}}{T_\text{H} - T_\text{C}} \comma
    \end{equation}
    它通常是大于 $1$ 的。
    
  \subsection{热力学第二定律}
    \emphA{热力学第二定律}解决了有关过程\emphB{方向性}的问题,它的主要建立者有Carnot、Clausius、Kelvin等。
    
    \begin{myThm}{热力学第二定律(Kelvin表述)}
      不可能从单一热源吸热使之完全变为有用的功而不产生其他影响,即第二类永动机不可能实现。
    \end{myThm}
    \begin{myThm}{热力学第二定律(Clausius表述)}
      不可能把热量从低温物体传到高温物体而不产生其他影响。
    \end{myThm}
    
    \begin{myProof}%TODO:20160318 图片
      Kelvin表述 $\implies$ Clausius表述:
      
      采用反证法,即证明 $\neg \, (\text{Clausius表述}) \implies \neg \, (\text{Kelvin表述})$。
      
      Carnot热机A工作于高温热源 $T_\text{H}$ 和低温热源 $T_\text{C}$ 之间。它从 $T_\text{H}$ 处吸收热量 $Q_\text{H}$,向 $T_\text{C}$ 放出热量 $Q_\text{C}$,并做功 $W = Q_\text{H} - Q_\text{C}$。假设Clausius表述不成立,就可以在不产生其他影响的前提下,使低温热源获得的热量 $Q_\text{C}$ 重新回到高温热源。净结果便是从单一热源 $T_\text{H}$ 吸收了热量 $Q_\text{H} - Q_\text{C}$,并将其完全转化为功,这就违背了Kelvin表述。因此原假设不成立,即有 $\neg \, (\text{Clausius表述}) \implies \neg \, (\text{Kelvin表述})$。
      
      \blankline
      
      Clausius表述 $\implies$ Kelvin表述:
      
      同样采用反证法,即证明 $\neg \, (\text{Kelvin表述}) \implies \neg \, (\text{Clausius表述})$。
      
      假设Kelvin表述不成立,就可以在不产生其他影响的前提下,从单一热源 $T_\text{H}$ 吸热 $Q_\text{H}$ 并将其完全转化为有用功 $W = Q_\text{H}$。它可以推动Carnot制冷机从低温热源 $T_\text{CH}$ 吸收 $Q_\text{C}$ 的热量并传给高温热源 $Q_\text{H} + Q_\text{C}$ 的热量。净结果是热量 $Q_\text{C}$ 从低温热源传给了高温热源,却没有产生其他影响,这就违背了Clausius表述。因此原假设不成立,即有 $\neg \, (\text{Kelvin表述}) \implies \neg \, (\text{Clausius表述})$。
    \end{myProof}
    
    热力学第二定律的核心内容可以概括为:自然界一切热现象过程都是不可逆的。
    
\section{热力学第二定律的数学解释;熵}
  \subsection{Carnot定理}
    \begin{myThm}{Carnot定理}
      工作于两个确定温度之间的所有热机中,可逆热机效率最高。
    \end{myThm}
    
    设两个热机A、B工作于高温热源 $\th_\text{H}$ 和低温热源 $\th_\text{C}$ 之间\footnote{
      这里用 $\th$ 表示温度,而不是像前文一样使用 $T$,原因见下一小节。
    },它们分别从 $\th_\text{H}$ 吸收 $Q_\text{H,\,A}$ 与 $Q_\text{H,\,B}$ 的热量,向 $\th_\text{C}$ 放出 $Q_\text{C,\,A}$ 与 $Q_\text{C,\,B}$ 的热量,并对外做功 $W_\text{A}$ 与 $W_\text{B} \,$\footnote{
      前文用撇号表示系统(热机)对外界做功,这里方便起见直接用 $W$。但需注意,$W_\text{A}$ 与 $W_\text{B}$ 均大于 $0$。
    }。根据定义,其效率分别为
    \begin{equation}
      \h_\text{A} = \frac{W_\text{A}}{Q_\text{H,\,A}} \comma \, \h_\text{B} = \frac{W_\text{B}}{Q_\text{H,\,}} \fullstop
    \end{equation}
    设A是一个可逆热机,因此我们把 $\h_\text{A}$ 写成 $\h_\text{rev,\,A}$。根据Carnot定理,有
    \begin{equation}
      \h_\text{rev,\,A} \geqslant \h_\text{B} \fullstop
    \end{equation}
    
    \begin{myProof}
      下面利用反证法证明Carnot定理,即假设 $\h_\text{rev,\,A} < \h_\text{B}$。因此
      \begin{equation}
        \frac{W_\text{A}}{Q_\text{H,\,A}} < \frac{W_\text{B}}{Q_\text{H,\,B}} \fullstop
      \end{equation}
      令A、B从高温热源 $\th_\text{H}$ 处吸收相同的热量,即 $Q_\text{H,\,A} = Q_\text{H,\,B}$,那么就有 $W_\text{A} < W_\text{B}$。因为A是可逆热机,所以不妨让B热机输出功的一部分 $W_\text{A}$ 推动A热机逆向运行(此时A就是一个制冷机)。此时,B热机还可以输出功 $W_\text{B} - W_\text{A}$。
      
      根据热力学第一定律,有
      \begin{mySubEq}
        \begin{empheq}[left=\empheqlbrace]{align}
          & W_\text{A} = Q_\text{H,\,A} - Q_\text{C,\,A} \comma \\
          & W_\text{B} = Q_\text{H,\,B} - Q_\text{C,\,B} \comma
        \end{empheq}
      \end{mySubEq}
      因此
      \begin{equation}
        W_\text{B} - W_\text{A} = Q_\text{C,\,A} - Q_\text{C,\,B} \fullstop
      \end{equation}
      
      若A、B联合运行,其净结果便是从低温热源 $\th_\text{C}$ 处吸收 $Q_\text{C,\,A} - Q_\text{C,\,B}$ 的热量,并对外做了 $W_\text{B} - W_\text{A}$ 的功,即在不产生其他影响的情况下完全把热转化为了功。这显然违背了热力学第二定律的Kelvin表述。因此原假设不成立,于是Carnot定理得证。
    \end{myProof}
    
    由Carnot定理,可以得到如下推论:
    \begin{myThm*}
      所有工作于两个确定温度之间的可逆热机效率均相等。
    \end{myThm*}
    
  \subsection{热力学温标}
    \emphA{温标(scale of temperature)},是以量化数值,配以温度单位来表示温度的方法。它包含三个要素:
    \begin{myEnum2}
      \item \emphB{测温质}与\emphB{测温参量};
      \item 测温参量与温度的\emphB{函数关系};
      \item \emphB{温度标准点}的选定。
    \end{myEnum2}
    
    常用的经验温标有摄氏温标、华氏温标等。利用理想气体状态方程,可以定义\emphA{理想气体温标}:
    \begin{equation}
      T \eqdef \frac{1}{n R} \lim\limits_{p \approach 0} p V \comma
    \end{equation}
    同时需要规定水的三相点温度 $T_\text{tr} \eqdef \SI{273.16}{\kelvin}$。
    
    在 \secref{sec:理想气体与Carnot循环_热力学第二定律} \subsecref{subsec:Carnot循环} 中,我们使用 $T$ 表示温度。实际上,那里的“温度”是用\emphB{理想气体温标}表示的值。
    
    \blankline
    
    根据Carnot定理,可逆热机的效率只与两个热源的温度有关,而与工作物质的性质、吸放热多少、做功多少均无关。因此,可逆热机的效率是两个温度 $\th_\text{H}$、$\th_\text{C}$ 的\emphB{普适函数}。根据定义,热机的效率
    \begin{equation}
      \h = \frac{W}{Q_\text{H}} = 1 - \frac{Q_\text{C}}{Q_\text{H}} \fullstop
    \end{equation}
    因此有
    \begin{equation}
      \frac{Q_\text{C}}{Q_\text{H}} = F(\th_\text{H}, \, \th_\text{C}) \comma
    \end{equation}
    其中的 $F(\th_\text{H}, \, \th_\text{C})$ 是 $\th_\text{H}$ 与 $\th_\text{C}$ 的普适函数。
    
    下面证明
    \begin{equation}
      F(\th_\text{H}, \, \th_\text{C}) = \frac{f(\th_\text{C})}{f(\th_\text{H})} \comma
    \end{equation}
    其中的 $f$ 是另一个普适函数。%TODO:20160322 热力学温标证明
    
    由式~\eqref{EQ_CARNOT_CYCLE_EFFICIENCY_WITH_IDEAL_GAS},理想气体Carnot热机效率为
    \begin{equation}
      1 - \frac{T^*_\text{C}}{T^*_\text{H}} \comma
    \end{equation}
    这里用带 $^*$ 的 $T$ 表示理想气体温标下的温度。这与式 是相同的,即温度尺度相同。又因为理想气体温标也规定在水的三相点处 $T^*_\text{tr} = \SI{273.16}{\kelvin}$,因此,理想气体温标与热力学温标是相同的。
    
  \subsection{Clausius不等式}
    根据Carnot定理,工作于两个确定温度之间的所有热机,其效率均满足
    \begin{equation}
      \h = 1 - \frac{Q_2}{Q_1} \leqslant 1 - \frac{T_2}{T_1} \comma \footnote{
        以后均直接用 $T$ 表示温度。
      }
    \end{equation}
    对于可逆热机,取等号;对于不可逆热机,则取小于号。
    
    上式稍作变形,可得
    \begin{align}
      &\mathrel{\phantom{\implies}} \frac{Q_2}{Q_1} \geqslant \frac{T_2}{T_1} \\
      &\implies \frac{Q_1}{T_1} - \frac{Q_2}{T_2} \leqslant 0 \fullstop
    \end{align}
    约定 $Q$ 始终表示吸收的热量,则放热应写作 $-Q$。于是
    \begin{equation}
      \frac{Q_1}{T_1} + \frac{Q_2}{T_2} \leqslant 0 \fullstop
    \end{equation}
    假设系统先后与温度分别为 $T_1, \, T_2 \, \dots, \, T_n$ 的 $n$ 个热源接触,又分别吸热 $Q_1, \, Q_2 \, \dots, \, Q_n$,则可以证明\emphA{Clausius不等式}:
    \begin{equation}
      \iTonSum \frac{Q_i}{T_i} \leqslant 0 \fullstop
    \end{equation}
    
    \begin{myProof}
      设有程%TODO:20160322 证明过程没写
    \end{myProof}
    
    在 $n \approach \infty$ 的极限下,Clausius不等式过渡到积分形式:
    \begin{equation} \label{EQ_CLAUSISU_INEQUALITY_IN_INTEGRAL}
      \lim\limits_{n \approach \infty} \iTonSum \frac{Q_i}{T_i} \leqslant 0 \approach \oint \frac{\db Q}{T} \leqslant 0 \fullstop
    \end{equation}
    
  \subsection{熵的定义}
    对于可逆循环,根据式~\eqref{EQ_CLAUSISU_INEQUALITY_IN_INTEGRAL},有
    \begin{equation}
      \oint \frac{\db Q_\text{rev}}{T} = 0 \fullstop
    \end{equation}
    如图%TODO:20160323 图片
    可以表示成两段路径之和:
    \begin{equation}
      \underset{C_1\phantom{M}}{\int_{(P_0)}^{(P)}} \, \frac{\db Q_\text{rev}}{T}
      + \underset{C_2\phantom{M}}{\int_{(P)}^{(P_0)}} \, \frac{\db Q_\text{rev}}{T} = 0 \comma
    \end{equation}
    即
    \begin{equation}
      \underset{C_1\phantom{M}}{\int_{(P_0)}^{(P)}} \, \frac{\db Q_\text{rev}}{T}
      = \underset{C_2\phantom{M}}{\int_{(P_0)}^{(P)}} \, \frac{\db Q_\text{rev}}{T} = \const
    \end{equation}
    可以看出,$\db Q_\text{rev} / T$ 是一个与路径无关的量。由此,定义一个新的状态函数——\emphA{熵(entropy)}:
    \begin{equation}
      S - S_0 = \int_{(P_0)}^{(P)} \, \frac{\db Q_\text{rev}}{T} \fullstop
    \end{equation}
    
  \subsection{不可逆过程的数学表述}
    \begin{myEnum1}
      \myItem{初终态均是平衡态}
        根据Clausius不等式,有
        \begin{align}
          &\mathrel{\phantom{\implies}} \underset{\text{irrev} + \text{rev}}{\oint} \frac{\db Q}{T} < 0 \notag \\
          &\implies \int_{(P_0)}^{(P)} \, \frac{\db Q_\text{irrev}}{T} + \int_{(P)}^{(P_0)} \, \frac{\db Q_\text{rev}}{T} < 0 \notag \\
          &\implies S - S_0 > \int_{(P_0)}^{(P)} \, \frac{\db Q_\text{irrev}}{T} \fullstop
        \end{align}
        
      \myItem{初终态均是非平衡态}
        采用\emphB{局域平衡近似},仍旧可以推得
        \begin{equation}
          S - S_0 > \int_{(P_0)}^{(P)} \, \frac{\db Q_\text{irrev}}{T} \fullstop
        \end{equation} %TODO:20160323 局域平衡近似的证明
    \end{myEnum1}
    
    \blankline
    
    把对可逆过程与不可逆过程的表述合起来,就有
    \begin{equation} \label{EQ_2ND_LAW_IN_INTEGRAL_FORM}
      \incr S = S - S_0 \geqslant \int_{(P_0)}^{(P)} \, \frac{\db Q}{T} \semicomma
    \end{equation}
    写成微分形式,为
    \begin{boxedEq} \label{EQ_2ND_LAW_IN_DIFFERENTIAL_FORM}
      \dd S \geqslant \frac{\db Q}{T}
    \end{boxedEq}
    以上两式中,“$=$”适用于可逆过程,“$>$”适用于不可逆过程。这两式实际上便是热力学第二定律的数学表述。
    
  \subsection{熵的性质}
    这里小结一下熵的性质。
    
    \begin{myEnum2}
      \item 熵是\emphB{状态函数}。
      \item 熵是\emphB{广延量}。
      \item 对微小的\emphB{可逆}过程,$\dd S = \db Q / T$。因此有
      \begin{equation} \label{EQ_dQ=TdS}
        \db Q = T \dd S \fullstop
      \end{equation}
      对于\emphB{绝热}过程,有 $\db Q = 0$,因此
      \begin{equation} \label{EQ_dS=0_FOR_ADIABATIC_REVERSIBLE_PROCESS}
        \dd S = 0 \fullstop
      \end{equation}
    \end{myEnum2}%TODO:20160323 卡诺循环的TS表述
    
  \subsection{热力学基本方程}
    热力学第一定律式~\eqref{EQ_1ST_LAW_IN_DIFFERENTIAL_FORM}:
    \begin{equation}
      \dd U = \db Q + \db W \semicomma
    \end{equation}
    由热力学第二定律,得可逆过程微热量的表达式 \eqref{EQ_dQ=TdS}:
    \begin{equation}
      \db Q = T \dd S \semicomma
    \end{equation}
    微功的一般表示式~\eqref{EQ_GENERAL_WORK}:
    \begin{equation}
      \db W = \iTonSum Y_i \dd y_i \fullstop
    \end{equation}
    联立以上三式,可得
    \begin{equation} \label{EQ_FUNDAMENTAL_EQUATION_OF_THERMODYNAMICS}
      \dd U = T \dd S + \iTonSum Y_i \dd y_i \fullstop
    \end{equation}
    这就是\emphA{热力学基本微分方程}。
    
    对于 $p\text{-}V\text{-}T$ 系统,上式可简化为
    \begin{equation} \label{EQ_FUNDAMENTAL_EQUATION_FOR_PVT_SYSTEM}
      \dd U = T \dd S - p \dd V \fullstop
    \end{equation}
    
    \begin{myExample}[理想气体的熵]
      下面推导不同过程下理想气体的熵。
      \begin{myEnum1}
        \myItem{等容过程}
          根据式~\eqref{EQ_dU=Cv*dT_IDEAL_GAS},有
          \begin{equation}
            \dd U = C_V \dd T \fullstop
          \end{equation}
          根据热力学基本微分方程式~\eqref{EQ_FUNDAMENTAL_EQUATION_FOR_PVT_SYSTEM},
          \begin{align}
            &\mathrel{\phantom{\implies}} T \dd S = \dd U + p \dd V \notag \\
            &\implies \dd S = \frac{\dd U}{T} + \frac{p \dd V}{T} \notag \\
            &\implies S = \frac{C_V}{T} \dd T + \frac{}{}
          \end{align}%TODO:20160323 有问题?
        
        \myItem{等压过程}
          没写%TODO:20160330 没写
        \myItem{等温过程}
      \end{myEnum1}
    \end{myExample}
  
\section{熵增加原理;最大功} \label{sec:熵增加原理与最大功}
  \subsection{熵增加原理}
    根据热力学第二定律[式~\eqref{EQ_2ND_LAW_IN_INTEGRAL_FORM}],
    \begin{equation}
      \incr S \geqslant \int_{\text{I}}^{\text{II}} \frac{\db Q}{T} \fullstop
    \end{equation}
    对于\emphB{绝热}过程(或孤立体系),有 $\db Q = 0$。因此
    \begin{equation} \label{EQ_PRINCIPLE_OF_ENTROPY_INCREASE}
      \incr S \geqslant 0 \fullstop \footnote{
        注意与 \eqref{EQ_dS=0_FOR_ADIABATIC_REVERSIBLE_PROCESS} 式对比,它还要求\emphA{可逆}过程。
      }
    \end{equation}
    这就是\emphA{熵增加原理},它说明绝热体系的熵永不减少。
  \subsection{不可逆过程的熵变}
    没写%TODO:20160323 没写
  \subsection{最大功}
    根据热力学第一定律[式~\eqref{EQ_1ST_LAW_IN_DIFFERENTIAL_FORM}],
    \begin{equation}
      \dd U = \db Q + \db W \fullstop
    \end{equation}
    令 $\db W' = - \db W$ 为系统对外界做的功,则
    \begin{equation} \label{EQ_dU=dQ-dW'_IN_SECTION_MAX_WORK}
      \db W' = \db Q - \dd U \fullstop
    \end{equation}
    根据热力学第二定律[式~\eqref{EQ_2ND_LAW_IN_DIFFERENTIAL_FORM}],
    \begin{equation}
      \db Q \leqslant T_\text{e} \dd S \fullstop
    \end{equation}
    代入式~\eqref{EQ_dU=dQ-dW'_IN_SECTION_MAX_WORK},可得
    \begin{equation}
      \db W' \leqslant T_\text{e} \dd S - \dd U \fullstop
    \end{equation}
    因此系统对外做的\emphA{最大功}为
    \begin{equation}
      \db W'_{\text{max}} = \db W'_{\text{rev}} = T \dd S - \dd U \comma
    \end{equation}
    这里的 $T = T_\text{e}$ 为系统的温度(因为是可逆过程)。
    
    对于不可逆过程,显然有
    \begin{equation}
      \db W'_{\text{irrev}} < \db W'_{\text{rev}} \fullstop
    \end{equation}
    
    \begin{myExample}[水的混合]
      两杯等量的水初始温度分别为 $T_1$、$T_2$。在等压、绝热条件下将它们混合均匀,求该过程的熵变。%TODO:20160323 没写
      %TODO:20160330 T取平均值:假设热容为常数
    \end{myExample}
    
    \begin{myExample}[制冷机所需的最小功]
      两物体初始温度均为 $T_1$。一台制冷机工作于其间,使一物体温度升高至 $T_2$。假设这是一个等压过程,并且不考虑相变。证明:制冷机所需的最小功
      \begin{equation}
        W_{\text{min}} = C_p \left( \frac{T_1^2}{T_2} + T_2 - 2 T_1 \right) \fullstop%TODO:20160323 没写
      \end{equation}
    \end{myExample}
  
\section{自由能与Gibbs函数}
%	根据熵增加原理[式~\eqref{EQ_PRINCIPLE_OF_ENTROPY_INCREASE}],对于孤立系统,有
%	\begin{equation}
%		\incr S \geqslant 0 \fullstop
%	\end{equation}
  \subsection{自由能}
    考虑这样的\emphB{等温过程}:热源维持恒定温度 $T$;系统初终态温度 $T_1$、$T_2$ 与热源温度相同,即 $T_1 = T_2 =T$。对于可逆过程,在全程中系统温度均为 $T$;而对于不可逆过程,仅满足 $T_1 = T_2 =T$。
    
    由Clausius不等式%[\eqref{}]
    \begin{equation}
      \incr S = S_2 - S_1 \geqslant \int_{\text{I}}^{\text{II}} \frac{\db Q}{T} = \frac{1}{T} \int_{\text{I}}^{\text{II}} \db Q = \frac{Q}{T}\comma 
    \end{equation}
    即
    \begin{equation}
      Q \leqslant T (S_2 - S_1) \fullstop
    \end{equation}
    根据热力学第一定律[式~\eqref{EQ_1ST_LAW_IN_INTEGRAL_FORM}],
    \begin{equation}
      U_2 - U_1 = W + Q \comma
    \end{equation}
    因此
    \begin{align}
      -W &= (U_1 - U_2) + Q \notag \\
      &\leqslant (U_1 - U_2) - T (S_2 - S_1) \notag \\
      &=(U_1 - T S_1) - (U_2 - T S_2) \fullstop
    \end{align}
    定义\emphA{自由能} $F = U - T S$,则
    \begin{equation}
      -W \leqslant F_1 - F_2 \fullstop
    \end{equation}
    
    如果该过程除了保持等温,还保持等容,即 $W = 0$,则有%TODO:20160325 怎么会有等温等容?
    \begin{equation}
      \incr F = F_2 - F_1 \leqslant 0 \comma
    \end{equation}
    这说明在等温等容过程中,系统向自由能减小的方向前进。
    
    自由能具有以下的性质:
    \begin{myEnum2}
      \item 态函数 %TODO:没写
    \end{myEnum2}
  \subsection{Gibbs函数}
    考虑等温等压过程